\lecture{2}{13-06-2023}{Das Gravitationsgesetz (Labor)}


Hier noch bild vonder simulation.


\begin{table}[htbp]
    \centering
    \begin{tabular}{c|c}
        \toprule
        $m_1 100kg$ & $1.34\cdot 10^{-7}$ \\ 
        \midrule
        $m_1 200kg$ & $2.67\cdot 10^{-7}$ \\ 
        \midrule
        $m_1 300kg$ & $4.00\cdot 10^{-7}$ \\ 
        \midrule
        $m_1 400kg$ & $5.34\cdot 10^{-7}$  \\ 
        \midrule
        $m_1 500kg$ & $6.67\cdot 10^{-7}$  \\ 
        \midrule
        $m_1 600kg$ & $8.01\cdot 10^{-7}$  \\ 
        \midrule
        $m_1 700kg$ & $9.34\cdot 10^{-7}$  \\ 
        \midrule
        $m_1 800kg$ & $10.70\cdot 10^{-7}$  \\ 
        \midrule
        $m_1 900kg$ & $12.00\cdot 10^{-7}$  \\ 
        \midrule
        $m_1 1000k$g & $13.40\cdot 10^{-7}$  \\ 
        \bottomrule
    \end{tabular}
    \caption{Versuch 1, $m_2 = 100kg$}
\end{table}

\begin{figure}[htbp]
    \centering
\begin{tikzpicture}
    \begin{axis}[
        axis lines=middle,
        xmin=0, xmax=1000,
        ymin=0, ymax=15,
        xtick=\empty, ytick=\empty ]
        \addplot [only marks] table {
        100  1.34
        200  2.67
        300  4.00
        400  5.34
        500  6.67
        600  8.01
        700  9.34
        800  10.70
        900  12.00
        1000 13.40
        };
    \end{axis}
\end{tikzpicture}
\caption{Messwerte aus der Simulation}
\end{figure}

Hier Diagram mit tikz.

Aus Regression: \[
0.0133751515 x -0.009333333 \left( r^2 = 0.999983057 \right) 
.\] 

Hier noch proportionalitaet.


Versuch 2


\begin{table}[htbp]
    \centering
    \begin{tabular}{c|c}
        \toprule
        $m_2$  100kg & $1.34\cdot 10^{-7}$ \\ 
        \midrule
        $m_2$  200kg & $2.67\cdot 10^{-7}$ \\ 
        \midrule
        $m_2$  300kg & $4.00\cdot 10^{-7}$ \\ 
        \midrule
        $m_2$  400kg & $5.34\cdot 10^{-7}$  \\ 
        \midrule
        $m_2$  500kg & $6.67\cdot 10^{-7}$  \\ 
        \midrule
        $m_2$  600kg & $8.01\cdot 10^{-7}$  \\ 
        \midrule
        $m_2$  700kg & $9.34\cdot 10^{-7}$  \\ 
        \midrule
        $m_2$  800kg & $10.70\cdot 10^{-7}$  \\ 
        \midrule
        $m_2$  900kg & $12.00\cdot 10^{-7}$  \\ 
        \midrule
        $m_2$  1000kg & $13.40\cdot 10^{-7}$  \\ 
        \bottomrule
    \end{tabular}
    \caption{Versuch 2, $m_2 = 500kg$}
\end{table}

Haegnt ab mit dem Faktor der aus der ableitung der Regression kommt.

DIGRAM EINFUEGEN.

Regression machjen



Versuch 3

\begin{table}[htbp]
    \centering
    \begin{tabular}{c|c}
        \toprule
        2m & $4.17\cdot 10^{-6}$ \\
        3m & $1.85\cdot 10^{-6}$ \\
        4m & $1.04\cdot 10^{-6}$ \\
        5m & $6.67\cdot 10^{-7}$ \\
        6m & $4.64\cdot 10^{-7}$ \\
        7m & $3.41\cdot 10^{-7}$ \\
        8m & $2.61\cdot 10^{-7}$ \\
        9m & $2.06\cdot 10^{-7}$ \\
        10m &$ 1.67\cdot 10^{-7}$ \\
        \bottomrule
    \end{tabular}
    \caption{Versuch 3, $m_2 = m_2= 500kg$}
\end{table}

Aus exponentieller Regression: 
\[ reg = 57e^{-0.38x} .\] 

%Aus der Regression mit $r^2=0.95$ kann man sehen es hat die asymptote nach
%$\infty$ von x=0. und eine gegen 0 in $\to x=\infty$


ABBILDUNG EINFUEGEN $\frac{1}{r^2}-F_G$ Diagramm im vergleich zu dem exponentiellen.

Ist proportional zueinander (begruenden) .

GLEICHUNG AUFSTELLEN



Versuch 4

Tabelle ausfuellen, $F_G$ mit Simulation.

