\lecture{3}{16-06-2023}{Untericht 13-Juni}
\section{Gravitationsgesetz}
\tbox{
\[
F_G=G\cdot \frac{M_1\cdot M_2}{r^2}
.\] 
$M_1 \ \& \ M_2$ sind die Massen der K\"orper, $G$ ist die Gravitationskonstante
und $r^2$ als Abstand der Schwerpunkte.

\[
G=6,674\cdot 10^{-11}m^3\cdot kg^{-1}\cdot 5^{-2}
.\] 
}

\begin{figure}[ht]
    \centering
    \incfig[0.6]{gravitationsgesetz}
    \caption{Gravitationsgesetz}
    \label{fig:gravitationsgesetz}
\end{figure}

\ex{\textbf{aufgabe 3c) 2.}

Bestimmen Sie die Masse der Erde jeweils ausschlieslich mit den
    folgenden Werten:

Abstand Erde - Mond $d=60\cdot R_{Erde}$ und Umlaufdauer des Mondes t=27,3d.

\vspace{10pt}
\hrule
\vspace{10pt}

Ansatz: $F_Z = F_G$ (Zentripeltalkraft)
 \begin{align*}
    m\cdot \frac{v^2}{d}&= G\cdot \frac{M\cdot m}{d^2} \\
    m\cdot \frac{\left( \frac{2\pi\cdot d}{t} \right)^2}{d}&= G\cdot \frac{M\cdot m }{d^2} \quad \text{hier kann man masse vom mond (m) k\"urzen}\\
    M&= \frac{\frac{4\pi^2}{t^2}\cdot d^2}{d\cdot G}= \frac{4\pi^2\cdot d^3}{G*t^2} \\
    M&= \frac{4\cdot \pi^2\cdot 60\cdot (6,370\cdot 10^{6}m)^3}{6,674\cdot 10^{-11}m^3\cdot kg^{-1}s^{-2}\cdot (27,3\cdot 60\cdot 60\cdot 24s)^2} \\
    M&= 6,068\cdot 10^{24}kg \quad \text{in echt ist es } 5,79\cdot 10^{24}kg \\
\end{align*}
}

\subsection{Aufaben}
\paragraph{Aufgabe 1} Berechnen sie die Gravitationskraft zwischen:\\

\begin{enumerate}
    \item Zwei Schiffen mit einer Masse von je 100 000 Tonnen, die sich mit dem Schwerpunktabstand r=200m begegnen.
    Hier ist irgendeine rechnung dann $F_G=6,674\cdot
    10^{-11}m^3kg^{-1}s^{-2}\cdot \frac{100 000 000 kg}{200m^2} = 1,6685\cdot 10$.

    \item Ein Raumschiff umkreist die Erde in einer H\"ohe 2000km zur Erdoberfleche. Berechnen sie die dafuer erforderliche Geschwindigkeit.

        Hier muss man dann den Radius der Erde + den Abstand nehmen f\"ur die Zentripetalkraft.
\end{enumerate}


