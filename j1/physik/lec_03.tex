\clearpage
\lecture{4}{30-06-2023}{Stunde vor der Arbeit}

\regel{Konzepte:\\
    \begin{enumerate}
        \item \[ F_{Res}=m\cdot a .\] 
        \item Energieerhaltungssatz erg\"anzt durch "Arbeit (\"Ubertragene Energie)"
    \end{enumerate}
}

\subsection{Tafelaufschrieb}

\begin{multicols}{2}[Im folgenden sind immer die Inhalte der einzelnen Tafeln in boxen.]

    \tbox{
        \[ F_{el}=q\cdot \frac{U}{d} .\] 
        Hier ist $e=1,6\cdot 10^{-19}C$

        \[ F_G=m\cdot g .\] 
        Hier ist $m=9,1\cdot 10^{-31}kg$ und $g=9,81 \frac{m}{s^2}$ 

    \[ 10^{-15}=10^{-30}\cdot 10^x .\] }


    \tbox{
        \begin{align*}
            F_{res}&= m_e\cdot a \\
            F_{el}&= m_e\cdot a \\
            e\cdot \frac{U}{d}&= m_e\cdot a \\
            a&= \frac{a}{m_e}\cdot \frac{U}{d} \\
            a&= 3,6\cdot 10^{14}\cdot g \\
        \end{align*}
    Hier ist $\frac{e}{m_e}$ die spezifische ladung des elektrons }


    \tbox{
        \[ v(t)=a\cdot t .\] 
        Bringt uns aber nicht vor, weil wir zeit wollen und geschwindigkeit nicht brauchen.

        \[ s(t)=\frac{1}{2}a\cdot t^2 .\] 

        \begin{align*}
        \Leftrightarrow t&=\sqrt{\frac{2s}{a}}\\
        v&=1,9\cdot 10^7 \frac{m}{s}\\
        \frac{1}{2}mv^2&= F_{el}\cdot d \\
        \frac{1}{2}m_e\cdot v^2&= e\cdot u \\
        \end{align*}
    }

    \regel{
        \begin{align*}
            W&= e\cdot U \\
            &= e\cdot 1000V \\
            &= 1000 eV \\
            a&= \frac{\Delta W}{q} \\
        \end{align*}
        Elektronenvolt }

\end{multicols}

\subsection{Aufgabe elektrisches Feld}
geg.: Abstand-P-Q=8cm, Spannung $U_2=1,2$kV\\
ges.: Entefernung in der $P$ umkehrt.
\begin{align*}
    v&= \sqrt{v^2_0+2as}  \\
    v^2&= v_0^2 + 2as \\
    v^2-v_0^2&= 2as \\
    s&= \frac{v^2-v_0^2}{2a} \\
    s&= \frac{-v_0^2}{2a} \\
    &= \frac{-(1,9\cdot 10^7)^2}{2\cdot 2,64\cdot 10^{15}} \\
    s&= 6.84\ cm \\
\end{align*}

$v_0$ ist die initiale Kraft.

\tbox{ Kreisbewegung, Gravitationsfeld, Teilchen im E-Feld}


\subsection{Elektronenstrahlr\"ohre}
Lernen, wie es funktioniert, was wo steht, und wie es funktionert.
Mit Elektronen Kanone, Anode, Vertikal-/Horizontalablenktung...

HIER EVTL NOCH BILD IMPORTIEREN.

\subsubsection{Bahngleichung}

Wir interpretieren die Bewegung der Elektronen als \"Uberlagerung zweier
Teilbewegungen:
\begin{enumerate}
    \item Bewegung mit konstanter Geschwindigkeit in x-richtung
        \begin{align}
            v(t)&=v_x =\quad  konstant \quad  \text{(denn $F_{res, x}$=0!)}\\
            x(t)&=v_x\cdot t 
        \end{align}
    \item Bewegung mit konstanter Beschleunigung in y-Richtung
        \begin{align}
            v(t)&=a\cdot t  \\
            y(t)&=\frac{1}{2}a\cdot t^2
        \end{align}
\end{enumerate}


(1) in (3)
\begin{align*}
    y(t)&= \frac{1}{2}\cdot a\cdot \left( \frac{x(t)}{v_x}\right)  \\
    \implies y &= \frac{a}{2\cdot v_x^2}\cdot x^2
\end{align*}



