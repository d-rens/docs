\lecture{4}{27-06-2023}{Integralfunktionen}

\ex{
    Hier geht es glaube ich darum, dass man aufgeleitete Funktionen wieder
    ableiten kann um die ``normale'' Funktion $f(x)$ zu bekommen.
    \[ f(x)=2x-3 \] 
    \begin{align*}
        J_1(x)&=\int_{1}^{x} f(t)\ dt \\
              &= \int_{1}^{x} (2t-3)\ dt   \\
              &= [t^2=3t]_{1}^{x} = x^2=3\cdot x=(1^2-3\cdot 1) \\
              &= x^2-3x+2 \\ 
              &\\
       J'_1(x)&= 2x-3 = f(x) \\
    \end{align*}

    $\implies J_1$ ist eine Stammfunktion von $f$.
}

\q{Ist die Stammfunktion $F$ mit \[
F(x) = x^2-3x+3 \] 
Integralfunktion von $f$? }

W\"are $F$ Integralfunktion von $f$, so g\"abe es ein $a \in \R$ mit

\begin{align*}
       F(x)&=\int_{a}^{x} f(t)\ dt \\
            &= x^2-3x-(a^2-3a) \\
        -a^2+3a &= 3 \\
        a^2-3a+3 &=0 \\
        a_{1/2}&= \frac{3}{2}\pm \sqrt{(\frac{3}{2})^2-3}  \\
        & \text{ NICHT m\"oglich zu rechnen}  \\
\end{align*}

\regel{Satz: Jede Integralfunktion $J_a$ von $f$ ist eine Stammfunktion von $f$.}
Dir Umkehrung des Satzes gilt nicht, d.h. nicht jede Stammfunktion ist auch Integralfunktion.

\nt{Hier noch die Abbildungen f\"ur die Integralfunktionen einf\"ugen.}




