\usepackage[a4paper,margin=3.5cm]{geometry}
\usepackage{cancel}
\usepackage{tikz}
\usepackage{amsmath,amsfonts,amssymb, amsthm}
\usepackage{xcolor}
\usepackage{tcolorbox}
\usepackage{polynom}
\usepackage{wrapfig}
\usepackage{booktabs}
\usepackage{tabularx}
\usepackage{multicol}
%\usepackage{hyperref}
%\usepackage{babel}
%\usepackage{titlesec}
\usepackage{tikz}
\usepackage{pgfplots}
\pgfplotsset{compat=1.18}

\usepackage{pstricks}
\usepackage{pst-plot}
\usepackage{textcomp}
\usepackage{import}
\usepackage{pdfpages}
%\usepackage{transparent}


\newcommand{\incfig}[2][1]{%
    \def\svgwidth{#1\columnwidth}
    \import{./figures/}{#2.pdf_tex}
}

%\setlength{\parindent}{0pt}
\usepackage{parskip}


% makes arrows shorter
\let\implies\Rightarrow
\let\impliedby\Leftarrow
\let\iff\Leftrightarrow
\let\epsilon\varepsilon

% things i'll surly need somewhen, but not yet:
\newcommand\N{\ensuremath{\mathbb{N}}}
\newcommand\R{\ensuremath{\mathbb{R}}}
\newcommand\Z{\ensuremath{\mathbb{Z}}}
\renewcommand\O{\ensuremath{\emptyset}}
\newcommand\Q{\ensuremath{\mathbb{Q}}}
\newcommand\C{\ensuremath{\mathbb{C}}}


% defining chapter so i can use it in non-book classes
%\titleformat{\chapter}[display]
  %{\normalfont\huge\bfseries}{\chaptertitlename\ \thechapter}{20pt}{\Huge}
%\titlespacing*{\chapter}{0pt}{50pt}{40pt}
%%\newcommand{\chapterbreak}{\clearpage}



% nice looking dices for probability
\font\domino=domino
\def\die#1{{\domino#1}}



% Boxes
\newcommand{\tbox}[2][0.8\linewidth]{
    \begin{center}
        \begin{tcolorbox}[colback=white, colframe=gray, width=#1]
            #2
        \end{tcolorbox}
\end{center}}

\newcommand{\regel}[2][0.8\linewidth]{
    \begin{center}
        \begin{tcolorbox}[title=Regel, colback=white, colframe=blue, width=#1]
            #2
        \end{tcolorbox}
\end{center}}

\newcommand{\nt}[2][0.9\linewidth]{
    \begin{center}
        \begin{tcolorbox}[title=Notiz:, colback=white, colframe=gray, width=#1]
            #2
        \end{tcolorbox}
\end{center}}


\newcommand{\ex}[2][\linewidth]{
    \begin{center}
        \begin{tcolorbox}[title=Example, colback=white, colframe=brown, width=#1]
            #2
        \end{tcolorbox}
\end{center}}


\newcommand{\q}[2][\linewidth]{
    \begin{center}
        \begin{tcolorbox}[title=Frage:, colback=white, colframe=purple, width=#1]
            #2
        \end{tcolorbox}
\end{center}}


\newcommand{\ff}[2][\linewidth]{
    \begin{center}
        \begin{tcolorbox}[title=Forschungsfrage:, colback=white, colframe=black, width=#1]
            #2
        \end{tcolorbox}
\end{center}}


\newcommand{\steps}[2][0.8\linewidth]{
    \begin{center}
        \begin{tcolorbox}[title=Steps:, colback=white, colframe=blue, width=#1]
            \begin{enumerate}
                #2
            \end{enumerate}
        \end{tcolorbox}
\end{center}}


\usepackage{xifthen}
\makeatother
\def\@lecture{}%
\newcommand{\lecture}[3]{
    \ifthenelse{\isempty{#3}}{%
        \def\@lecture{Lecture #1}%
    }{%
        \def\@lecture{Lecture #1: #3}%
    }%
    \subsection*{\@lecture}
    \marginpar{\small\textsf{\mbox{#2}}}
}
\makeatletter

\author{Daniel Renschler}


