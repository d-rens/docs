\lecture{3}{03-06-2023}{4.5}
\section{Complementary events, at least one}

``At least one'' Means one or more.
\begin{itemize}
    \item the \textbf{complement} of ``at least one'' is none
\end{itemize}
$P(``at \ least \ one") = 1-P(``none")$

\ex{
    Flip coin 3 times. Prob of getting at least one head?
    \[ P(at \ least \ one \ head) = \frac{7}{8}
    .\] 
    Sample space: \{HHH, HHT, HTH, HTT, TTT, TTH, THT, THH\}

    \begin{align*}
        P(at \ least \ one \ head) &= 1-P(no \ heads)\\
    &=   1- P(T \ and \  T \ and \ T)= 1-\frac{1}{2}\cdot \frac{1}{2}\cdot \frac{1}{2}\\
    &= 1-\frac{1}{8} = \frac{7}{8}\\
    \end{align*}


    Flip coin 20 times, prob of getting at least one head:
    \[ P = 1- \frac{1}{2}^{20} = 1- \frac{1}{1,0485,276}= \frac{1,048,575}{1,048,576} = 0.999999046 .\] 
}

Less than $P=0.05$ means something is ``rare''.

$P(B|A)=$ probability that B happens given that A already happened.
$P(B \ and \ A)=$ probability that B and then A.

\[ P(B|A)=\frac{P(A \ and \ B)}{P(A)} .\] 

\[ P(A \ and \ B) = P(A) \cdot P(B|A) .\] 


Again refering to guilty table 4.2.
\[ P(Guilty|did\ it)= \frac{72}{81} .\] 
\[ P(did\ it|guilty) = \frac{72}{83} .\] 



