\documentclass{article}

\usepackage{biblatex}
\usepackage[a4paper,margin=3cm]{geometry}

\addbibresource{pug.bib}
\setlength{\parindent}{0pt}

\title{Was ist eigentlich in der Ukraine los?}
\author{Gresa, Jakub \& Daniel}
\date{3. Mai $\to$ 14. Juni}

\begin{document}
\maketitle

\section{Historisch}
Viktor Janokavitch war Russlands Marionette in der Ukraine an der Macht, als er
EU freundliche Abkommen unterschreiben sollte und es nicht macht
(wahrscheinlich auf russischen einfluss) wird die Bevoelkerung wuetend, es
endet in seiner Flucht nach Russland und blutigen Protesten.~\cite{was-ist-geschehen}

Darauf entstand eine EU freunldiche regierung in der Ukraine, damit war Russland
unzufrieden da die Ukraine von Russland abhaengen sollte wirtschaftlich sowie
politisch (z.B. Eurasische Union) und Russland ein naeheres buendnis mit der EU
als schaedlich fuer sich selbst sieht.~\cite{was-ist-geschehen}


\subsection{Krim}
In der Krim waren viele Russland nahe Buerger, z.B. durch russische Vorfahren,
diese waren mit der EU nahen, oder naehernden Regierung unzufrieden und wollte
darauf Teil Russlands sein.~\cite{was-ist-geschehen}

Russland statioinierte darauf vermummte soldaten auf dem Gebiet der Halbinsel
Krim. Ein wenig spaeter fuehrten sie Abstimmungen durch (16. Marz 2014), ob die Krim nun teil
Russlands werden solle, dies wurde mit 96,6\% pro entschieden dann nahm Putin
sie auf  (18. Maerz 2014).~\cite{was-ist-geschehen}

Die Ukraine erkannte die Abstimmung nicht, hatte aber auch nicht weiter
Einfluss, sie sah es als feindseelich und unrechmaesig (annexion).
Die Westlichen Staaten (EU und NATO Staaten) teilten die Ansicht der Ukraine
das es ein feindseelicher Akt ist.~\cite{was-ist-geschehen}


\subsection{Donezk und Lugansk}
In Donezk und Lugansk gab es Prorussische Seperatisten welche gewaltsam die
Macht an sich brachten. Nach Vorbild der Krim fuerten sie zwei Abstimmungen
durch, Ergebnisse waren fuer Donezk (89\%) und fuer Lugansk (98\%).

Weiter haben diese eigene Volksrepubliken gegrundet, und wollten sich dann an
Russland anschliesen. \textit{rausfinden: hat russland geholfen?}
Die Ukraine kannte die Volkrepubliken nicht an und bezeichnete sie als Terroristen.
Weiter teilten EU und NATO die Stellung der Ukraine.







\section{notizen}
recherche:
- abstimmungen in den seperatistischen gegenden


quellen:
- darf man alles nehmen, moeglichst wissenschaftlich, quellen alle angeben
\subsubsection{test-quelleon}
~\cite{was-ist-geschehen}
~\cite{what-could-come-next}
~\cite{open-source-intelligence}
~\cite{putins-war}
~\cite{avoiding-a-long-war}
~\cite{how-big-is-the-storm}
~\cite{why-did-putin}



\clearpage
\printbibliography


\end{document}
