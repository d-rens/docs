\documentclass{article}
\usepackage[margin=2cm]{geometry}
\usepackage{amsmath}
\usepackage{tabularx}
\usepackage{enumitem}
\usepackage{booktabs}
\setlength{\parindent}{0pt}


\title{Wirtschaft Zusammenfassung fuer KA 2}
\author{d.r.}


\begin{document}
\maketitle

\section{Standortfaktoren}
Hier irgendwas aus dem Heft:

\begin{table}[h]
\centering
\begin{tabular}{|p{4cm}|p{5cm}|p{6cm}|} 
\hline
\textbf{Standortfaktor} & \textbf{Erlaeuterung} & \textbf{Fragen des Unternehmens an moegliche Standorte} \\
\hline
Naturgegebene Bedingungen & Klima, Wassermenge, -qualitaet Rohstoffe & "Sind die benoetigten Rohstoffe erhaltlich?" \\ 
\hline
Arbeitskraeftepotential & Zahl und Qualitaet vorhandener Arbeitskraefte & "Wohnen in der Umgebung genuegend qualifizierte Arbeitskraefte?" \\
\hline
Abgaben und Steuern & Unterschiedliche Steuersaetze der Gemeinden; besondere Steuerverguenstigungen & "Kommen uns gewisse Gemeinden entgegen und schaffen guenstige Steuerbedinungen?" \\
\hline
Grundstuekpreise & Grundstuekpreise sind in staetischen Raeumen hoeher als auf dem Land. & "Welche Standorte wollen wir uns ueberhaupt leisten?" \\
\hline
Transportmoeglichkeiten & Strassen, Eisenbahn, Wasserstrasen, Flugverbindungen & "Auf welche Weise und wie schnell kommen Mitarbeitende, Kunden und Geschaeftspartner zum neuen Standort?" \\
\hline
Absatzmoeglichkeiten & Zahl und Finanzkraft potentieller Kunden & "Erreichen wir am neuen Standort genuegend zahlungskraeftige Kunden?" \\
\hline
Agglomerationsvorteile & Naehe von Zuliefer- bzw. weiterverarbeitenden Betrieben & "Ergeben sich fuer und Vorteile durch die Naehe anderer Unternehmen?"\\
\hline
\end{tabular}
\end{table}

\section{Unternehmensform (OHG)}

\textbf{Die offene Handelsgesellschaft} 

Sie entsteht durch Geselschaftsvertrag von zwei oder mehreren Personen. Die
Kapitaleinlagen konnen Geld, Sachwerte (z.B. Grundstucke) oder Rechten (z.B.
Patente) geleistet werden. Die Kapitaleinlagen werden getrennt gebucht, werden
aber gemeinschaftliches Vermogen, uber das die Gesellschafter nur gemeinsam
verfugen konnen. Der Firmenname kann die Familiennamen aller Gesellschafter
enthalten. Er muss aber den Namen mindestens eines Gesellschafters mit einem
die Gesellschaftsform kennzeichnenden Zusatz, z.B. \textbf{OHG \& Co.} Enthalten.

Die Rechbeziehung der Gesellschafter untereinander \textbf{(Innenverhaltnis)} werden
durch das Handelsgesetzbuch oder davon abweichende Verinbarungen im
Gesellschaftsvertrag geregelt. Laut \textbf{HGB} ist jeder gesellschafter zur
Geschaftsfurung berechtigt und verpflichtet. Bei ausergewonlichen Geschaften
ist ein Beschluss samtlicher Gesellschafter erforderlich. 

Der Gewinn wird so verteilt, dass zunachst jeder Gesellschafter eine
\textbf{vier prozentige Verzinsung seiner Kapitaleinlage erhaelt; der
Restgewinn wird zu gleichen Teilen (nach Koepfen) vergeben, falls im
Gesellschaftsvertrag nichts anderes bestimmt wird.} Ein eventueller Verlust
wird ebenfalls nach Koepfen verteilt.

Die Rechtsverhaltnisse der Gesellschafter gegenueber Ausenstehenden oder
Dritten, z.B. Lieferanten, Kunden, Kreditinstituten (\textbf{Ausenverhaltnis}) werden
allein durch die Bestimmungen des Handelsgesetzbuches geregelt. Jeder
Gesellschafter hat das Recht, die Gesellschaft nach ausen zu vertreten und
Geschafte fuer die OHG abzuschliesen (Einzelvertretungsbefugnis).

Die Haftung der Gesellschafter gegenueber den Glaubigern der OHG ist umfassend, sie ist:
\begin{itemize}
  \item \textbf{unmittelbar}, das heist jeder Gesellschafter haftet personlich
  \item \textbf{unbeschraenkt}, das heist jeder haftet mit seiner Kapitaleinlage und seinem Privatvermogen
  \item \textbf{gesamtschuldnerisch}, d.h. jeder Gesellschafter haftet fur die gesamten Schulden der OHG, auch fuer diejenigen, welche die Mitgesellschafter verursacht haben.
\end{itemize}


\section*{Unmittelbare, unbeschränkte und gesamtschuldnerische Haftung bei OHG}

Bei einer OHG haftet jeder Gesellschafter unmittelbar und unbeschränkt. Dies
bedeutet, dass jeder Gesellschafter persönlich für die Schulden der OHG haftet,
sowohl mit seiner Kapitaleinlage als auch mit seinem Privatvermögen. Zudem
haften die Gesellschafter gesamtschuldnerisch, das heißt jeder haftet für die
gesamten Schulden der OHG, auch wenn diese Schulden von den Mitgesellschaftern
verursacht wurden.

\section*{Beispielrechnung für Gewinnbeteiligung bei OHG}

Angenommen, Obermeier, Holzer und Gauser gründen gemeinsam eine OHG. Obermeier
bringt Maschinen und ein Grundstück im Wert von 300.000€ ein, Holzer ein Patent
im Wert von 180.000€ und Gauser Bargeld in Höhe von 250.000€. Im Jahr 2022 wird
ein Gewinn von 53.200€ erzielt. Wie wird dieser Gewinn aufgeteilt?

Zunächst erhält jeder Gesellschafter eine Verzinsung seiner Kapitaleinlage in Höhe von 4\%:

\begin{itemize}
  \item Obermeier: 300.000€ * 0,04 = 12.000€
  \item Holzer: 180.000€ * 0,04 = 7.200€
  \item Gauser: 250.000€ * 0,04 = 10.000€
\end{itemize}

Der verbleibende Gewinn nach Abzug der Verzinsung beträgt 24.000€. Da die
Gesellschafter gleichberechtigt sind, wird dieser Gewinn zu gleichen Teilen
aufgeteilt. Jeder Gesellschafter erhält also zusätzlich 8.000€:

\begin{itemize}
  \item Obermeier: 12.000€ + 8.000€ = 20.000€
  \item Holzer: 7.200€ + 8.000€ = 15.200€
  \item Gauser: 10.000€ + 8.000€ = 18.000€
\end{itemize}

\section*{Ratendarlehen}

Ein Ratendarlehen ist ein Kredit mit gleichbleibenden Tilgungen, variabler Zinsen und einer festen
Laufzeit. Jede Rate setzt sich aus einem Tilgungsanteil und einem Zinsanteil
zusammen.\\
Hier ein Beispiel für ein Ratendarlehen:

\begin{center}
\begin{tabular}{|c|c|c|c|c|c|}
\hline
Jahr & Darlehenssumme & Zinsen & Tilgung & Rate & Restdarlehen \\
\hline
1 & 40.000€ & 2.000€ & 8.000€ & 10.000€ & 32.000€ \\
2 & 32.000€ & 1.600€ & 8.000€ & 9.600€ & 24.000€ \\
3 & 24.000€ & 1.200€ & 8.000€ & 9.200€ & 16.000€ \\
4 & 16.000€ & 800€ & 8.000€ & 8.800€ & 8.000€ \\
5 & 8.000€ & 400€ & 8.000€ & 8.400€ & 0€ \\
\hline
Summe & & 6.000€ & 40.000€ & 46.000€ & \\
\hline
\end{tabular}
\end{center}

\section*{Annuitätendarlehen}
Ein Annuitätendarlehen ist ebenfalls ein Kredit mit gleichbleibenden Raten und
fester Laufzeit. Anders als beim Ratendarlehen besteht jede Rate jedoch aus
einem Tilgungsanteil und einem Zinsanteil, die sich im Laufe der Zeit ändern.
Die Höhe der Raten bleibt jedoch gleich. 

\subsection*{Ohne Laufzeit}

Beispiel Darlehenssumme \textbf{120.000}, Zinsen 5\%, Tilgung 1\%.

\textit{Gerunded auf Integers}

\begin{center}
\begin{tabular}{|c|c|c|c|c|c|}
\hline
Jahr & Darlehenssumme & Tilgung & Zinsen & Annuit\"at & Restdarlehen \\
\hline
1 &120.000  & 1.200  & 6.000  & 7.200 & 118.800 \\
2 &118.800  & 1.260  & 5.940  & 7.200 & 117.540 \\
3 &117.540  & 1.132  & 5.877  & 7.200 & 116.217 \\
4 &116.217  & 1.389  & 5.810  & 7.200 & 114.827 \\
5 &114.827  & 1.458  & 5.741  & 7.200 & 113.369 \\
\hline
\end{tabular}
\end{center}


\subsection*{Mit Laufzeit}
Annuit\"atsdarlehen mit Laufzeit hat eine Formel um die Annuit\"at zu bestimmen:
\begin{align*}
  i &= \text{Zinssatz}\\  
  n &= \text{Laufzeit in Jahren}\\  
  s &= \text{Darlehenssumme}
\end{align*}

\[\text{Annuit\"at}=\frac{i(1+i)^n}{(1+i)^n-1}\cdot s\]

\textbf{Beispiel:}
\begin{align*}
  i &= \text{6\%}\\  
  n &= \text{4 Jahre}\\  
  s &= \text{50.000}
\end{align*}

\[\text{Annuit\"at}=\frac{0,06(1+0,06)^4}{(1+0,06)^4-1}\cdot 50.000 = 14.429,55\]

\textit{Gerunded auf Integers}

\begin{center}
\begin{tabular}{|c|c|c|c|c|c|}
\hline
Jahr & Darlehenssumme & Zinsen & Tilgung & Annuit\"at & Restdarlehen \\
\hline
    1 & 50.000  & 3.000  & 11.429 & 14.429 & 38.570  \\
    2 & 38.570  & 2.314  & 12.115 & 14.429 & 26.455  \\
    3 & 26.570  & 1.578  & 12.842 & 14.429 & 13.612  \\
    4 & 13.612  &   816  & 13.621 & 14.429 & -       \\
$\sum$&         & 7718,2 & 50.000 & 57.718,2 & -       \\
\hline
\end{tabular}
\end{center}




\section*{Finanzplan}

Ein Finanzplan ist ein wichtiges Instrument bei der Planung von
unternehmerischen Aktivitäten. Er gibt einen Überblick über die erwarteten
Einnahmen und Ausgaben in einem bestimmten Zeitraum und ermöglicht so eine
realistische Einschätzung der finanziellen Situation. 

\section*{Kreditsicherung}

Kreditsicherung bezeichnet Maßnahmen, die ein Kreditinstitut ergreift, um das
Ausfallrisiko bei der Kreditvergabe zu minimieren. Hierzu zählen beispielsweise
die Verpfändung von Sicherheiten oder die Bürgschaft Dritter.

\begin{table}[!htb]
  \centering
  \begin{tabular}{|p{0.45\linewidth}|p{0.45\linewidth}|}
    \hline
    \textbf{Personalkredite} & \textbf{Realkredite} \\
    \hline
    d.h. der Kredit wird aufgrund der Bonität des Kreditnehmers gewährt
    \begin{itemize}[nosep, leftmargin=*]
      \item Keine Kreditsicherung, da Kreditnehmer als persönlich zuverlässig gilt
      \item Bürgschaft
    \end{itemize} &
    d.h. der Kredit wird über Immobilien oder Mobilien abgesichert.
    \begin{itemize}[nosep, leftmargin=*]
      \item Gesichert durch bewegliche Sachen in Form eines Pfandrechts (Lombardkredit) oder durch Sicherungsübereignung
      \item Gesichert durch unbewegliche Sachen (Grundschuldkredit)
    \end{itemize} \\
    \hline
  \end{tabular}
\end{table}


\begin{table}[!htb]
  \centering
   \textbf{Bürgschaft}
  \begin{tabularx}{\linewidth}{|X|X|}
    \hline
    \textbf{Ausfallbürgschaft}, d.h. ein Dritter verbürgt sich für die Rückzahlung des Darlehens. Ihm steht die Einrede der Vorausklage zu. & \textbf{Selbstschuldnerische Bürgschaft}, d.h. der Bürge verpflichtet sich so, als wäre er selbst der Schuldner. \\
    \hline
  \end{tabularx}
\end{table}


%:pagebreak
\hrulefill

\section*{Leasing}
Leasing ist eine Finanzierungsmöglichkeit, bei der ein Vermögensgegenstand
(z.B. ein Auto oder eine Maschine) nicht gekauft, sondern nur gemietet wird.
Der Leasingnehmer zahlt eine monatliche Rate an den Leasinggeber und kann den
Gegenstand nutzen, ohne ihn kaufen zu müssen. Leasing kann insbesondere für
Unternehmen sinnvoll sein, um Investitionen zu tätigen, ohne das Eigenkapital
belasten zu müssen.

\subsection*{Fragen:}
\subsubsection*{Direktes und Indirektes Leasing:}
Beim direkten Leasing tritt der Leasinggeber selbst als Eigentümer der
geleasten Sache auf. Der Leasinggeber besitzt das Objekt und vermietet es an
den Leasingnehmer. Der Leasingnehmer hat kein Anrecht auf das Eigentum an der
Sache am Ende des Leasingvertrags.

Beim indirekten Leasing hingegen erwirbt der Leasinggeber die Sache, die er dem
Leasingnehmer zur Verfügung stellt, nicht selbst, sondern finanziert den Kauf
durch eine dritte Partei, wie zum Beispiel eine Bank oder ein Finanzinstitut.
Der Leasinggeber tritt in diesem Fall als Vermittler auf und erhält eine
Provision für die Vermittlung des Leasinggeschäfts. Der Leasingnehmer hat auch
hier kein Anrecht auf das Eigentum an der Sache am Ende des Leasingvertrags.

Der Unterschied zwischen direktem und indirektem Leasing liegt somit
hauptsächlich im Eigentumsverhältnis während der Laufzeit des Leasingvertrags.
Beim direkten Leasing bleibt der Leasinggeber Eigentümer der Sache, während
beim indirekten Leasing ein Dritter der Eigentümer ist.

\subsubsection*{Privates und Gewerbliches Leasing:}
Beim privaten Leasing handelt es sich um das Leasing eines Fahrzeugs oder einer
anderen Sache durch eine Privatperson für den persönlichen Gebrauch. Die
Leasingraten werden in der Regel aus dem persönlichen Einkommen des
Leasingnehmers bezahlt und sind nicht steuerlich absetzbar.

Im Gegensatz dazu bezieht sich das gewerbliche Leasing auf das Leasing einer
Sache durch ein Unternehmen für den betrieblichen Einsatz. Die Leasingraten
können in der Regel als Betriebsausgaben steuerlich abgesetzt werden, was für
das Unternehmen steuerliche Vorteile bringen kann. Zudem können bei einem
gewerblichen Leasing die Vorsteuerbeträge geltend gemacht werden.

Ein weiterer Unterschied zwischen privatem und gewerblichem Leasing betrifft
die Vertragsgestaltung. Gewerbliche Leasingverträge sind oft flexibler und
bieten mehr Anpassungsmöglichkeiten an die spezifischen Anforderungen des
Unternehmens, während private Leasingverträge in der Regel standardisiert und
weniger anpassungsfähig sind.

Es ist jedoch zu beachten, dass die steuerlichen Auswirkungen des Leasings von
Land zu Land unterschiedlich sein können und dass die genauen Details des
Leasingvertrags sowie die individuelle steuerliche Situation des Leasingnehmers
berücksichtigt werden sollten, um die spezifischen Auswirkungen des Leasings zu
verstehen.

\subsubsection*{Unterschied zwischen Leasing und Miete:}
Der Hauptunterschied zwischen Leasing und Miete besteht darin, dass beim
Leasing der Leasingnehmer das Nutzungsrecht an einer Sache für einen längeren
Zeitraum erwerben kann, während bei der Miete die Sache nur für einen kurzen
Zeitraum gemietet wird und danach zurückgegeben werden muss. \textbf{Im Gegensatz zum
Leasing hat der Mieter kein langfristiges Nutzungsrecht an der gemieteten Sache
und ist nicht für deren Instandhaltung verantwortlich.}

\subsubsection*{Wie mit Leasing kosten sparen:}
Leasing kann Kosten sparen, indem es Unternehmen oder Privatpersonen
ermöglicht, eine Sache für einen bestimmten Zeitraum zu nutzen, ohne den vollen
Kaufpreis bezahlen zu müssen. Dadurch müssen sie weniger Kapital investieren,
was die Liquidität erhöhen und die Finanzierung von Projekten oder anderen
Anschaffungen erleichtern kann. Zudem fallen oft geringere Anzahlungen und
niedrigere monatliche Raten an als bei einem Kreditkauf. Leasing kann auch
steuerliche Vorteile bieten, insbesondere für Unternehmen, da die Leasingraten
als Betriebsausgaben absetzbar sein können.


\end{document}
