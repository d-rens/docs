\usepackage[a4paper,margin=3.5cm]{geometry}
\usepackage{amsmath,amsfonts,amssymb}
\usepackage{cancel}
\usepackage{xcolor}
\usepackage{tcolorbox}
\usepackage{polynom}
\usepackage{wrapfig}
\usepackage{booktabs}
\usepackage{tabularx}
\usepackage{multicol}
\usepackage{hyperref}
\usepackage[ngerman]{babel}
\usepackage{pgfplots}
\pgfplotsset{compat=1.18}
\usepackage{tikz}
\usepackage{pstricks}
\usepackage{pst-plot}
\usepackage{textcomp}
\usepackage{import}
\usepackage{pdfpages}
\usepackage{transparent}

\newcommand{\incfig}[2][1]{%
    \def\svgwidth{0.9\textwidth}
    \import{./math-figures/test/{#2.eps_tex}}

\pdfsuppresswarningpagegroup=1

\setlength{\parindent}{0pt}

\newcommand{\tbox}[2][0.8\linewidth]{
    \begin{center}
        \begin{tcolorbox}[colback=white, colframe=gray, width=#1]
                #2
        \end{tcolorbox}
    \end{center}}

\newcommand{\regel}[2][0.8\linewidth]{
    \begin{center}
        \begin{tcolorbox}[title=regel, colback=white, colframe=blue, width=#1]
                #2
        \end{tcolorbox}
    \end{center}}

\newcommand{\nt}[2][0.9\linewidth]{
    \begin{center}
        \begin{tcolorbox}[title=Notitz:, colback=white, colframe=gray, width=#1]
                #2
        \end{tcolorbox}
    \end{center}}


\newcommand{\ex}[2][\linewidth]{
    \begin{center}
        \begin{tcolorbox}[title=Beispiel:, colback=white, colframe=brown, width=#1]
                  #2
        \end{tcolorbox}
    \end{center}}


\newcommand{\q}[2][\linewidth]{
    \begin{center}
        \begin{tcolorbox}[title=Frage:, colback=white, colframe=purple, width=#1]
                  #2
        \end{tcolorbox}
    \end{center}}
