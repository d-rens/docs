\documentclass{book}

\usepackage[a4paper,margin=3.5cm]{geometry}
\usepackage{cancel}
\usepackage{tikz}
\usepackage{amsmath,amsfonts,amssymb}
\usepackage{xcolor}
\usepackage{tcolorbox}
\usepackage{polynom}
\usepackage{wrapfig}
\usepackage{booktabs}
\usepackage{tabularx}
\usepackage{multicol}
\usepackage{hyperref}
\usepackage[ngerman]{babel}
\usepackage{tikz}
\usepackage{pgfplots}
\pgfplotsset{compat=1.18}
\usepackage{pstricks}
\usepackage{pst-plot}
\usepackage{textcomp}
\usepackage{import}
\usepackage{pdfpages}
%\usepackage{transparent}


\newcommand{\incfig}[2][1]{%
    \def\svgwidth{#1\columnwidth}
    \import{./figures/}{#2.pdf_tex}
}

\setlength{\parindent}{0pt}

\newcommand{\tbox}[2][0.8\linewidth]{
    \begin{center}
        \begin{tcolorbox}[colback=white, colframe=gray, width=#1]
                #2
        \end{tcolorbox}
    \end{center}}

\newcommand{\regel}[2][0.8\linewidth]{
    \begin{center}
        \begin{tcolorbox}[title=regel, colback=white, colframe=blue, width=#1]
                #2
        \end{tcolorbox}
    \end{center}}

\newcommand{\nt}[2][0.9\linewidth]{
    \begin{center}
        \begin{tcolorbox}[title=Notitz:, colback=white, colframe=gray, width=#1]
                #2
        \end{tcolorbox}
    \end{center}}


\newcommand{\ex}[2][\linewidth]{
    \begin{center}
        \begin{tcolorbox}[title=Beispiel:, colback=white, colframe=brown, width=#1]
                  #2
        \end{tcolorbox}
    \end{center}}


\newcommand{\q}[2][\linewidth]{
    \begin{center}
        \begin{tcolorbox}[title=Frage:, colback=white, colframe=purple, width=#1]
                  #2
        \end{tcolorbox}
    \end{center}}



\title{some tests}
\author{}
\date{}

\begin{document}

\maketitle
\tableofcontents

%\paragraph{notice}
%this is a test of my config with a calculus lecture, in which i try to use it
%in realtime to test for speed, and missing features.

\chapter{Calc 5.1 -- Area between two curves}


\begin{figure}[ht]
    \centering
    \incfig[0.7]{area-between-curves-1}
    \caption{area-between-curves-1}
    \label{fig:area-between-curves-1}
\end{figure}

Area between $f(x)\quad \& \quad g(x)$.

\[ A= \text{Area under $f(x)$ - Area under $g(x)$} .\] 

\[ A= \int^b_a f(x) dx - \int^b_a g(x) dx .\] 

\[
    A=\int^b_a \left[f(x)-g(x)\right] dx
.\] 

\nt{
    \[ f(x) \geq g(x)\] 
    \[ \forall x \in [a,b] .\] 
    \[ \left( f(x) \text{ is above } g(x)\right) .\] 
}


\begin{figure}[htbp]
    \centering
    \incfig[0.5]{are-between-curves-w}
    \caption{are-between-curves-2}
    \label{fig:are-between-curves-2}
\end{figure}

\[ \int_{a}^{c} \left[ f(x)-0 \right] dx + \int_{c}^{b}  \left[ 0-f(x) \right] dx \] 
\[ \int_{a}^{c}  f(x) dx - \int_{c}^{b} f(x) dx \] 


\ex{Find the area bounded above by $y=2x+5$, and bounded below by $y=x^3$ on $\left[ 0,2 \right]$ }



\begin{align*}
A &=\int_{0}^{2}  (2x+5) - x^3 \text{dx}\\
&= \int_{0}^{2}2x+5=x^3 dx   \\
&= x^2+5x - \frac{x^4}{4}]^2_0  \\
A &= \left[ 2^2+5\cdot 2 \frac{2^4}{4} \right] - \left[ 0 \right]   \\
A&= 10 \\
\end{align*}

\ex{Find area between $y=x^2$ and  $y=x+6$.}

\begin{figure}[htbp]
    \centering
    \incfig[0.5]{area-between-curves-3}
    \caption{area-between-curves-3}
    \label{fig:area-between-curves-3}
\end{figure}    

\nt{
\textbf{Steps:}
\begin{enumerate}
    \item Find x-cords of the Intersection of the curves. ( Set $f(x)=g(x)$ ) 
    \item Which function is on the top? \\
        (Pick one point for each interval)
    \item set-up and solve
\end{enumerate}
}   

\begin{align*}
    x+6&= x^2 \\
    x^2-x-6&= 0 \\
    \left( x-3 \right)&  \left( x+2 \right) \\ 
    x-3 = 0 \quad &\quad  x+2 = 0\\
    x=3 \quad&\quad x=-2
\end{align*}

Those are the only places $f(x)$ and  $g(x)$ are intercepting, so those are the bounds of integration.
\[
A=\int_{-2}^{3}  
.\] 


\begin{figure}[htbp]
    \centering
    \incfig[0.4]{area-between-curves-4}
    \caption{figures/area-between-curves-4}
    \label{fig:area-between-curves-4}
\end{figure}

\begin{align*}
    A&= \int_{-2}^{3} \left( x+6 \right) -(x^2) \text{dx} \implies \int_{-2}^{3} x+6-x^2 \text{dx}   \\
    &= \left[ \frac{x^2}{2}+6x -\frac{x^3}{3}\right]_{-2}^3 = \left[
    \frac{3^2}{2}+6\cdot 3-\frac{3^3}{3} \right] - \left[
\frac{(-2)^2}{2}+6(-2)-\frac{-2^3}{3} \right]  \\
&= \left[  \frac{9}{2}+18-9  \right]-\left[ 2-12+\frac{8}{3} \right]=\left[ \frac{9}{2}+9 \right] -\left[ -10+\frac{8}{3} \right]  \\
&= \frac{27}{2}-\frac{-22}{3}\to \frac{27}{2}+\frac{22}{3}= \frac{125}{6} \\
\end{align*}


\ex{
Find area bound by $y=x^3$ and  $y=x$.
}

\begin{align*}
    x^3&= x \\
    x^3-x&= 0 \\
    x(x^2-1)&= 0 \\
    x(x+1)(x-1)&=0 \\
    x=0 ,\quad & x=1, \quad  x=-1
\end{align*}

\regel{Here we test again which function is where above the other, to know, in which
direction to integrate. (always the one on the top minus the one on the bottom,
as shown at the start of this chapter.) }

\begin{figure}[htbp]
    \centering
    \incfig[0.4]{area-between-curves-5}
    \caption{figures/area-between-curves-5}
    \label{fig:area-between-curves-5}
\end{figure}

Here we add those integrals together, because we want both ares together.

\begin{align*}
    A&= \int_{-1}^{0} x^3-x \ \text{dx} + \int_{0}^{1} x-x^3 \ \text{dx}   \\
    &= \left[  \frac{x^4}{4}-\frac{x^2}{2} \right]^0_{-1} + \left[ \frac{x^2}{2}-\frac{x^4}{4} \right]^1_0  \\
    &= \left[ \left( \frac{0}{4}- \frac{0}{2} \right) - \left( \frac{(-1)^4}{4}- \frac{(-1)^2}{2} \right)  \right]+ 
    \left[ \left( \frac{1^2}{2}- \frac{1^4}{4} \right)- \left( \frac{0}{2}- \frac{0}{4} \right)   \right]  \\
    &= \left[ 0-\left( \frac{1}{4} - \frac{1}{2} \right)  \right] + \left[ \left( \frac{1}{2}-\frac{1}{4} \right) -0 \right]  \\
    &= \frac{1}{4}+\frac{1}{4}= \frac{1}{2} \\
\end{align*}


\begin{figure}[ht]
    \centering
    \incfig[0.5]{area-between-curves-6}
    \caption{figures/area-between-curves-6}
    \label{fig:area-between-curves-6}
\end{figure}

In this case, the area represents the distance car 'me' is ahead of car 'you'.

This can then be solved by:

\[
A=\int_{0}^{b} V_1(t) - V_2(t) \ \text{dt}
.\] 

\ex{
Find the area bound by $x=y^2$ and $y=x-2$.
}

\begin{figure}[ht]
    \centering
    \incfig[0.4]{area-between-curves-7}
    \caption{figures/area-between-curves-7}
    \label{fig:area-between-curves-7}
\end{figure}

\begin{align*}
    x=y^2 \quad & \quad y=x-2\\
    x=y^2 \quad & \quad x=y+2\\
    y^2&= y+2 \\
       &\downarrow \\
    y^2-y-2&= 0 \\
    (y-2)(y+1)&= 0 \\
    y-2=0 \quad & \quad y+1=0 \\
    y=2 \quad & \quad y=-1 \\
\end{align*}



\end{document}
