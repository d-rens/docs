\documentclass{book}

\usepackage[a4paper,margin=3.5cm]{geometry}
\usepackage{cancel}
\usepackage{tikz}
\usepackage{amsmath,amsfonts,amssymb}
\usepackage{xcolor}
\usepackage{tcolorbox}
\usepackage{polynom}
\usepackage{wrapfig}
\usepackage{booktabs}
\usepackage{tabularx}
\usepackage{multicol}
\usepackage{hyperref}
\usepackage[ngerman]{babel}
\usepackage{tikz}
\usepackage{pgfplots}
\pgfplotsset{compat=1.18}
\usepackage{pstricks}
\usepackage{pst-plot}
\usepackage{textcomp}
\usepackage{import}
\usepackage{pdfpages}
%\usepackage{transparent}

\newcommand{\incfig}[2][1]{%
    \def\svgwidth{0.7\textwidth}
\import{./figures/}{#2.eps_tex}
}

\setlength{\parindent}{0pt}

\newcommand{\tbox}[2][0.8\linewidth]{
    \begin{center}
        \begin{tcolorbox}[colback=white, colframe=gray, width=#1]
                #2
        \end{tcolorbox}
    \end{center}}

\newcommand{\regel}[2][0.8\linewidth]{
    \begin{center}
        \begin{tcolorbox}[title=regel, colback=white, colframe=blue, width=#1]
                #2
        \end{tcolorbox}
    \end{center}}

\newcommand{\nt}[2][0.9\linewidth]{
    \begin{center}
        \begin{tcolorbox}[title=Notitz:, colback=white, colframe=gray, width=#1]
                #2
        \end{tcolorbox}
    \end{center}}


\newcommand{\ex}[2][\linewidth]{
    \begin{center}
        \begin{tcolorbox}[title=Beispiel:, colback=white, colframe=brown, width=#1]
                  #2
        \end{tcolorbox}
    \end{center}}


\newcommand{\q}[2][\linewidth]{
    \begin{center}
        \begin{tcolorbox}[title=Frage:, colback=white, colframe=purple, width=#1]
                  #2
        \end{tcolorbox}
    \end{center}}



\title{Statistics}
\author{}
\date{}

\begin{document}

\maketitle
\tableofcontents


\chapter{1.1 -- Key Words and Definitions}

\section{Key words}

\begin{table}[htbp]
  \centering
  \begin{tabular}{>{\bfseries}l l}
      \toprule
      Data & Any observations that hvae been collected. \\
      \midrule
      Statistics & Collect, analyze, summarize, interpret and draw conclustions from there. \\
      \midrule
      Population & The complete set of elements being studied. \\
      \midrule
      Samples & Some subset of the population.\\
      \midrule
      Census & Collection from every member of a population. \\
      \bottomrule
  \end{tabular}
  \caption{Statistics Vocabulary}
  \label{tab:vocab}
\end{table}

$\to$ If you take a sample, it must be collected \textbf{randomly}.

\section{Types pf Data}

\begin{table}[htbp]
  \centering
  \begin{tabular}{l |>{\bfseries}l l}
      \toprule
      P-P & Parameter & A characteristic of a population. \\
      \midrule
      S-S & Statistic & A characteristic of a sample. \\
      \bottomrule
  \end{tabular}
  \caption{Statistics Vocabulary}
  \label{tab:vocab-2}
\end{table}

\section{Two Types of Data}

\begin{table}[htbp]
  \centering
  \begin{tabular}{>{\bfseries}l l}
      \toprule
      Qualitative (Categorical) & Data that is non-numerical \\
                                & e.g. color, gender, race, zip-codes...\\
                                & Mathematical operations are \textbf{meaningless}.\\
      \midrule
      Quantitative & Numerical \\
                   & e.g. height/weight, wages, temperature, time.  \\
                   & Mathematical operations are \textbf{meaningful}.  \\
      \bottomrule
  \end{tabular}
  \caption{table}
  \label{tab:vocab-3}
\end{table}

\clearpage


\subsection{Two types of Quantitative Data}

\begin{table}[htbp]
  \centering
  \begin{tabular}{>{\bfseries}l l}
      \toprule
      Discrete data & Countabe or finite \\
                    & Numbers of eggs, dice... \\
      \midrule
      Continious Data: & Infinite number of possible values (not countable) \\
                       & Usually a \textbf{measurement}, e.g. temperature. \\
      \bottomrule
  \end{tabular}
  \caption{Quantitative data}
  \label{tab:vocab-4}
\end{table}

\section{4 Levels of Measurement}

\begin{table}[htbp]
  \centering
  \begin{tabular}{>{\bfseries}l l}
      \toprule
      Nominal & Categories \textbf{not} ordered. e.g. religion \\
      \midrule
      Ordinal & Can be ordered, differences are meaningless \\
              & Rank, color (spectrum)...\\
      \midrule
      Interval & Ordered, differences are meaningful, no "Natural Zero" \\
               & e.g. temperature \\
      \midrule
      Ratio & Just like interval, but with a natural zero.\\
            & e.g. amount of money  \\
      \bottomrule
  \end{tabular}
  \caption{Measurements}
  \label{tab:vocab-5}
\end{table}


\section{Design of Experiments/Observations}

\subsection{Observation vs. Experiment}

An \textbf{observation} measures specific traits, but noes \textbf{not} modify subjects.

An \textbf{experiment} applies a treatment and then measuers the effect on the subjects.


\subsection{Random}
Each member of a population, has an equal chance of being selected in a sample.

\subsubsection{Simple random sample}
Each group of size 'n' has an equal chance of being selected.

\subsection{Common techniques to get a sample}
\begin{table}[htbp]
    \centering
    \caption{4 Common techniques to get a sample}
    \begin{tabular}{>{\bfseries}l p{10cm}}
        \toprule
        Convenience sample & You use the results, which you easily get (not random) \\
        \midrule
        Systematic sampling & Put a population in some order and select every "$k^{th}$" member. \\
        \midrule
        Stratafied Sample & Breaking population into sub-groups based on some
        characteristic, and then take a simple random sample out of each
        sub-groups.\\
        \midrule
        Cluster sample & Divide population into "clusters" (regardless of
        characteristic), randomly select a certain number of clusters, and then
        collect data from the entire cluster.\\
        \bottomrule
    \end{tabular}
    \label{tab:vocab-6}
\end{table}


\section{Frequency Distribution}
A frequency distribution is a list of values with corresponding frequencies.

\begin{table}[htbp]
    \centering
    \begin{tabular}{>{\bfseries}l p{10cm}}
        \toprule
        Class width & Difference between two "lower class limits" \\
        \midrule
        Lower class limit & Smallest value belonging to a class \\
        \midrule
        Upper class limit & Highest value belonging to a class \\
        \bottomrule
    \end{tabular}
    \caption{Frequency Distribution Terms}
\end{table}

\steps{
    \item Determine number of classes: 8
    \item class width: \[
    \frac{\text{Max Value - Min value}}{\text{number of classes}} \leadsto \frac{44-18}{8} \leadsto \frac{26}{8} \leadsto 3.25 \] 
    Round \textbf{up}. $\leadsto 4$
    \item Start with the minimum value: 18
    \item Creat classes with class width (4)
    \item Find the class midpoint: \[
            \frac{\text{upper class limit + lower class limit}}{2} \leadsto  .\] 
    \item Class boundaries: used to seperate classes without gaps.
}

\begin{multicols}{2}
    

\textbf{class width:} 4

\textbf{Lower class limit:} 18, 22, 26, $\ldots$ 46

\textbf{upper class limit:} 21,25 $\ldots$ 49

\textbf{class midpoint:}
\[ \frac{\text{upper class limit + lower class limit}}{2}\] 
$\leadsto 19.5, 23.5, 27.5, 31.5, 35.5, 39.5, 43.5, 47.5$

class-width inbetween

\textbf{class boundaries:}
Used to seperate classes without gaps.
17.5, 21.5, 25.5, 29.5, 33.5, 37.5, 41.5, 49.5

\textbf{Relative frequency distribution:} Percentage \[\frac{\text{class } f.}{\sum f. (n)}\]

\textbf{Cumulative Frequency Distribution}
Adds sequential classes together.
\end{multicols}

\begin{table}[htbp]
    \centering
    \begin{tabular}{c|c|c|c}
        \toprule
        Age & Freq. & Rel. Freq. & Cum. Freq.\\
        \midrule
        18-21 & 25 & 58.1\% & 25  \\
        22-25 & 10 & 23.3\% & 35  \\
        26-29 & 4  & 9.3\%  & 39  \\
        30-33 & 2  & 4.7\%  & 41  \\
        34-37 & 1  & 2.3\%  & 42  \\
        38-41 & 0  & 0\%    & 42  \\
        42-43 & 1  & 2.3\%  & 43  \\
        46-49 & 0  & 0\%    & 43  \\
        \midrule
              & n=43 & 100\% \\
              & $\sum f \uparrow $ \\
              \bottomrule
    \end{tabular}
    \caption{Frequency Distribution}
\end{table}



\subsection{Touching Bar Chart}

\begin{figure}[ht]
    \centering
    \incfig{stats-1}
    \caption{figures/stats-1}
    \label{fig:stats-1}
\end{figure}

A cumulative chart would look exactly the same, but instead of having
boundaries numered it'd be in the middle of the bars with teh cumulative
frequency from class 1-8. And also the y-axis would be the percentage.

There is also a last one, where one takes the cumulative stuff, so that the graph colums are getting bigger and bigger...

\textbf{Horizontal:} Class midpoints or boundaries.

\textbf{Vertical:} Frequency.

\nt{next lesson: 3.2}



    





\end{document}
