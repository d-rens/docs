\documentclass{book}

\usepackage[a4paper,margin=3.5cm]{geometry}
\usepackage{cancel}
\usepackage{tikz}
\usepackage{amsmath,amsfonts,amssymb}
\usepackage{xcolor}
\usepackage{tcolorbox}
\usepackage{polynom}
\usepackage{wrapfig}
\usepackage{booktabs}
\usepackage{tabularx}
\usepackage{multicol}
\usepackage{hyperref}
\usepackage[ngerman]{babel}
\usepackage{tikz}
\usepackage{pgfplots}
\pgfplotsset{compat=1.18}
\usepackage{pstricks}
\usepackage{pst-plot}
\usepackage{textcomp}
\usepackage{import}
\usepackage{pdfpages}
%\usepackage{transparent}

\newcommand{\incfig}[2][1]{%
    \def\svgwidth{0.7\textwidth}
\import{./figures/}{#2.eps_tex}
}

\setlength{\parindent}{0pt}

\newcommand{\tbox}[2][0.8\linewidth]{
    \begin{center}
        \begin{tcolorbox}[colback=white, colframe=gray, width=#1]
                #2
        \end{tcolorbox}
    \end{center}}

\newcommand{\regel}[2][0.8\linewidth]{
    \begin{center}
        \begin{tcolorbox}[title=regel, colback=white, colframe=blue, width=#1]
                #2
        \end{tcolorbox}
    \end{center}}

\newcommand{\nt}[2][0.9\linewidth]{
    \begin{center}
        \begin{tcolorbox}[title=Notitz:, colback=white, colframe=gray, width=#1]
                #2
        \end{tcolorbox}
    \end{center}}


\newcommand{\ex}[2][\linewidth]{
    \begin{center}
        \begin{tcolorbox}[title=Beispiel:, colback=white, colframe=brown, width=#1]
                  #2
        \end{tcolorbox}
    \end{center}}


\newcommand{\q}[2][\linewidth]{
    \begin{center}
        \begin{tcolorbox}[title=Frage:, colback=white, colframe=purple, width=#1]
                  #2
        \end{tcolorbox}
    \end{center}}



\title{Statistics}
\author{}
\date{}

\begin{document}

\maketitle
\tableofcontents


\chapter{1.1 -- Key Words and Definitions}

\section{Key words}

\begin{table}[htbp]
  \centering
  \begin{tabular}{>{\bfseries}l l}
      \toprule
      Data & Any observations that hvae been collected. \\
      \midrule
      Statistics & Collect, analyze, summarize, interpret and draw conclustions from there. \\
      \midrule
      Population & The complete set of elements being studied. \\
      \midrule
      Samples & Some subset of the population.\\
      \midrule
      Census & Collection from every member of a population. \\
      \bottomrule
  \end{tabular}
  \caption{Statistics Vocabulary}
  \label{tab:vocab}
\end{table}

$\to$ If you take a sample, it must be collected \textbf{randomly}.

\section{Types pf Data}

\begin{table}[htbp]
  \centering
  \begin{tabular}{l |>{\bfseries}l l}
      \toprule
      P-P & Parameter & A characteristic of a population. \\
      \midrule
      S-S & Statistic & A characteristic of a sample. \\
      \bottomrule
  \end{tabular}
  \caption{Statistics Vocabulary}
  \label{tab:vocab-2}
\end{table}

\section{Two Types of Data}

\begin{table}[htbp]
  \centering
  \begin{tabular}{>{\bfseries}l l}
      \toprule
      Qualitative (Categorical) & Data that is non-numerical \\
                                & e.g. color, gender, race, zip-codes...\\
                                & Mathematical operations are \textbf{meaningless}.\\
      \midrule
      Quantitative & Numerical \\
                   & e.g. height/weight, wages, temperature, time.  \\
                   & Mathematical operations are \textbf{meaningful}.  \\
      \bottomrule
  \end{tabular}
  \caption{table}
  \label{tab:vocab-3}
\end{table}

\clearpage


\subsection{Two types of Quantitative Data}

\begin{table}[htbp]
  \centering
  \begin{tabular}{>{\bfseries}l l}
      \toprule
      Discrete data & Countabe or finite \\
                    & Numbers of eggs, dice... \\
      \midrule
      Continious Data: & Infinite number of possible values (not countable) \\
                       & Usually a \textbf{measurement}, e.g. temperature. \\
      \bottomrule
  \end{tabular}
  \caption{Quantitative data}
  \label{tab:vocab-4}
\end{table}

\section{4 Levels of Measurement}

\begin{table}[htbp]
  \centering
  \begin{tabular}{>{\bfseries}l l}
      \toprule
      Nominal & Categories \textbf{not} ordered. e.g. religion \\
      \midrule
      Ordinal & Can be ordered, differences are meaningless \\
              & Rank, color (spectrum)...\\
      \midrule
      Interval & Ordered, differences are meaningful, no "Natural Zero" \\
               & e.g. temperature \\
      \midrule
      Ratio & Just like interval, but with a natural zero.\\
            & e.g. amount of money  \\
      \bottomrule
  \end{tabular}
  \caption{Measurements}
  \label{tab:vocab-5}
\end{table}

\nt{pause at 1.5}







\end{document}
