\documentclass{book}

\usepackage[a4paper,margin=3.5cm]{geometry}
\usepackage{cancel}
\usepackage{tikz}
\usepackage{amsmath,amsfonts,amssymb}
\usepackage{xcolor}
\usepackage{tcolorbox}
\usepackage{polynom}
\usepackage{wrapfig}
\usepackage{booktabs}
\usepackage{tabularx}
\usepackage{multicol}
\usepackage{hyperref}
\usepackage[ngerman]{babel}
\usepackage{tikz}
\usepackage{pgfplots}
\pgfplotsset{compat=1.18}
\usepackage{pstricks}
\usepackage{pst-plot}
\usepackage{textcomp}
\usepackage{import}
\usepackage{pdfpages}
%\usepackage{transparent}

\newcommand{\incfig}[2][1]{%
    \def\svgwidth{0.7\textwidth}
\import{./figures/}{#2.eps_tex}
}

\setlength{\parindent}{0pt}

\newcommand{\tbox}[2][0.8\linewidth]{
    \begin{center}
        \begin{tcolorbox}[colback=white, colframe=gray, width=#1]
                #2
        \end{tcolorbox}
    \end{center}}

\newcommand{\regel}[2][0.8\linewidth]{
    \begin{center}
        \begin{tcolorbox}[title=regel, colback=white, colframe=blue, width=#1]
                #2
        \end{tcolorbox}
    \end{center}}

\newcommand{\nt}[2][0.9\linewidth]{
    \begin{center}
        \begin{tcolorbox}[title=Notitz:, colback=white, colframe=gray, width=#1]
                #2
        \end{tcolorbox}
    \end{center}}


\newcommand{\ex}[2][\linewidth]{
    \begin{center}
        \begin{tcolorbox}[title=Beispiel:, colback=white, colframe=brown, width=#1]
                  #2
        \end{tcolorbox}
    \end{center}}


\newcommand{\q}[2][\linewidth]{
    \begin{center}
        \begin{tcolorbox}[title=Frage:, colback=white, colframe=purple, width=#1]
                  #2
        \end{tcolorbox}
    \end{center}}



\title{Statistics}
\author{}
\date{}

\begin{document}

\maketitle
\tableofcontents


\chapter{1.1 -- Key Words and Definitions}

\section{Key words}

\begin{table}[htbp]
    \centering
    \begin{tabular}{>{\bfseries}l l}
        \toprule
        Data & Any observations that hvae been collected. \\
        \midrule
        Statistics & Collect, analyze, summarize, interpret and draw conclustions from there. \\
        \midrule
        Population & The complete set of elements being studied. \\
        \midrule
        Samples & Some subset of the population.\\
        \midrule
        Census & Collection from every member of a population. \\
        \bottomrule
    \end{tabular}
    \caption{Statistics Vocabulary}
    \label{tab:vocab}
\end{table}

$\to$ If you take a sample, it must be collected \textbf{randomly}.

\section{Types pf Data}

\begin{table}[htbp]
    \centering
    \begin{tabular}{l |>{\bfseries}l l}
        \toprule
        P-P & Parameter & A characteristic of a population. \\
        \midrule
        S-S & Statistic & A characteristic of a sample. \\
        \bottomrule
    \end{tabular}
    \caption{Statistics Vocabulary}
    \label{tab:vocab-2}
\end{table}

\section{Two Types of Data}

\begin{table}[htbp]
    \centering
    \begin{tabular}{>{\bfseries}l l}
        \toprule
        Qualitative (Categorical) & Data that is non-numerical \\
                                  & e.g. color, gender, race, zip-codes...\\
                                  & Mathematical operations are \textbf{meaningless}.\\
        \midrule
        Quantitative & Numerical \\
                     & e.g. height/weight, wages, temperature, time.  \\
                     & Mathematical operations are \textbf{meaningful}.  \\
        \bottomrule
    \end{tabular}
    \caption{table}
    \label{tab:vocab-3}
\end{table}

\clearpage


\subsection{Two types of Quantitative Data}

\begin{table}[htbp]
    \centering
    \begin{tabular}{>{\bfseries}l l}
        \toprule
        Discrete data & Countabe or finite \\
                      & Numbers of eggs, dice... \\
                      \midrule
        Continious Data: & Infinite number of possible values (not countable) \\
                         & Usually a \textbf{measurement}, e.g. temperature. \\
                         \bottomrule
    \end{tabular}
    \caption{Quantitative data}
    \label{tab:vocab-4}
\end{table}

\section{4 Levels of Measurement}

\begin{table}[htbp]
    \centering
    \begin{tabular}{>{\bfseries}l l}
        \toprule
        Nominal & Categories \textbf{not} ordered. e.g. religion \\
        \midrule
        Ordinal & Can be ordered, differences are meaningless \\
                & Rank, color (spectrum)...\\
        \midrule
        Interval & Ordered, differences are meaningful, no "Natural Zero" \\
                 & e.g. temperature \\
        \midrule
        Ratio & Just like interval, but with a natural zero.\\
              & e.g. amount of money  \\
        \bottomrule
    \end{tabular}
    \caption{Measurements}
    \label{tab:vocab-5}
\end{table}


\section{Design of Experiments/Observations}

\subsection{Observation vs. Experiment}

An \textbf{observation} measures specific traits, but noes \textbf{not} modify subjects.

An \textbf{experiment} applies a treatment and then measuers the effect on the subjects.


\subsection{Random}
Each member of a population, has an equal chance of being selected in a sample.

\subsubsection{Simple random sample}
Each group of size 'n' has an equal chance of being selected.

\subsection{Common techniques to get a sample}
\begin{table}[htbp]
    \centering
    \caption{4 Common techniques to get a sample}
    \begin{tabular}{>{\bfseries}l p{10cm}}
        \toprule
        Convenience sample & You use the results, which you easily get (not random) \\
        \midrule
        Systematic sampling & Put a population in some order and select every "$k^{th}$" member. \\
        \midrule
        Stratafied Sample & Breaking population into sub-groups based on some
        characteristic, and then take a simple random sample out of each
        sub-groups.\\
        \midrule
        Cluster sample & Divide population into "clusters" (regardless of
        characteristic), randomly select a certain number of clusters, and then
        collect data from the entire cluster.\\
        \bottomrule
    \end{tabular}
    \label{tab:vocab-6}
\end{table}


\chapter{Frequency Distribution}
A frequency distribution is a list of values with corresponding frequencies.

\begin{table}[htbp]
    \centering
    \begin{tabular}{>{\bfseries}l p{10cm}}
        \toprule
        Class width & Difference between two "lower class limits" \\
        \midrule
        Lower class limit & Smallest value belonging to a class \\
        \midrule
        Upper class limit & Highest value belonging to a class \\
        \bottomrule
    \end{tabular}
    \caption{Frequency Distribution Terms}
\end{table}

\steps{
    \item Determine number of classes: 8
    \item class width: \[
            \frac{\text{Max Value - Min value}}{\text{number of classes}} \leadsto
        \frac{44-18}{8} \leadsto \frac{26}{8} \leadsto 3.25 \] 
        Round \textbf{up}. $\leadsto 4$
    \item Start with the minimum value: 18
    \item Creat classes with class width (4)
    \item Find the class midpoint: \[
        \frac{\text{upper class limit + lower class limit}}{2} \leadsto  .\] 
    \item Class boundaries: used to seperate classes without gaps. }

\begin{multicols}{2}


    \textbf{class width:} 4

    \textbf{Lower class limit:} 18, 22, 26, $\ldots$ 46

    \textbf{upper class limit:} 21,25 $\ldots$ 49

    \textbf{class midpoint:}
    \[ \frac{\text{upper class limit + lower class limit}}{2}\] 
    $\leadsto 19.5, 23.5, 27.5, 31.5, 35.5, 39.5, 43.5, 47.5$

    class-width inbetween

    \textbf{class boundaries:}
    Used to seperate classes without gaps.
    17.5, 21.5, 25.5, 29.5, 33.5, 37.5, 41.5, 49.5

    \textbf{Relative frequency distribution:} Percentage \[\frac{\text{class } f.}{\sum f. (n)}\]

    \textbf{Cumulative Frequency Distribution}
    Adds sequential classes together.
\end{multicols}

\begin{table}[htbp]
    \centering
    \begin{tabular}{c|c|c|c}
        \toprule
        Age & Freq. & Rel. Freq. & Cum. Freq.\\
        \midrule
        18-21 & 25 & 58.1\% & 25  \\
        22-25 & 10 & 23.3\% & 35  \\
        26-29 & 4  & 9.3\%  & 39  \\
        30-33 & 2  & 4.7\%  & 41  \\
        34-37 & 1  & 2.3\%  & 42  \\
        38-41 & 0  & 0\%    & 42  \\
        42-43 & 1  & 2.3\%  & 43  \\
        46-49 & 0  & 0\%    & 43  \\
        \midrule
              & n=43 & 100\% \\
              & $\sum f \uparrow $ \\
              \bottomrule
    \end{tabular}
    \caption{Frequency Distribution}
\end{table}



\section{Touching Bar Chart}

\begin{figure}[ht]
    \centering
    \incfig{stats-1}
    \caption{figures/stats-1}
    \label{fig:stats-1}
\end{figure}

A cumulative chart would look exactly the same, but instead of having
boundaries numered it'd be in the middle of the bars with teh cumulative
frequency from class 1-8. And also the y-axis would be the percentage.

There is also a last one, where one takes the cumulative stuff, so that the
graph colums are getting bigger and bigger...

\textbf{Horizontal:} Class midpoints or boundaries.

\textbf{Vertical:} Frequency.



\chapter{Describing Data}

\section{5 Caracteristics} \begin{enumerate}
    \item Center
    \item Variation
    \item Distribution
    \item Outliers 
    \item Changes over time.
\end{enumerate}

\paragraph{Center}  The "middle: of the data set. 3 ways: 
\begin{enumerate}
    \item \textbf{mean:} Arthimetric Average, add all the values and divide by
        the numbers you added.

        \begin{align*}
            \text{Mean} &= \frac{\sum x}{\text{Number of values}} \\
            \sum &= \text{sum} \\
            x &= \text{data value} \\
            n &= \text{number of items in a sample} \\
            N &= \text{Number of items in a population} \\
            \overline{x} &= \text{sample mean} \\
            \mu &= \text{population mean} \\
        \end{align*}

        We can write the sameple mean then as: \[
        \overline{x} = \frac{\sum x}{n} .\] 
        And the population mean as \[
        \mu = \frac{\sum x}{N} .\] 

        \ex{ \textbf{Sample data:} 5.40, 1.10, 0.42, 0.73, 0.48, 1.10

            $\overline{x}=\frac{\sum x}{n}$ is the formula we have to use, because it's a sample, then we get:
        \[ \overline{x} = \frac{ 5.40+ 1.10+ 0.42+ 0.73+ 0.48+ 1.10 }{6} = \frac{9.23}{6} = 1.54 .\] }


    \item \textbf{Median:} The middle value of the dataset.
        \begin{itemize}
            \item Must be in order.
            \item Find middle value.
                \begin{itemize}
                    \item If odd number of values, the median is the middle number.
                    \item If even number of values, the median is the \textbf{mean} of the two middle values.
                \end{itemize}
                \ex{8, 3, 5, 11, 13, 4, 6\\
                    To find the median we first need to order em, so:\\
                    3, 4, 5, 6, 8, 11, 13.\\
                    We have seven values so we can just take the middle one which is 6.\\
                    If we'd then add 412, so our numbers are:\\
                    3, 4, 5, 6, 8, 11, 13, 412.\\
                Then our median is: $M=\frac{6+8}{2}=7$\\ }
                And it's obviously the same with decimals.
        \end{itemize}
        \tbox{
        The Median is \textbf{not} affected by outliers, the mean is.}


    \item \textbf{Mode:} The most commonly ocurring data value.
        \ex{
            \begin{enumerate}
                \item 5.40, 1.10, 0.42, 0.73, 0.48, 1.10 \\
                    Here the mode is 1.10 becuase it's ocuring most often.
                \item 27, 27, 27, 55, 55, 55, 88, 88, 89 \\ 
                    Modes: 27, 55
                \item 1, 2, 4, 7, 9, 10, 12 \\
                    Mode: $\emptyset$
        \end{enumerate} }
\end{enumerate}

\tbox{ One rounds always to one more value than the beginning values, so one
more decibel, and rounded is not before the most final step. }



%clearpage to not mess the table and text for it up.
\clearpage
\section{Mean of a frequency distribution}

\begin{table}[htbp]
    \centering
    \begin{tabular}{c|c|c|c}
        \toprule
        Age & freq. & x (midpoint) & $freq. \cdot x$ \\
        \midrule
        21-30  & 28 & 25.5 & 714  \\
        31-40  & 30 & 35.5 & 1065 \\
        41-50  & 12 & 45.5 & 546  \\
        51-60  &  2 & 55.5 & 111  \\
        61-70  &  2 & 65.5 & 131  \\
        71-80  &  2 & 75.5 & 151  \\
        \midrule
               & n=76 &    & $\sum f\cdot x = 2718$ \\
               \bottomrule
    \end{tabular}
    \caption{Another age distribution}
\end{table}

So now we can get the sample mean: \[
\overline{x}=\frac{\sum f\cdot x}{n} = \frac{2718}{76} = 35.76 .\] 


And here another table, this time about a grade's distribution.

\begin{table}[htbp]
    \centering
    \begin{tabular}{c|c|c|c}
        \toprule
          & w & points & $x\cdot w$ \\
          \midrule
        Hw & 15\% & 70 & 10.5 \\
        $T_1$ & 20\%    & 90 & 18.0 \\
        $T_2$ & 20\%    & 68 & 13.6 \\
        $T_3$ & 20\%    & 85 & 17.0 \\
        $F$ & 25\%      & 95 & 23.75 \\
        \midrule
            &           &    & $\sum x\cdot w = 82.85 $ \\
            \bottomrule
    \end{tabular}
    \caption{Grade example}
\end{table}
\[ \overline{x} = \frac{\sum x\cdot w}{\sum \cdot w} \to \frac{82.85}{100} = .8285 \to 82.85\% \] 

We can also do the same just half way in the class with the following table:

\begin{table}[htbp]
    \centering
    \begin{tabular}{c|c|c|c}
        \toprule
          & w & points & $x\cdot w$ \\
          \midrule
        Hw & 15\% & 70 & 10.5 \\
        $T_1$ & 20\%    & 90 & 18.0 \\
        $T_2$ & 20\%    & 68 & 13.6 \\
        \midrule
              & 55\%      &    & $\sum x\cdot w = 42.10 $ \\
              \bottomrule
    \end{tabular}
    \caption{Grade example}
\end{table}
\[ \overline{x} = \frac{\sum x\cdot w}{\sum \cdot w} \to \frac{42.10}{55} = .765 \to 76.50\% \] 


\begin{figure}[ht]
    \centering
    \incfig[0.6]{stats-2}
    \caption{figures/stats-2}
    \label{fig:stats-2}
\end{figure}

\vspace{50pt}


\section{Variation}

\begin{itemize}
    \item How the data is spread.
\end{itemize}

\begin{table}[htbp]
    \centering
    \begin{tabular}{c|| c|c|c|c}
        no  &   &   &   & $\overline{x}$\\
        \toprule
        no. 1 & 6 & 6 & 6  & 6 \\
        no. 2 & 4 & 7 & 7  & 6 \\
        no. 3 & 1 & 3 & 14 & 6 \\
        \bottomrule
    \end{tabular}
    \caption{Bank lines (Waiting times, different strategies.}
\end{table}

\paragraph{Ways to measure Variation}
\begin{enumerate}
    \item Range: Max Value - Min Value
        \begin{itemize}
            \item easy to find
            \item Does not consider all values
        \end{itemize}
    \item Standard deviation: 
        Measures the average distance your data values are from the mean.
        \begin{itemize}
            \item Never negative and never 0 unless all entries are the same.
            \item Greatly affected by outliers.
        \end{itemize}
\end{enumerate}
Sample standard deviation is denoted ``s''.
The formula is:
\[s= \sqrt{\frac{\sum (x-\overline{x})^2}{n-1}}  \] 
\textbf{or}
\[s= \sqrt{\frac{n\sum (x^2) - (\sum x)^2 }{n(n-1)}}   \] 
Here you dont need the mean.

\ex{Find the standard deviation of: 1, 3, 14. }

\vspace{50pt}

\begin{table}[htbp]
    \centering
    \begin{tabular}{c|c|c|c}
        %\toprule
        x & $x-\overline{x}$ & $(x-\overline{x})^2$ & \\
        \midrule
        1  & $1-6=-5$ & 25 &  \\
        3  & $3-6=-3$ & 9  &  \\
        14 & $14-6=8$ & 64 &  \\
        \midrule
           &          &$\sum (x-\overline{x})^2 = 98$
           %\bottomrule
    \end{tabular}
    \caption{table to make it easier}
\end{table}

So we get the standard deviation as:
\[
    s= \sqrt{\frac{98}{3-1}} \to s=\sqrt{\frac{98}{2}} \to s=\sqrt{49} = 7
.\] 

Now we take the other formula:
\begin{table}[htbp]
    \centering
    \begin{tabular}{c|c}
        \toprule
        $x$ & $x^2$ \\
        \midrule
        1 & 1 \\
        3 & 9 \\
        14 & 196 \\
        \midrule
        $\sum x=18$ & $\sum(x^2)=206$ \\
        \bottomrule
    \end{tabular}
    \caption{another standard deviation }
\end{table}

\[ s = \sqrt{\frac{3\cdot 206 - (18)^2}{3(3-1)}} 
= \sqrt{\frac{618-324}{3\cdot 6}} = \sqrt{\frac{294}{6}} = 7\] 


\ex{Do standard deviation on 4, 7, 7.}
One first needs the superior formula to solve it:
\[s= \sqrt{\frac{n\sum (x^2) - (\sum x)^2 }{n(n-1)}}\] 
Then it's easier when making a table to solve for some things:


\begin{table}[htbp]
    \centering
    \begin{tabular}{c|c}
        \toprule
        $x$ & $x^2$ \\
        \midrule
        4 & 16 \\
        7 & 49 \\
        7 & 49 \\
        \midrule
        $\sum x=18$ & $\sum(x^2)=114$ \\
        \bottomrule
    \end{tabular}
\end{table}

So now we just do it inside the function to get the solution:
\[
    s= \sqrt{\frac{3\cdot 114 - (18)^2}{3(3-1)}} = \sqrt{3} \approx 1.73
.\] 


\subsection{Standard deviation for a population}

The differences to a sample deviation are firstly the following notations, 
$n \leadsto N$ and $\overline{x} \leadsto \mu$.

To get sigma ($\sigma$) we do the following:

\[ \sigma = \sqrt{\frac{\sum (x-\mu)^2}{N}} .\] 


\subsection{Variance}
The number you have before you take the squareroot of the deviation formula.

\paragraph{Sample Variance:} $s^2$
\paragraph{Population Variance:} $\sigma^2$

So as an example, when $s$ is $\sqrt{49}$ it is $49$. Or when  $s$ is $7$ then
you got to squre it so you get $49$.

\paragraph{some general properties}
\begin{itemize}
    \item closely grouped data will have a small standard deviation.
    \item spread-out data will have a large standard deviation.
\end{itemize}

\section{empiric rules}
If a data set is normally distributed, we can use the \textbf{empirical rule}.
\begin{itemize}
    \item 68\% of the data will fall within 1 standard deviation of the mean.
    \item 95\% of the data will fall within 2 standard deviations of the mean.
    \item 99.7\% of the data will fall within 3 standard deviations of the mean.
\end{itemize}

If a data value lies within 2 standard deviations of the mean, it's considered
``normal''. A data values outside of 3 standard deviations is very rare ($\frac{3}{1000}$).

\paragraph{sample}
Heights are normally distributed with a mean of 65 and a standard deviation of 3.


\begin{figure}[ht]
    \centering
    \incfig[0.85]{stats-3}
    \caption{figures/stats-3}
    \label{fig:stats-3}
\end{figure}


\ex{mean is 34kg, standard deviation 8kg. What percent of data will fall
    between 10kg and 58kg? to the left one s.d. is 26, two are 18 and three are
    10. to the right three are 58.

    So with the some empiric rules we get 99.7\% of the sample within our three
    standard deviations left and right of the mean.
}

\begin{table}[htbp]
    \centering
    \begin{tabular}{c|c|c}
        \toprule
        & $\overline{x}$ s \\
        \midrule
        Height & 65 & 3 \\
        \midrule
        Weight & 175 & 7 \\
        \bottomrule
    \end{tabular}
\end{table}

\paragraph{Coefficient of Variation:}
\[ C.V. = \frac{s}{\overline{x}}\cdot 100\% .\] 

So in the top example that would be:
\[\frac{3}{65}\cdot 100\% = 4.6\% \]
\[\frac{4}{175}\cdot 100\% = 23\% .\] 


\nt{next lecture is: 3.4}



\end{document}
