\documentclass{article}
\usepackage[a4paper, left=2cm, right=2cm, top=2.5cm, bottom=2cm]{geometry}
\usepackage{amsmath}

\title{Short introduction to electrodynamics}
\begin{document}
\maketitle

\tableofcontents

\clearpage

\section{Introduction} % (fold)
\label{sec:Introduction}
Welcome to the exciting world of Electrodynamics - a branch of physics that
deals with the interaction between electrically charged particles and
electromagnetic fields. This comprehensive book delves into the fundamental
principles and advanced concepts of Electrodynamics, exploring a wide range of
topics that are essential to understanding the behavior of electrical and
magnetic fields.

The contents of this book will cover a diverse array of subjects including
Coulomb's Law, electric potential, Gauss' Law, Faraday's Law of Induction,
Maxwell's equations, the electromagnetic spectrum, electromagnetic waves in
matter, and electromagnetic interactions in particle physics. The book will
also delve into the properties of conductors, insulators, and semiconductors,
as well as the dielectric constant, permittivity, magnetic susceptibility, and
permeability.

Whether you are a student, researcher, or a professional in the field of
physics, this book provides a comprehensive overview of Electrodynamics that is
both accessible and engaging. By the end of this book, you will have a deeper
understanding of the fundamental principles of Electrodynamics and how they
apply to a wide range of physical phenomena. So get ready to explore the world
of Electrodynamics and expand your knowledge of the fascinating interaction
between electricity and magnetism.

% section section name (end)

\section{Electric fields} % (fold)
\label{sec:Electric fields}
\subsection{Coulomb's law and electric forces} % (fold)
\label{ssub:Coulomb's law and electric forces}
Coulomb's law is a fundamental law in electromagnetism that describes the
interaction between electrically charged particles. It states that the force
between two point charges is proportional to the product of the charges and
inversely proportional to the square of the distance between them.
Mathematically, Coulomb's law can be expressed as:

\[F=\frac{k\cdot q_1\cdot q_2}{r^2}\]

where F is the force between the two charges, q1 and q2 are the magnitudes of
the charges, r is the distance between the charges, and k is the Coulomb
constant. The direction of the force is along the line connecting the two
charges and can be either attractive (if the charges have opposite sign) or
repulsive (if the charges have the same sign).

Electric forces play a crucial role in many physical and biological phenomena,
ranging from the behavior of charged particles in electric and magnetic fields
to the interactions between charged proteins and other molecules in living
systems. Understanding the behavior of electric forces is therefore essential
for a wide range of scientific and engineering disciplines, including physics,
chemistry, biology, and electrical engineering.
% subsection subsection name (end)
\subsection{Electric potential and potential energy} % (fold)
\label{ssub:Electric potential and potential energy}
Electric potential, also known as voltage, is a measure of the electric
potential energy per unit charge at a given point in space. It is defined as
the amount of work required to move a unit charge from a reference point to a
given point, and is usually denoted by the symbol V. The electric potential
difference between two points is equal to the work done by the electric field
on a unit charge as it moves from one point to the other.

Potential energy is a scalar quantity that describes the amount of energy
stored in a system due to its configuration or arrangement. In the context of
electromagnetism, electric potential energy is the energy stored in an electric
field due to the presence of charged particles. The electric potential energy
of a system of charges is proportional to the magnitude of the charges and the
electric potential difference between them.

In electric circuits, electric potential energy is often stored in capacitors,
which are electrical components that store energy in an electric field. When a
charged capacitor is connected to a circuit, the electric potential energy
stored in its electric field can be converted into other forms of energy, such
as kinetic energy or thermal energy. Understanding the relationship between
electric potential, potential energy, and the behavior of charged particles in
electric fields is important for a wide range of applications, including the
design of electrical devices and the analysis of electrical circuits.
% subsection subsection name (end)
\subsection{Electric field in various configurations} % (fold)
\label{ssub:Electric field in various configurations}
The electric field is a vector field that describes the force experienced by a
charged particle in a given location. It is defined as the force per unit
charge, and its direction is that of the force on a positive test charge. The
electric field produced by a single point charge is proportional to the charge
and inversely proportional to the square of the distance from the charge. The
electric field due to a point charge can be represented by radial vectors
pointing away from the charge for positive charges and towards the charge for
negative charges.

For electric dipoles, the electric field is proportional to the separation of
the charges and the strength of the dipole moment. Electric dipoles have a net
charge of zero, but have a separation of positive and negative charges. The
electric field due to an electric dipole points from the positive charge to the
negative charge and its magnitude decreases as the inverse square of the
distance from the dipole.

For continuous charge distributions, such as a uniformly charged sphere or a
charged ring, the electric field can be calculated using Gauss's law. Gauss's
law states that the total electric flux through a closed surface is
proportional to the charge enclosed by the surface. This allows for the
calculation of the electric field produced by extended charge distributions,
such as those found in many electrical and electronic devices.

In all of these configurations, the electric field produced by the charged
particles has important implications for a wide range of physical and
biological systems. Understanding the behavior of electric fields is therefore
essential for many scientific and engineering disciplines, including physics,
chemistry, biology, and electrical engineering.
% subsection electric field in various configurations(end)
\subsection{Gauss's law and its applications} % (fold)
\label{ssub:Gauss's law and its applications}
Gauss's law is a fundamental law in electromagnetism that relates the
distribution of electric charge to the electric field it produces. It states
that the total electric flux through any closed surface is proportional to the
total charge enclosed within that surface. Mathematically, Gauss's law can be
expressed as:

\[\int \vec{E} \cdot d\vec{A} = \frac{Q}{\epsilon_0}\]

where $E$ is the electric field, $d\vec{A}$ is an infinitesimal element of the closed
surface, $Q$ is the total charge enclosed within the surface, and $\epsilon_0$ is the
vacuum permittivity.

One of the key applications of Gauss's law is to calculate the electric field
produced by a continuous charge distribution. By dividing the charge
distribution into small elements and applying Gauss's law to each element, it
is possible to determine the electric field produced by the entire
distribution. This can be especially useful for solving problems involving
extended charge distributions, such as those found in many electrical and
electronic devices.

Another important application of Gauss's law is in the design of electrical
shielding and containment systems. By carefully controlling the distribution of
electric charge on the surface of a conductive material, it is possible to
create an electric field that cancels out external electric fields, effectively
shielding the interior from external electric fields. This can be used, for
example, to protect delicate electronic components from external
electromagnetic interference.

Gauss's law also has important implications for our understanding of the
behavior of electric fields in a variety of physical and biological systems.
For example, it can be used to calculate the electric field inside a charged
conductor, or to determine the distribution of charge on the surface of a
dielectric material in the presence of an external electric field.

In conclusion, Gauss's law is a powerful tool for understanding and analyzing
the behavior of electric fields in a wide range of physical and biological
systems, and is an essential part of the study of electromagnetism.
% subsection subsection name (end)
% section electric fields(end)

\section{Magnetic fields} % (fold)
\label{sec:Magnetic fields}
\subsection{Biot-Savart law and Ampere's law} % (fold)
\label{ssub:Biot-Savart law and Ampere's law}
Biot-Savart law and Ampere's law are two important laws in electromagnetism
that describe the relationship between magnetic fields and electric currents.

Biot-Savart law is a fundamental law that describes the magnetic field produced
by an electric current. It states that the magnetic field at any point in space
is proportional to the product of the current and the vector cross product of
the distance from the point to the current and the direction of the current.
Mathematically, the Biot-Savart law can be expressed as:

\[d\vec{B} = \frac{\mu_0}{4\pi} \frac{Id\vec{l} \times \vec{r}}{r^3}\]

where $d\vec{B}$ is an infinitesimal element of magnetic field, I is the current,
$d\vec{l}$ is an infinitesimal element of current, $\vec{r}$ is the vector from the
current to the point of interest, r is the distance from the current to the
point of interest, and $μ_0$ is the vacuum permeability.

Ampere's law, on the other hand, relates the magnetic field to the electric
current that produces it. It states that the line integral of the magnetic
field around a closed loop is proportional to the current enclosed within the
loop. Mathematically, Ampere's law can be expressed as:

\[\oint \vec{B} \cdot d\vec{l} = \mu_0 I_{enc}\]

where $\vec{B}$ is the magnetic field, $d\vec{l}$ is an infinitesimal element of
the closed loop, and $I_{enc}$ is the current enclosed within the loop.

Both Biot-Savart law and Ampere's law are essential tools for understanding the
behavior of magnetic fields and electric currents, and have important
applications in a wide range of scientific and engineering disciplines,
including physics, electrical engineering, and materials science.
% subsection subsection name (end)
\subsection{Magnetic field due to current-carrying wires} % (fold)
\label{ssub:Magnetic field due to current-carrying wires}
The magnetic field due to a current-carrying wire is a fundamental concept in
electromagnetism that describes the magnetic field produced by an electric
current flowing through a wire.

According to the Biot-Savart law, the magnetic field at a point in space due to
a current-carrying wire is proportional to the current in the wire, the
distance from the wire to the point, and the tangent to the wire at that point.
The direction of the magnetic field is perpendicular to both the direction of
the current and the direction from the wire to the point.

For a long, straight wire, the magnetic field at a point perpendicular to the
wire is proportional to the current in the wire and inversely proportional to
the distance from the wire. The magnetic field is directed perpendicular to
both the wire and the line connecting the point to the wire, and is strongest
near the wire.

For a current-carrying loop, the magnetic field is proportional to the current
in the loop and the number of turns in the loop. The magnetic field is
strongest at the center of the loop and is directed perpendicular to the plane
of the loop.

The magnetic field due to a current-carrying wire has important applications in
many areas of science and technology, including electric motors, generators,
transformers, and magnetic storage devices. Understanding the behavior of the
magnetic field due to a current-carrying wire is essential for the design and
operation of these and many other electrical and electronic devices.
% subsection subsection name (end)
\subsection{Faraday’s law of induction} % (fold)
\label{ssub:Faraday’s law of induction}
Faraday's law of induction is a fundamental law of electromagnetism that
describes the relationship between magnetic fields and electric currents. The
law states that a changing magnetic field generates an electric field (also
known as an electromotive force or "EMF"). This electric field, in turn,
induces an electric current in a conductor.

Mathematically, Faraday's law can be expressed as:

\[\frac{d\Phi_B}{dt} = -\frac{d \mathcal{E}}{dt}\]

where $\frac{d\Phi_B}{dt}$ is the rate of change of the magnetic flux through a loop (ΦB), and
$-\frac{d \mathcal{E}}{dt}$ is the induced electromotive force (EMF) in the loop.

Faraday's law is an important principle in the operation of many electrical and
electronic devices, including generators, transformers, and inductors. It
provides a fundamental explanation of how energy can be transformed from one
form to another, and is a cornerstone of our understanding of electromagnetic
phenomena.

Faraday's law of induction is also closely related to Lenz's law, which states
that the direction of the induced current is such that it opposes the change in
magnetic flux that caused it. This relationship between magnetic fields and
electric currents forms the basis of electromagnetic induction and has numerous
applications in a wide range of scientific and engineering disciplines.
% subsection subsection name (end)
\subsection{Inductance and magnetic energy} % (fold)
\label{ssub:Inductance and magnetic energy}
Inductance is a measure of the ability of a coil of wire to store energy in a
magnetic field. It is defined as the ratio of the magnetic flux through the
coil to the current flowing in the coil. The unit of inductance is the henry
(H), which is the amount of inductance that will induce an electromotive force
(EMF) of one volt in a coil when the current in the coil changes at a rate of
one ampere per second.

The magnetic energy stored in a coil with inductance L and current I can be
calculated using the formula:

\[U = \frac{1}{2} LI^2\]

where U is the magnetic energy, L is the inductance, and I is the current.

This equation shows that the magnetic energy stored in a coil is proportional
to the square of the current and the inductance. This relationship is important
in the operation of devices such as transformers and inductors, where energy is
stored in magnetic fields and then transferred from one circuit to another.

Inductance is a key concept in the study of electromagnetism and is closely
related to Faraday's law of induction and Lenz's law. The ability of a coil to
store energy in a magnetic field is an important principle in many electrical
and electronic devices and has numerous applications in a wide range of
scientific and engineering disciplines.
% subsection subsection name (end)
% section section name (end)
\section{Electromagnetic waves} % (fold)
\label{sec:Electromagnetic waves}
\subsection{Maxwell's equations and the wave equation} % (fold)
\label{ssub:Maxwell's equations and the wave equation}
Maxwell's equations are a set of four partial differential equations that describe the behavior of the electric and magnetic fields in a given region of space. The equations are:

\begin{enumerate}
\item Gauss's law:
\[\nabla \cdot \vec{E} = \frac{\rho}{\epsilon_0}\]
where $\vec{E}$ is the electric field, $\rho$ is the charge density, and $\epsilon_0$ is the vacuum permittivity.

\item Gauss's law for magnetism:
\[\nabla \cdot \vec{B} = 0\]
where $\vec{B}$ is the magnetic field.

\item Faraday's law of induction:
\[\nabla \times \vec{E} = -\frac{\partial \vec{B}}{\partial t}\]
where $\vec{E}$ is the electric field and $\vec{B}$ is the magnetic field.

\item Ampere's law:
\[\nabla \times \vec{B} = \mu_0 \vec{J} + \mu_0 \epsilon_0 \frac{\partial \vec{E}}{\partial t}\]
where $\vec{B}$ is the magnetic field, $\vec{J}$ is the current density,
$\mu_0$ is the vacuum permeability, and $\epsilon_0$ is the vacuum
permittivity.
\end{enumerate}

These equations describe how electric and magnetic fields interact with charges
and currents in a given region of space. They are fundamental to the
understanding of the behavior of electromagnetic fields and are widely used in
a range of scientific and engineering disciplines.

In addition to Maxwell's equations, the wave equation is also a key concept in
electromagnetism. The wave equation describes how an electric field will
propagate through space over time:

\[\frac{\partial^2 \vec{E}}{\partial t^2} = c^2 \nabla^2 \vec{E}\]

where $\vec{E}$ is the electric field, $t$ is time, $c$ is the speed of light,
and $\nabla^2$ is the Laplacian operator. The wave equation is used to study
the behavior of electromagnetic waves, including light, radio waves, and
microwaves, among others.
% subsection subsection name (end)
\subsection{Electromagnetic wave properties} % (fold)
\label{ssub:Electromagnetic wave properties}
Electromagnetic waves are transverse waves that consist of oscillating electric
and magnetic fields. The properties of an electromagnetic wave are related to
its velocity, frequency, and wavelength. Here are some key concepts:

\begin{enumerate}
\item Velocity: The velocity of an electromagnetic wave is equal to the speed
  of light, which is approximately 299,792,458 meters per second in a vacuum.
  In a medium with a refractive index other than 1, the velocity of the wave
  will be reduced.
\item Frequency: The frequency of an electromagnetic wave is the number of
  cycles of the wave that occur per second. It is measured in Hertz (Hz) and is
  related to the wavelength and velocity of the wave.
\item Wavelength: The wavelength of an electromagnetic wave is the distance
  between two consecutive peaks of the wave. It is related to the velocity and
  frequency of the wave and is typically measured in meters.
\end{enumerate}

The relationship between the velocity, frequency, and wavelength of an
electromagnetic wave can be described by the following equation:

\[v = \lambda f\]

where $v$ is the velocity, $\lambda$ is the wavelength, and $f$ is the frequency of the wave.

The properties of an electromagnetic wave are important for a range of
applications, including communication, medical imaging, and remote sensing,
among others. Understanding these properties is essential for the design and
development of new technologies that utilize electromagnetic waves.
% subsection subsection name (end)
\subsection{Polarization and reflection/refraction of electromagnetic waves}
Polarization and reflection/refraction are two important properties of
electromagnetic waves.

Polarization refers to the direction of the electric field in an
electromagnetic wave. The electric field in an electromagnetic wave can be
polarized in any direction, but it is most often polarized in a single
direction, either vertically or horizontally. When the electric field is
polarized in a single direction, it is said to be linearly polarized. When the
electric field is polarized in multiple directions, it is said to be circularly
polarized. The polarization of an electromagnetic wave affects how it interacts
with matter and how it is affected by certain devices, such as polarizing
filters.

Reflection and refraction are two related phenomena that occur when an
electromagnetic wave encounters an interface between two different materials,
such as air and glass. Reflection occurs when an electromagnetic wave is
reflected back into the same material it came from, while refraction occurs
when an electromagnetic wave changes direction as it passes from one material
to another. The amount of reflection and refraction that occurs depends on the
properties of the two materials and the angle of incidence of the wave.

Both polarization and reflection/refraction play important roles in many areas
of science and technology, including optics, telecommunications, and remote
sensing. Understanding these properties is critical for designing and improving
a wide range of technologies and for making accurate predictions about how
electromagnetic waves will behave in different situations.
% subsection subsection name (end)
% section section name (end)

\section{Electromagnetic radiation} % (fold)
\label{sec:Electromagnetic radiation}
\subsection{Electromagnetic spectrum} % (fold)
\label{ssub:Electromagnetic spectrum}  
The electromagnetic spectrum refers to the full range of electromagnetic
radiation, from low-frequency radio waves to high-frequency gamma rays. The
different types of electromagnetic radiation in the spectrum are characterized
by their wavelength, frequency, and energy. Here is an overview of some of the
main types of electromagnetic radiation in the spectrum:

\begin{enumerate}
\item Radio Waves: Radio waves have the longest wavelength and lowest frequency
  in the electromagnetic spectrum. They are used for a range of communication
  technologies, including broadcast radio, television, and wireless
  communication.
\item Microwaves: Microwaves have a slightly shorter wavelength and higher
  frequency than radio waves. They are used for a range of applications,
  including microwave ovens, GPS, and radar.

\item Infrared: Infrared radiation has a shorter wavelength and higher
  frequency than microwaves. It is emitted by warm objects and is used for
  sensing temperature changes and remote control.

\item Visible Light: Visible light is the part of the electromagnetic spectrum
  that can be seen by the human eye. It has a relatively short wavelength and
  high frequency and is responsible for our sense of sight.

\item Ultraviolet: Ultraviolet radiation has a shorter wavelength and higher
  frequency than visible light. It is responsible for sunburn and is used for
  sterilization and black light effects.

\item X-rays: X-rays have a shorter wavelength and higher frequency than
  ultraviolet radiation. They are used for medical imaging and for penetrating
  materials to reveal their internal structure.

\item Gamma Rays: Gamma rays have the shortest wavelength and highest frequency
  in the electromagnetic spectrum. They are emitted by radioactive materials
  and are used in cancer treatment.
\end{enumerate}

The different types of electromagnetic radiation in the spectrum have different
properties and uses, and understanding the electromagnetic spectrum is
important for a wide range of scientific and technological applications.
% subsection subsection name (end)
\subsection{Properties of electromagnetic radiation} % (fold)
\label{ssub:Properties of electromagnetic radiation}
The properties of electromagnetic radiation, such as absorption, emission, and
scattering, play important roles in a wide range of natural and technological
processes.

Absorption is the process by which electromagnetic radiation is absorbed by a
material, reducing the intensity of the radiation. Absorption occurs when the
energy of the electromagnetic radiation is transferred to the atoms and
molecules in a material, causing them to become excited and eventually return
to their ground state, releasing the absorbed energy as heat or other forms of
radiation. Absorption can be selective, meaning that certain frequencies of
electromagnetic radiation are absorbed more or less than others, depending on
the properties of the material. This is the basis for many spectroscopic
techniques, such as infrared spectroscopy and ultraviolet-visible spectroscopy.

Emission is the process by which electromagnetic radiation is released by a
material. Emission can occur when excited atoms and molecules return to their
ground state, releasing energy in the form of electromagnetic radiation. This
is the basis for many light-emitting technologies, such as light-emitting
diodes (LEDs) and fluorescent lights.

Scattering is the process by which electromagnetic radiation is redirected or
dispersed as it passes through a material. Scattering occurs when the
electromagnetic radiation interacts with the atoms and molecules in a material,
causing it to change direction. Scattering can cause the radiation to become
diffuse, reducing the intensity and clarity of the image, or it can cause the
radiation to be reflected in specific directions, producing a pattern known as
Rayleigh scattering.

These properties are important for a wide range of scientific and technological
applications, including remote sensing, medical imaging, and lighting
technology. Understanding the properties of electromagnetic radiation is
critical for developing and improving these technologies and for making
accurate predictions about how electromagnetic radiation will behave in
different situations.
% subsection subsection name (end)
% section section name (end)

\section{Electromagnetic properties of materials} % (fold)
\label{sec:Electromagnetic properties of materials}
\subsection{Conductors, insulators, and semiconductors} % (fold)
\label{ssub:Conductors, insulators, and semiconductors}
Conductors, insulators, and semiconductors are different types of materials
that have different electromagnetic properties.

Conductors are materials that allow electric charge to flow freely through
them. Metals, such as copper and aluminum, are good conductors. This makes them
useful in many electrical and electronic applications, such as in electrical
wires, conductors in electrical circuits, and in electrical components.

Insulators are materials that do not allow electric charge to flow through them
easily. Glass, air, rubber, and some plastics are good insulators. This makes
them useful in many electrical and electronic applications, such as in
electrical insulation, protective coatings, and in electrical insulators.

Semiconductors are materials that have electrical conductivity between that of
conductors and insulators. Silicon and germanium are two examples of
semiconductors. Semiconductors are useful in many electronic devices, such as
transistors and diodes, because of their unique electronic properties that
allow them to be used to control and manipulate the flow of electric charge.

The conductivity, or ability to conduct electric charge, of a material is
dependent on the type and number of electrons available in its atomic
structure. By controlling the number of electrons in a material, it is possible
to change its conductivity and make it into a conductor, insulator, or
semiconductor. This makes it possible to create many different types of
electronic devices and to customize their properties for specific applications.
% subsection subsection name (end)
\subsection{Dielectric constant and permittivity} % (fold)
\label{ssub:Dielectric constant and permittivity}
The dielectric constant and permittivity are related concepts in
electromagnetism that describe the ability of a material to store electrical
energy in an electric field.

The dielectric constant, also known as the relative permittivity, is a
dimensionless constant that describes the ratio of the electrical energy stored
in a material to the energy stored in a vacuum. The dielectric constant of a
material is a measure of its electrical polarization, or the ability of its
electrons to be re-arranged in response to an electric field.

The permittivity, also known as the electric constant, is a measure of the
ability of a material to store electrical energy in an electric field. It is a
measure of the electric field strength required to produce a unit charge
density in a material. The permittivity is related to the dielectric constant
by the formula:

$$\epsilon = \epsilon_0 \kappa$$

where $\epsilon_0$ is the permittivity of free space, which has a value of
approximately 8.85 x 10$^{-12}$ farads per meter, and $\kappa$ is the
dielectric constant of the material.

The dielectric constant and permittivity are important parameters in many
electrical and electronic applications, including capacitors, electric power
transmission, and in the design of electrical insulation materials. They are
also used to describe the behavior of materials in electric fields, including
the amount of energy that is stored in a material in response to an electric
field, and the efficiency with which a material can store and release
electrical energy.
% subsection subsection name (end)
\subsection{Magnetic susceptibility and permeability} % (fold)
\label{ssub:Magnetic susceptibility and permeability}
The magnetic susceptibility of a material refers to how susceptible it is to
being magnetized in the presence of a magnetic field. It is a dimensionless
scalar quantity that is defined as the ratio of the magnetic induction produced
in a material to the magnetic field strength applied to it. The magnetic
permeability of a material is a measure of its ability to support the formation
of a magnetic field within it. It is a scalar value that is usually represented
as a relative permeability, which is the ratio of the magnetic induction
produced in a material to the magnetic field strength applied to it.

Together, the magnetic susceptibility and magnetic permeability of a material
determine how the material will respond to magnetic fields. For example,
magnetic materials such as iron have high magnetic susceptibility and
permeability, and they are therefore able to support strong magnetic fields. On
the other hand, materials with low magnetic susceptibility and permeability,
such as copper and aluminum, are not very susceptible to being magnetized and
cannot support strong magnetic fields.
% subsection subsection name (end)
% section section name (end)

\section{Electrodynamic phenomena} % (fold)
\label{sec:Electrodynamic phenomena}
\subsection{Electric circuits} % (fold)
\label{ssub:Electric circuits}
Electric circuits refer to networks of electrical components that are connected
together to perform a specific task, such as the transmission or transformation
of electrical energy. In the context of electromagnetism, electric circuits can
be used to study and understand a variety of electromagnetic phenomena.

Resistors are electrical components that are designed to resist the flow of
electric current, and they are often used to control the amount of current in a
circuit. Capacitors are components that store electrical energy in an electric
field, and they are used in circuits to smooth out fluctuations in current.
Inductors are components that store energy in a magnetic field, and they play a
key role in circuits that involve alternating current (AC).

AC circuits are circuits that involve the use of AC current, which is a type of
current that alternates in direction. AC circuits are used in a variety of
applications, such as power transmission and the operation of electrical
motors. On the other hand, DC circuits are circuits that involve the use of
direct current, which is a type of current that flows in one direction. DC
circuits are commonly used in batteries and in applications that require a
constant voltage.

In electric circuits, the interplay between resistors, capacitors, and
inductors can result in a variety of interesting and useful phenomena, such as
oscillations, filtering, and the creation of AC waveforms. The study of
electric circuits and their behavior is an important part of electromagnetism
and electrical engineering.
% subsection subsection name (end)
\subsection{Electromagnetic waves in matter} % (fold)
\label{ssub:Electromagnetic waves in mattersubsection name}
In the context of electromagnetics, electromagnetic waves in matter refers to
the behavior of electromagnetic waves when they propagate through different
types of media, such as plasmas, waveguides, and resonators.

Plasmas are a type of matter composed of ions and free electrons, and they
exhibit unique behavior when subjected to electromagnetic fields. For example,
plasmas can support certain types of electromagnetic waves that cannot exist in
other types of media.

Waveguides are structures that are designed to confine and guide
electromagnetic waves, often over long distances. They have a variety of
applications, such as in microwave communication systems, where they are used
to transmit signals between antennas.

Resonators are structures that are designed to store electromagnetic energy and
provide it in a controlled manner. They are used in many different
applications, such as in microwave and millimeter-wave filters, where they are
used to pass desired signals and reject unwanted ones.

In each of these examples, the behavior of electromagnetic waves is influenced
by the properties of the medium in which they are propagating, and this can
affect the wave's velocity, frequency, wavelength, and other properties.
% subsection subsection name (end)
\subsection{Electromagnetic fields in moving media} % (fold)
\label{ssub:Electromagnetic fields in moving media}
In the context of electromagnetic properties of materials, Electromagnetic
fields in moving media refer to the behavior of electromagnetic fields in media
that are in motion relative to an observer. This involves studying the
transformation of electromagnetic fields under Lorentz transformations, which
describe the change in space and time coordinates between two observers in
relative motion.

The study of electromagnetic fields in moving media is also known as
relativistic electrodynamics, which is a branch of electrodynamics that takes
into account the effects of special relativity on electromagnetic fields. In
this context, the theory of special relativity is applied to the equations of
electromagnetism, resulting in a more accurate and complete description of the
behavior of electromagnetic fields in moving media.
% subsection subsection name (end)
\subsection{Electromagnetic interactions in particle physics} % (fold)
\label{ssub:Electromagnetic interactions in particle physics}
In particle physics, electromagnetic interactions refer to the way charged
particles interact with electromagnetic fields. This is one of the four
fundamental forces in the universe, along with the strong force, weak force,
and gravitational force. Electromagnetic interactions play a crucial role in
many processes, including the behavior of electrons in atoms and the
interactions between charged particles in nuclear systems. Some examples of
electromagnetic interactions in particle physics include:

\begin{enumerate}
  \item The electromagnetic force between two charged particles, which can be described by Coulomb's law.
  \item The energy levels and spectral lines of atoms, which are due to the interaction between electrons and the electromagnetic field produced by the nucleus.
  \item The scattering of charged particles by electromagnetic fields, which is important in many particle physics experiments.
  \item The production and decay of particles through electromagnetic processes, such as the electromagnetic decay of a meson into a photon and two electrons.
\end{enumerate}

In general, the study of electromagnetic interactions in particle physics helps
us to understand the behavior of charged particles on a small scale and to gain
insight into the fundamental forces of nature.
% subsection subsection name (end)
% section section name (end)

\section{Formulary} % (fold)
\label{sec:Formulary}

\begin{enumerate}
  \item Coulomb's Law: $$\vec{F} = \frac{1}{4\pi\epsilon_0} \frac{q_1q_2}{r^2} \hat{r}$$
  \item Electric Field Intensity:$$ \vec{E} = \frac{1}{4\pi\epsilon_0} \frac{Q}{r^2} \hat{r}$$
  \item Gauss's Law:$$ \frac{1}{\epsilon_0} \oint \vec{E} \cdot d\vec{A} = Q_{\text{enc}}$$
  \item Electric Potential:$$ V = -\frac{1}{4\pi\epsilon_0} \int \frac{Q}{r} dr$$
  \item Biot-Savart Law:$$ \vec{B} = \frac{\mu_0}{4\pi} \int \frac{I \vec{dl} \times \vec{r}}{r^2}$$
  \item Ampere's Law:$$ \oint \vec{B} \cdot d\vec{l} = \mu_0 I_{\text{enc}}$$
  \item Faraday's Law of Induction:$$ \frac{d\phi}{dt} = -\frac{d\vec{A}}{dt} = -\frac{\partial \vec{B}}{\partial t}$$
  \item Inductance:$$ L = \frac{\phi}{I}$$
  \item Magnetic Energy:$$ W = \frac{1}{2} LI^2$$
  \item Maxwell's Equations:
    \begin{align*}
    \nabla \cdot \vec{E} &= \frac{\rho}{\epsilon_0} \
    \nabla \cdot \vec{B} &= 0 \
    \nabla \times \vec{E} &= -\frac{\partial \vec{B}}{\partial t} \
    \nabla \times \vec{B} &= \mu_0 \left(\vec{J} + \epsilon_0 \frac{\partial \vec{E}}{\partial t}\right)
    \end{align*}
  \item Electromagnetic Wave Equation:$$ \frac{1}{c^2} \frac{\partial^2 \vec{E}}{\partial t^2} = \nabla^2 \vec{E}$$
  \item Electromagnetic Waves Velocity:$$ v = \frac{\lambda}{T} = c$$
  \item Electromagnetic Waves Frequency:$$ f = \frac{1}{T}$$
  \item Electromagnetic Waves Wavelength: $$\lambda = \frac{c}{f}$$
  \item Dielectric Constant:$$ \epsilon_r = \frac{\epsilon}{\epsilon_0}$$
  \item Permittivity:$$ \epsilon = \epsilon_0 \epsilon_r$$
  \item Magnetic Susceptibility:$$ \chi = \frac{\mu}{\mu_0}$$
  \item Permeability: $$\mu = \mu_0 \chi$$
  \item Ohm's Law: $$V = IR$$
  \item Power: $$P = IV = \frac{V^2}{R} = I^2R$$
\end{enumerate}
% section section name (end)
\end{document}
