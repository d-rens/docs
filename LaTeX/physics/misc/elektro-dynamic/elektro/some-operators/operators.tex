\documentclass{article}
\usepackage{amsmath}
\usepackage[a4paper, right=2cm, left=2cm, top=2.5cm, bottom= 3cm]{geometry}

\begin{document}
\section{first try} % (fold)
\paragraph{prompt} % (fold)
explain to me all operators used in electro dynamics, written in a new latex document.
% paragraph paragraph name (end)
In electro-dynamics, the main operators used are:
\begin{itemize}
    \item \textbf{Curl Operator ($\nabla \times$):} This operator is used to calculate the curl of a vector field. It is defined as $\nabla \times \textbf{A}=\left(\frac{\partial A_z}{\partial y}-\frac{\partial A_y}{\partial z}\right)\hat{\textbf{i}} + \left(\frac{\partial A_x}{\partial z}-\frac{\partial A_z}{\partial x}\right)\hat{\textbf{j}} + \left(\frac{\partial A_y}{\partial x}-\frac{\partial A_x}{\partial y}\right)\hat{\textbf{k}}$, where $\textbf{A}$ is a vector field and $\hat{\textbf{i}},\hat{\textbf{j}}$ and $\hat{\textbf{k}}$ are the unit vectors in the x, y and z directions respectively. 
    \item \textbf{Divergence Operator ($\nabla \cdot$):} This operator is used to calculate the divergence of a vector field. It is defined as $\nabla \cdot \textbf{A}=\frac{\partial A_x}{\partial x}+\frac{\partial A_y}{\partial y}+\frac{\partial A_z}{\partial z}$, where $\textbf{A}$ is a vector field.
    \item \textbf{Laplacian Operator ($\nabla^2$):} This operator is used to calculate the Laplacian of a scalar or vector field. It is defined as $\nabla^2 f=\frac{\partial^2 f}{\partial x^2}+\frac{\partial^2 f}{\partial y^2}+\frac{\partial^2 f}{\partial z^2}$, where $f$ is a scalar or vector field.
    \item \textbf{Gradient Operator ($\nabla$):} This operator is used to calculate the gradient of a scalar or vector field. It is defined as $\nabla f=\left(\frac{\partial f}{\partial x}\right)\hat{\textbf{i}}+\left(\frac{\partial f}{\partial y}\right)\hat{\textbf{j}}+\left(\frac{\partial f}{\partial z}\right)\hat{\textbf{k}}$, where $f$ is a scalar or vector field and $\hat{\textbf{i}},\hat{\textbf{j}}$ and $\hat{\textbf{k}}$ are the unit vectors in the x, y and z directions respectively.
\end{itemize}

\section{second try} % (fold)
\paragraph{prompt} % (fold)
Can you write a latex document about all variables used in electro dynamics?
please explain them all because i dont know how to use them or what they are
there for/supposed to mean

\subsection{Variables Used in Electrodynamics}

Electrodynamics is a branch of physics that studies the interaction of electric
charges and magnetic fields. In order to properly describe this interaction,
various variables are used. In this document, we will discuss the meaning and
use of each of these variables.

\subsubsection{Electric Field} The electric field, denoted by $\vec{E}$, is a
vector field that describes the force exerted by electric charges on other
electric charges. It is measured in units of newtons per coulomb (N/C).

\subsubsection{Magnetic Field} The magnetic field, denoted by $\vec{B}$, is a
vector field that describes the force exerted by magnetic materials on electric
charges. It is measured in units of tesla (T).

\subsubsection{Electric Potential} The electric potential, denoted by $V$, is a
scalar field that describes the potential energy of a system of electric
charges. It is measured in units of volts (V).

\subsubsection{Magnetic Potential} The magnetic potential, denoted by $\phi$, is a
scalar field that describes the potential energy of a system of magnetic
materials. It is measured in units of webers (Wb).

\subsubsection{Electric Current} The electric current, denoted by $I$, is a
measure of the flow of electric charge. It is measured in units of amperes (A).

\subsubsection{Magnetic Flux} The magnetic flux, denoted by $\Phi$, is a measure
of the strength of the magnetic field. It is measured in units of webers (Wb).

\subsubsection{Electric Dipole Moment} The electric dipole moment, denoted by
$\vec{\mu}$, is a measure of the charge separation in a system of electric
charges. It is measured in units of coulombs per meter (C/m).

\subsubsection{Magnetic Dipole Moment} The magnetic dipole moment, denoted by
$\vec{m}$, is a measure of the magnetic moment of a system of magnetic
materials. It is measured in units of ampere-m2 (A-m2).

\subsubsection{Electric Charge Density} The electric charge density, denoted by
$\rho$, is a measure of the amount of electric charge in a given system. It is
measured in units of coulombs per cubic meter (C/m3).

\subsubsection{Magnetic Charge Density} The magnetic charge density, denoted by
$\sigma$, is a measure of the amount of magnetic charge in a given system. It
is measured in units of ampere-m2 (A-m2).

\subsubsection{Electric Intensity} The electric intensity, denoted by $E$, is a
measure of the strength of the electric field. It is measured in units of volts
per meter (V/m).

\subsubsection{Magnetic Intensity} The magnetic intensity, denoted by $H$, is a
measure of the strength of the magnetic field. It is measured in units of
amperes per meter (A/m).

\subsubsection{Electric Displacement} The electric displacement, denoted by
$\vec{D}$, is a measure of the amount of electric charge in a given system. It
is measured in units of coulombs per meter (C/m).

\subsubsection{Magnetic Displacement} The magnetic displacement, denoted by
$\vec{B}$, is a measure of the amount of magnetic charge in a given system. It
is measured in units of ampere-m2 (A-m2).

\subsubsection{Electric Polarization} The electric polarization, denoted by
$\vec{P}$, is a measure of the electric dipole moment of a system of electric
charges. It is measured in units of coulombs per meter (C/m).

\subsubsection{Magnetic Polarization} The magnetic polarization, denoted by
$\vec{M}$, is a measure of the magnetic dipole moment of a system of magnetic
materials. It is measured in units of ampere-m2 (A-m2).

\subsubsection{Lorentz Force} The Lorentz force, denoted by $\vec{F}$, is the
force exerted by an electric field on a moving charge. It is measured in units
of newtons (N).

\section{third try} % (fold)
\label{sec:third try} 
\paragraph{prompt 3} forgor% (fold)
\label{par:prompt 3paragraph name}
% paragraph paragraph name (end)
\begin{align} 
\textbf{Easy Calculations:}\\ 
\text{Coulomb's Law:} \quad  F &= \frac{q_1q_2}{4\pi\epsilon_0r^2} \\ 
\text{Electric Field:} \quad E &= \frac{F}{q_1} = \frac{q_2}{4\pi\epsilon_0r^2}\\ 
\text{Electric Potential:} \quad V &= \frac{q_2}{4\pi\epsilon_0r} \\ 
\end{align}

\begin{align} 
\textbf{Moderate Calculations:} \\
\text{Induced Electric Field:} \quad E_{induced} &= -\frac{\Delta V}{\Delta t} \\ 
\text{Electric Flux:} \quad \Phi &= \int_S \mathbf{E} \cdot d\mathbf{A} \\ 
\text{Electric Dipole Moment:} \quad \mathbf{p} &= q\mathbf{r} \\ 
\text{Electric Field at a Point due to a Dipole:} \quad E &= \frac{1}{4\pi\epsilon_0}\frac{3(\mathbf{p}\cdot\mathbf{r})\mathbf{r}}{|\mathbf{r}|^5} - \frac{1}{4\pi\epsilon_0}\frac{\mathbf{p}}{|\mathbf{r}|^3}
\end{align}

\begin{align}
\textbf{Hard Calculations:} \\
\text{Electric Field of a Continuous Charge Distribution:} \quad E &= \frac{1}{4\pi\epsilon_0}\int \frac{\rho(\mathbf{r'})}{|\mathbf{r}-\mathbf{r'}|^3}(\mathbf{r}-\mathbf{r'})dV' \\ 
\text{Magnetic Field of a Continuous Current Distribution:} \quad B &= \frac{\mu_0}{4\pi}\int \frac{I(\mathbf{r'})\times(\mathbf{r}-\mathbf{r'})}{|\mathbf{r}-\mathbf{r'}|^3}dV' \\
\text{Electromagnetic Wave Equation:} \quad \left(\nabla^2 - \frac{1}{c^2}\frac{\partial^2}{\partial t^2}\right)\mathbf{E} &= \frac{\mu_0}{4\pi}J + \mu_0\varepsilon_0\frac{\partial^2 \mathbf{E}}{\partial t^2}
\end{align}
% section section name (end)
\end{document}
