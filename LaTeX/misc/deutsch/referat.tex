\documentclass{article}

\usepackage[a4paper,margin=3cm]{geometry}
\usepackage{biblatex}

\addbibresource{german.bib}
\setlength{\parindent}{0pt}

\title{Hermann Hesses Werke und der Existenzialismus}
\author{Daniel Renschler}
\date{}

\begin{document}
\maketitle

\section{Ideen}
\textbf{Hermann Hesses Werke und der Existenzialismus}
\begin{itemize}
  \item Betrachtung der existenzialistischen Merkmale in Hesses Werken, wie
    z.B. die Suche nach dem Sinn des Lebens, die Auseinandersetzung mit der
    eigenen Existenz und die Betonung der Individualit\"at.
  \item Vergleich von Hesses Werken mit dne Ideen der Existenzialisten, wie
    z.B. Nietzsche (Zarathustra), und Diskussion der Parallelen und
    Unterschiede.
  \item Untersuchung, wie existenzialistische Themen in Hesses Werken den Leser
    dazu anregen, \"uber sein eigenes Leben und seine eigenen Entscheidungen
    nachzudenken.
\end{itemize}

B\"ucher die man dazu verwenden kann: ~\cite{hesse1927steppenwolf},
~\cite{nietzsche1883also}, ~\cite{hesse1922siddhartha},
~\cite{hesse1930narziss} und ~\cite{nietzsche1886jenseits}.



\printbibliography

\end{document}
