\documentclass[a4paper]{report}
\usepackage[a4paper,margin=3.5cm]{geometry}
\usepackage{cancel}
\usepackage{tikz}
\usepackage{amsmath,amsfonts,amssymb, amsthm}
\usepackage{xcolor}
\usepackage{tcolorbox}
\usepackage{polynom}
\usepackage{wrapfig}
\usepackage{booktabs}
\usepackage{tabularx}
\usepackage{multicol}
%\usepackage{hyperref}
%\usepackage{babel}
%\usepackage{titlesec}
\usepackage{tikz}
\usepackage{pgfplots}
\pgfplotsset{compat=1.18}

\usepackage{pstricks}
\usepackage{pst-plot}
\usepackage{textcomp}
\usepackage{import}
\usepackage{pdfpages}
%\usepackage{transparent}


\newcommand{\incfig}[2][1]{%
    \def\svgwidth{#1\columnwidth}
    \import{./figures/}{#2.pdf_tex}
}

%\setlength{\parindent}{0pt}
\usepackage{parskip}


% makes arrows shorter
\let\implies\Rightarrow
\let\impliedby\Leftarrow
\let\iff\Leftrightarrow
\let\epsilon\varepsilon

% things i'll surly need somewhen, but not yet:
\newcommand\N{\ensuremath{\mathbb{N}}}
\newcommand\R{\ensuremath{\mathbb{R}}}
\newcommand\Z{\ensuremath{\mathbb{Z}}}
\renewcommand\O{\ensuremath{\emptyset}}
\newcommand\Q{\ensuremath{\mathbb{Q}}}
\newcommand\C{\ensuremath{\mathbb{C}}}


% defining chapter so i can use it in non-book classes
%\titleformat{\chapter}[display]
  %{\normalfont\huge\bfseries}{\chaptertitlename\ \thechapter}{20pt}{\Huge}
%\titlespacing*{\chapter}{0pt}{50pt}{40pt}
%%\newcommand{\chapterbreak}{\clearpage}



% nice looking dices for probability
\font\domino=domino
\def\die#1{{\domino#1}}



% Boxes
\newcommand{\tbox}[2][0.8\linewidth]{
    \begin{center}
        \begin{tcolorbox}[colback=white, colframe=gray, width=#1]
            #2
        \end{tcolorbox}
\end{center}}

\newcommand{\regel}[2][0.8\linewidth]{
    \begin{center}
        \begin{tcolorbox}[title=Regel, colback=white, colframe=blue, width=#1]
            #2
        \end{tcolorbox}
\end{center}}

\newcommand{\nt}[2][0.9\linewidth]{
    \begin{center}
        \begin{tcolorbox}[title=Notiz:, colback=white, colframe=gray, width=#1]
            #2
        \end{tcolorbox}
\end{center}}


\newcommand{\ex}[2][\linewidth]{
    \begin{center}
        \begin{tcolorbox}[title=Example, colback=white, colframe=brown, width=#1]
            #2
        \end{tcolorbox}
\end{center}}


\newcommand{\q}[2][\linewidth]{
    \begin{center}
        \begin{tcolorbox}[title=Frage:, colback=white, colframe=purple, width=#1]
            #2
        \end{tcolorbox}
\end{center}}


\newcommand{\ff}[2][\linewidth]{
    \begin{center}
        \begin{tcolorbox}[title=Forschungsfrage:, colback=white, colframe=black, width=#1]
            #2
        \end{tcolorbox}
\end{center}}


\newcommand{\steps}[2][0.8\linewidth]{
    \begin{center}
        \begin{tcolorbox}[title=Steps:, colback=white, colframe=blue, width=#1]
            \begin{enumerate}
                #2
            \end{enumerate}
        \end{tcolorbox}
\end{center}}


\usepackage{xifthen}
\makeatother
\def\@lecture{}%
\newcommand{\lecture}[3]{
    \ifthenelse{\isempty{#3}}{%
        \def\@lecture{Lecture #1}%
    }{%
        \def\@lecture{Lecture #1: #3}%
    }%
    \subsection*{\@lecture}
    \marginpar{\small\textsf{\mbox{#2}}}
}
\makeatletter

\author{Daniel Renschler}


%% For linear algebra


\newcommand{\atrix}[2]{%
  \left[
  \begin{array}{#1}
  #2
  \end{array}
  \right]
}


\input{/Users/daniel/docs/LaTeX/template-related/sir-mars-template/letterfonts}
\usepackage[ngerman]{babel}

\begin{document}

\title{\Huge{Physikhausaufgabe}\\Magnetfelder in Natur, Forschung und Technik}
\author{\huge{Daniel Renschler}}
\date{\today}
\maketitle

\section*{Wo werden Magnete in der Natur, Forschung und Technik genutzt?}
    \begin{itemize}
        \item Schrottplatz (in Form eines Elektromagnets)
            Hierbei werden meist bagger mit Elektromagneten genutzt, die dann magnetischen Schrott von nicht magnetischem trennen.
            \begin{figure}[htpb]
                \centering
                \includegraphics[width=0.4\textwidth]{Bagger.jpeg}
                \caption{Bagger mit Elektromagnet}
                \label{fig:Bagger}
            \end{figure}
%%%%%%%%%%%%%%%%%%%%%%%%%%%%%%%%%%%%%%%%%%%%%%%%%%%%%%%%%%%%%%%%%%
        \item Natur, in form von geladenen Teilen die auf Luftmoleküle treffen in Polarlichtern.
            Durch Sonnenwinde werden Protonen und Elektronen mit $400 \frac{km}{h}$ in das All geströmt. Die erreichen in wenigen Tagen unser erdisches Magnetfeld und deformieren es, ist aber meist nicht möglich einzudringen. Durch die deformierung werden die B-Feldlinien umkreist, was die Atmosphäre wie eine art dynamo antreibt, was die E-Felder Beschleunigt. Diese Elektronen werden dann in der nähe der Pole in ungefähr 100km Höhe gesehen werden. Dort steigen sie in oberen Luftschichten mit gerigner Dirchte hinab, die unter dem niedrigen Druck Atmosphäre wird zum leuchten angeregt und so entstehen dann Polarlichter. (Bei Sonnenfleckentätigkeit).
            \begin{figure}[htpb]
                \centering
                \includegraphics[width=0.4\textwidth]{Polarlicht.jpeg}
                \caption{Polarlicht}
                \label{fig:Polarlicht}
            \end{figure}
%%%%%%%%%%%%%%%%%%%%%%%%%%%%%%%%%%%%%%%%%%%%%%%%%%%%%%%%%%%%%%%%%%
        \item Strahlungsgürtel sind in 700km bis 60 000km höhe und können für Raumfahrer gefährlich werden. Ihre Protonen und Elektronen wirken wie ionisierende Strahlung. Sie umkreisen die B-Feldlinien  und pendeln einige minuten zwischen Nord- und Südpol der Erde hin und her, tragen aber nicht zum Polarlicht bei\footnote{Satz komplett übernommen weil er im Buch schon gut war.}. 
            \begin{figure}[htpb]
                \centering
                \includegraphics[width=0.4\textwidth]{Strahlungsgürtel.jpeg}
                \caption{Strahlungsguertel.jpeg}
                \label{fig:Strahlungsguertel-jpeg}
            \end{figure}
%%%%%%%%%%%%%%%%%%%%%%%%%%%%%%%%%%%%%%%%%%%%%%%%%%%%%%%%%%%%%%%%%%%
            \clearpage
        \item Penningfalle:
            Ist eine Falle für Elektronen, um genau ein Elektron einzufangen um es in einem kleinen Raubgebiet festzuhalten.
            Man verwendet dazu verschiedene Methoden, z. B. die Penning-Falle für geladene Teilchen. Hier nutzt man ein homogenes, zeitlich konstantes Magnetfeld und ein inhomogenes, ebenfalls zeitlich konstantes elektrisches Feld, um die Teilchen in einem kleinen Raumgebiet, dem Falleninneren, zu konzentrieren.\footnote{siehe Fußnote 1} Das Magnetfeld zwingt die Teilchen auf gekrümmte Bahnen, welche nciht verlassen werden können in vertikaler Richtung.
            All dies ist Technisch anspruchsvoll. Hans Dehmelt pionierte diese Technik und erhielt den Physik Nobelpreis dafür 1989. In Penning Fallen lassen sich auch Ionen und Antiprotonen speichern. In einem aktuellen Forschungsprojekt der Universität Heidelberg wurde mithilfe einer Penning-Falle und eines hochionisierten Kohlenstoffions die Masse des Elektrons mit einer 13-mal größeren Genauigkeit als bislang bestimmt.\footnote{sieht Fußnote 2}

        \item Die magnetische Linse, hier in einer Schattenekreuz Röhre:
            Hier werden Elektronen beschleunigt mit  4kV, dann fliegen sie auf ein Aluminiumkreuz und am Schatten des Leuchtschirms erkennt man  das sie sich geradlinig ausbreiten, wie in der Abbildung unterhalb.
            \begin{figure}[htpb]
                \centering
                \includegraphics[width=0.4\textwidth]{Magnetische-Linsse.jpeg}
                \caption{magnetische Linse}
                \label{fig:magnetische-Linse}
            \end{figure}
            Mit rotationssymetrischen, inhomogenenen B-Feldern erziehlt man erhebliche Vergrößerungen. Diese werden von Strom führenden Spulen geriner Größe erzeugt, die von einem Eisenpanzer eingehüllt sind. So wird das B-Feld auf einen einen kleinen Raum nahe der Achse konzentriert. Die B-Felder solcher Spulen bezeichnet man als magnetische Linsen \footnote{siehe Fußnote 3} (Abbildung 5).
            \begin{figure}[htpb]
                \centering
                \includegraphics[width=0.4\textwidth]{Prinzip-linse.jpeg}
                \caption{Prinzip der Magnetischen Linse}
                \label{fig:Prinzip-der-Magnetischen-Linse}
            \end{figure}
%%%%%%%%%%%%%%%%%%%%%%%%%%%%%%%%%%%%%%%%%%%%%%%%%%%%%%%%%%%%%%%%%
        \item Fesplatte (HD):
            Bei einer Magnetfestplatte (HD) werden die Daten in Form von langer Reihen von Matgnetisierungsrichtungen gespeichert.
            \begin{figure}[htpb]
                \centering
                \includegraphics[width=0.2\textwidth]{Festplatte+magnet.jpeg}
                \caption{Festplatte und Aufbau dazu}
                \label{fig:}
            \end{figure}
    \end{itemize}

\end{document}
