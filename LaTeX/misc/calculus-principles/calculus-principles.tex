\documentclass[a4paper, 12p]{report} 
%\usepackage[export]{adjustbox}
%\usepackage{biblatex}
%\usepackage{relsize}
%\usepackage[a4paper,left=2.5cm,right=2.5cm,top=3cm,bottom=3cm]{geometry}
%\usepackage{wrapfig}
%\usepackage{graphicx}
%\usepackage{pgfplots}
%\pgfplotsset{compat=1.18}
\usepackage[a4paper,margin=3.5cm]{geometry}
\usepackage{cancel}
\usepackage{tikz}
\usepackage{amsmath,amsfonts,amssymb, amsthm}
\usepackage{xcolor}
\usepackage{tcolorbox}
\usepackage{polynom}
\usepackage{wrapfig}
\usepackage{booktabs}
\usepackage{tabularx}
\usepackage{multicol}
%\usepackage{hyperref}
%\usepackage{babel}
%\usepackage{titlesec}
\usepackage{tikz}
\usepackage{pgfplots}
\pgfplotsset{compat=1.18}

\usepackage{pstricks}
\usepackage{pst-plot}
\usepackage{textcomp}
\usepackage{import}
\usepackage{pdfpages}
%\usepackage{transparent}


\newcommand{\incfig}[2][1]{%
    \def\svgwidth{#1\columnwidth}
    \import{./figures/}{#2.pdf_tex}
}

%\setlength{\parindent}{0pt}
\usepackage{parskip}


% makes arrows shorter
\let\implies\Rightarrow
\let\impliedby\Leftarrow
\let\iff\Leftrightarrow
\let\epsilon\varepsilon

% things i'll surly need somewhen, but not yet:
\newcommand\N{\ensuremath{\mathbb{N}}}
\newcommand\R{\ensuremath{\mathbb{R}}}
\newcommand\Z{\ensuremath{\mathbb{Z}}}
\renewcommand\O{\ensuremath{\emptyset}}
\newcommand\Q{\ensuremath{\mathbb{Q}}}
\newcommand\C{\ensuremath{\mathbb{C}}}


% defining chapter so i can use it in non-book classes
%\titleformat{\chapter}[display]
  %{\normalfont\huge\bfseries}{\chaptertitlename\ \thechapter}{20pt}{\Huge}
%\titlespacing*{\chapter}{0pt}{50pt}{40pt}
%%\newcommand{\chapterbreak}{\clearpage}



% nice looking dices for probability
\font\domino=domino
\def\die#1{{\domino#1}}



% Boxes
\newcommand{\tbox}[2][0.8\linewidth]{
    \begin{center}
        \begin{tcolorbox}[colback=white, colframe=gray, width=#1]
            #2
        \end{tcolorbox}
\end{center}}

\newcommand{\regel}[2][0.8\linewidth]{
    \begin{center}
        \begin{tcolorbox}[title=Regel, colback=white, colframe=blue, width=#1]
            #2
        \end{tcolorbox}
\end{center}}

\newcommand{\nt}[2][0.9\linewidth]{
    \begin{center}
        \begin{tcolorbox}[title=Notiz:, colback=white, colframe=gray, width=#1]
            #2
        \end{tcolorbox}
\end{center}}


\newcommand{\ex}[2][\linewidth]{
    \begin{center}
        \begin{tcolorbox}[title=Example, colback=white, colframe=brown, width=#1]
            #2
        \end{tcolorbox}
\end{center}}


\newcommand{\q}[2][\linewidth]{
    \begin{center}
        \begin{tcolorbox}[title=Frage:, colback=white, colframe=purple, width=#1]
            #2
        \end{tcolorbox}
\end{center}}


\newcommand{\ff}[2][\linewidth]{
    \begin{center}
        \begin{tcolorbox}[title=Forschungsfrage:, colback=white, colframe=black, width=#1]
            #2
        \end{tcolorbox}
\end{center}}


\newcommand{\steps}[2][0.8\linewidth]{
    \begin{center}
        \begin{tcolorbox}[title=Steps:, colback=white, colframe=blue, width=#1]
            \begin{enumerate}
                #2
            \end{enumerate}
        \end{tcolorbox}
\end{center}}


\usepackage{xifthen}
\makeatother
\def\@lecture{}%
\newcommand{\lecture}[3]{
    \ifthenelse{\isempty{#3}}{%
        \def\@lecture{Lecture #1}%
    }{%
        \def\@lecture{Lecture #1: #3}%
    }%
    \subsection*{\@lecture}
    \marginpar{\small\textsf{\mbox{#2}}}
}
\makeatletter

\author{Daniel Renschler}


%% For linear algebra


\newcommand{\atrix}[2]{%
  \left[
  \begin{array}{#1}
  #2
  \end{array}
  \right]
}


\input{/Users/daniel/docs/LaTeX/template-related/sir-mars-template/letterfonts}



\begin{document}
\title{Calculus}
\author{Daniel Renschler}
\date{As of \today}
\maketitle


\section{Limits}
\subsection{general ?}
Limit as a approaches x.
The limit always gives the slope, so with the limit above we get the slope [m] as variable a approaches x.
\[\lim_{a \to x}\]
\subsection{How to solve}
A limit is solvable by pluggin in.
\subsection{limits in form of derivative}
\[ \lim_{h \rightarrow 0} \frac{f(x+h)-f(x)}{h} \]
This is how one can solve a limit, here in case of a derivative:
Just plug in as said above, and if that's not possible just do polynomial division instead or add the "reverse" the denominator (the bottom thing of a fraction) I'll show it in a sec.
\subsection{what one can do with limits:}
\begin{eqnarray*}
    \lim_{x \to a}\left[f(x)±g(x)\right] &=& \lim_{x \to a}f(x)±\lim_{x \to a}g(x)\\
    \lim_{x \to a}\left[f(x)*g(x)\right]&=&\lim_{x \to a}f(x)*\lim_{x \to a}g(x) \\
    \lim_{x \to a}\left[\frac{f(x)}{g(x)}\right] &=& \frac{\lim_{x \to a}f(x)}{\lim_{x \to a}g(x)}, \lim_{x \to a}g(x) \neq 0 \\
    \lim_{x \to a}\left[f(x)\right]^n &=& \left[\lim_{x \to a}f(x)\right]^n \to  \lim_{x \to a} \sqrt[n]{f(x)} \to \sqrt[n]{\lim_{x \to a}f(x)}    \\
\end{eqnarray*}



\section{Derivatives}
\subsection{in general}
\[\frac{d}{dx}[x^n] = nx^{n-1}\]

\subsection{product rule}
\[\frac{d}{dx}\left[f(x)*g(x)\right]=f'(x)*g(x)+f(x)*g'(x)\]

\subsection{quotient rule}
\[\frac{d}{dx} \left[\frac{f(x)}{g(x)}\right]=
\frac{g(x)*f'(x)-f(x)*g'(x)}{[g(x)]^2}\]

\subsection{derivatives with roots and how to solve them}
Ex: $\frac{d}{dx}(\sqrt{3x})$\\
Rewrite $\sqrt{3x}$ as $(3x)^\frac{1}{2}$ using the fact that
$\sqrt{y}=y^\frac{1}{2}$. Then rewrite $(3x)^\frac{1}{2}$ as
$3^\frac{1}{2}x^\frac{1}{2}$ using the rule $(cx)^n = c^n x^n$. Compare
$3^\frac{1}{2} x^\frac{1}{2}$ with the general form $ax^b$ to see that the
coefficient is $a = 3^\frac{1}{2}$ and the exponent is $b = \frac{1}{2}$. Plug 
$a = 3^\frac{1}{2}$ and $b=\frac{1}{2}$ into the formula
$\frac{d}{dx}(ax^b)=bax^{b-1}$ (the bottom thing of a fraction).
\[\frac{d}{dx}(\sqrt{3x})=\frac{d}{dx}(3^\frac{1}{2}x^\frac{1}{2})=(\frac{1}{2})(3^{\frac{1}{2}})x^{\frac{1}{2}-1}=\frac{3^{\frac{1}{2}}}{2}x^{-\frac{1}{2}}\]

\subsection{derivatives of trigonometric functions}
\[\lim_{h \rightarrow 0}\frac{\sin(h)}{h}=1, \lim_{h \rightarrow 0}\frac{1-\cos(h)}{h}=0\]
Find $f'(x)$ for $f(x)=\sin(x)$,\\
$f(x+h)=\sin(x+h)$\\
$f(x)=\sin(x)$


\begin{eqnarray*}
     f'(x) &= \lim_{h \rightarrow 0} \frac{f(x+h)-f(x)}{h}	\\ 
    \lim_{h \rightarrow 0}&\frac{\sin(x+h)-\sin(x)}{h}	\\
    \lim_{h \rightarrow 0}&\frac{\sin(x)\cos(h)+\cos(xj)\sin(h)}{h}	\\
    \lim_{h \rightarrow0}&\frac{\sin(x)\cos(h)-\sin(x)}{h}\frac{\cos(x)\sin(h)}{h}		\\
    \lim_{h \rightarrow0}&\frac{\sin(x)(\cos(h)-1)}{h}+\cos(x)\frac{\sin(h)}{h}		\\
    \lim_{h \rightarrow0}&-\sin(x)*\frac{1-\cos(h)}{h}+\cos(x)*\frac{\sin(h)}{h}=0+\cos(x)*1=\cos(x) \\
    =&\larger{\frac{d}{dx}\sin(x)=\cos(x)}
\end{eqnarray*}

\subsubsection{trig derivatives to remember}
\begin{eqnarray*}
    \frac{d}{dx}[\sin x]&=&\cos (x)\\
    \frac{d}{dx}[\cos x]&=&-\sin(x)\\
    \frac{d}{dx}[\tan x]&=&\sec^2 (x)\\
    \frac{d}{dx}[\sec x]&=&\sec(x)\tan(x)\\
    \frac{d}{dx}[-\csc x]&=&\csc(x)\cot(x)\\
    \frac{d}{dx}[\cot x]&=&\csc^2(x)\\
\end{eqnarray*}
\begin{align*}
    \frac{d}{dx}\left[\sin^{-1}\right]&= \frac{1}{\sqrt{1-x^2}} \text{where} |x|<1\\
    \frac{d}{dx}\left[\cos^{-1}\right]&= \frac{-1}{\sqrt{1-x^2}} \text{where} |x|<1 \\
    \frac{d}{dx}\left[\tan^{-1}\right]&=\frac{1}{1+x^2}  \\
    \frac{d}{dx}\left[\cot^{-1}\right]&= \frac{-1}{1+x^2} \\
    \frac{d}{dx}\left[\sec^{-1}\right]&=\frac{1}{|x|\sqrt{x_2-1}} \text{where} |x|>1\\
    \frac{d}{dx}\left[\csc^{-1}\right]&= \frac{-1}{|x|\sqrt{x^2-1}} \text{where} |x|>1 \\
    \frac{d}{dx}\left[e^{ax}\right]&= ae^{ax} \\
    \frac{d}{dx}\left[\ln(ax)\right]&= \frac{1}{x} \\
    \frac{d}{dx}\left[\log_bx\right]&= \frac{1}{x \ln b} \\
    \frac{d}{dx}\left[b^x\right]&= b^x \ln b \\
    \frac{d}{dx}\left[\sin h\right]&= \cos h \\
    \frac{d}{dx}\left[\cos h\right]&= \sin h \\
    \frac{d}{dx}\left[\tan h\right]&= \sec h^2= \frac{1}{\cos h^{2}}
\end{align*}

\nt{Things may be missing}



\end{document}
