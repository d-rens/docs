\documentclass{article}

\usepackage[a4paper,landscape,margin=1cm]{geometry}
\usepackage{booktabs}
\usepackage{longtable}

\fontsize{9pt}{11pt}\selectfont % sets the font size to 10pt with 12pt line spacing


\begin{document}


\begin{longtable}{p{3cm}|p{4cm}|p{6cm}|p{5cm}|p{5cm}}
\toprule
Tag & Lesen & Physik & Mathe & Pyton \\
\midrule
\endhead % everything after this line will appear at the top of each subsequent page
5. Mittwoch   & Modern india half 67P& rev: grundlegende begriffe, columbsches gesetz, definition und Formel \& Beispiel anwendungen & Review derivatives with the chain, product, and exponent rules. Make sure to understand the concepts behind these rules and be able to apply them to various functions. & Introduce Python and the basic syntax, including variables, data types, arithmetic operations, and print statements. \\ 
\vspace{5pt}\\
6. Donnerstag & Modern india half 67P& Elektrisches Feld, definition und bedeutung, elektrisches feld von Punktladungen, feldlinien und ihre eigenschaften & same as day before & Introduce control structures like if-else statements, while loops, and for loops. Show how to use them to control the flow of a program. \\
\vspace{5pt}\\
7. Freitag    & Fith of Nietzsche 38P& Elektrischs Feld eines Plattenkondensators, Plattenkondensator: definition und eigenschaften, die kapazitp\"at eines Kondensators & Review general derivatives. This includes the power rule, quotient rule, and trigonometric functions. Again, make sure to understand the concepts and be able to apply them. & Introduce functions and how to define and call them in Python. Show how to use modules to organize code into reusable pieces. \\
\vspace{5pt}\\
8. Samstag    & Fith of Nietzsche 38P& Elektrisches potential: def und bedeutung, potentialdifferenz und elektrische spannung, beispiele f\"ur andwendung davon & same as day before & Introduce file handling in Python, including opening, reading, and writing to files. Also, explain how to handle errors using try-except blocks. \\
\vspace{5pt}\\
9. Sonntag    & Fith of Nietzsche 38P& Zusammenhang zwischen Kraftvektor und elektrischem feld f\"ur $\pm$ Ladungen, elektrische Feldst\"arke: def und formel& Review how to get a tangent function. This includes understanding what a tangent function is, how to find its equation, and how to use it to find tangent lines to curves. & Review the concepts learned so far and provide exercises or small projects for practice. \\
\vspace{5pt}\\
10. Montag     & Fith of Nietzsche 38P& Stromst\"arke def und bedeutung, elek. Widerstand: def und eigenschaften, ohmsches gesetz: def und formel & same as day before& Introduce the concept of object-oriented programming in Python. Explain classes, objects, and inheritance. \\
\vspace{5pt}\\
11. Dienstag   & Fith of Nietzsche 38P& Beziehung zwischen elektrischem Strom, Spannung und Widerstand, parralel- und reihenschaltung von widerst\"anden & Review all of the above topics. Practice applying the rules and concepts to a variety of functions. & Introduce popular Python libraries and frameworks for different domains like data science (NumPy, Pandas, Scikit-learn), web development (Django, Flask), and machine learning (TensorFlow, Keras). \\
\vspace{5pt}\\
12. Mittwoch   & Third of El túnel 30P& Elektrische energie, def. und bed., elek. Arbeit udn Leistung, Energieerhaltungssatz in elektrischen Schaltungen& same as day before & Introduce data structures like lists, tuples, and dictionaries. Show how to create, modify, and access their elements. \\
\vspace{5pt}\\
13. Donnerstag & Third of El túnel 30P& Kondensatoren: def. und eigens., arten von kondensatoren, kondensatoren in elektrischen Schaltungen & Learn something new! Pick a topic that interests you and learn about it. This could be something related to calculus, or a different branch of math altogether. & Introduce regular expressions and how to use them in Python to search for patterns in text. \\
\vspace{5pt}\\
14. Freitag    & Third of El túnel 30P& Zusammenfassen und wiederholen aller konzepte, aufgaben suchen und l\"osen.& same as day before & Assign a small project to apply the concepts learned throughout the week. This could be a simple game, a web application, or a data analysis project. \\
\vspace{5pt}\\
15. Samstag    & Reading what isn't done or start sth new & weiter wiederholen
und l\"osen & Practice! Spend these last two days practicing everything you've
learned. Work through practice problems, review your notes, and make sure you
feel confident with the material. & review \\
\vspace{5pt}\\
16. Sonntag    & Reading what isn't done or start sth new & weiter wiederholen
und l\"osen & same as day before & review \\
\bottomrule
\end{longtable}
%
\end{document}
