\documentclass{article}

\usepackage[margin=3cm,a4paper]{geometry}
\usepackage{amsmath,amssymb}
\usepackage{booktabs}
\usepackage{colortbl}
\usepackage{xcolor}
\usepackage{tcolorbox}

\setlength{\parindent}{0pt}

\title{Ferien Mathe}
\author{d.}
\date{5. bis 16. april}

\begin{document}
\maketitle
\tableofcontents
\clearpage

\section{5., Mittwoch}

Bin zu weit um jetzt die grundlagen zu wiederholen, aber hier mal paar sachen
f\"ur Ableitungen und beispiel rechnungen die m\"oglichst schwer sind um auch
m\"oglichst viel zu lernen.

Potentzregel:
\begin{equation*}
    \frac{d}{dx}\left[ax^b\right]=b\cdot ax^{b-1}
\end{equation*}
Beispielrechnung: $f(x) = x^{\sin(x)} + \sin(x)^x$.
\begin{align*}
f(x) &= x^{\sin(x)} + \sin(x)^x \\
f'(x) &= \frac{d}{dx} \left(x^{\sin(x)}\right) + \frac{d}{dx} \left(\sin(x)^x\right) \\
&= \frac{d}{dx} \left(e^{\sin(x) \ln(x)}\right) + \frac{d}{dx} \left(e^{x \ln(\sin(x))}\right) \\
&= e^{\sin(x) \ln(x)} \cdot \left(\cos(x) \ln(x) + \frac{\sin(x)}{x}\right) +
e^{x \ln(\sin(x))} \cdot \left(\ln(\sin(x)) + \frac{x \cos(x)}{\sin(x)}\right)\\
\end{align*}
Die war eventuell ein bisschen schwerer, da sie Kettenregel, Produktregel und
die Ableitung der Logarithmusfunktion beinhaltet, deswegen hier noch eine
erkl\"arung der Regeln.

Zunächst muss man die Potenzregel für Ableitungen kennen. Die Potenzregel besagt, dass wenn man eine Funktion der Form $f(x) = x^n$ ableitet, erhält man $f'(x) = nx^{n-1}$. Dabei ist $n$ eine Konstante und $x$ eine Variable.

Für diese spezifische Aufgabe müssen wir nun die Funktion $f(x) = (x^3 - 2x^2 + 1)^{10}$ ableiten.

Um dies zu tun, verwenden wir die Kettenregel, die besagt, dass wenn wir eine Funktion der Form $g(h(x))$ ableiten müssen, dann ist die Ableitung gegeben durch $g'(h(x)) \cdot h'(x)$.

In diesem Fall ist $g(x) = x^{10}$ und $h(x) = x^3 - 2x^2 + 1$. Wir müssen also die Ableitungen von $g$ und $h$ berechnen, sowie $h'$ in die Kettenregel einsetzen.

Die Ableitung von $g(x) = x^{10}$ ist $g'(x) = 10x^9$.

Die Ableitung von $h(x) = x^3 - 2x^2 + 1$ ist $h'(x) = 3x^2 - 4x$.

Nun können wir die Kettenregel anwenden:

\begin{align*}
f'(x) &= g'(h(x)) \cdot h'(x) \\
&= 10(x^3 - 2x^2 + 1)^9 \cdot (3x^2 - 4x) \\
&= 10x(3x^2 - 4x)(x^3 - 2x^2 + 1)^9 \\
&= 30x^3(x^3 - 2x^2 + 1)^9 - 40x^2(x^3 - 2x^2 + 1)^9 \\
&= 30x^3(x^3 - 2x^2 + 1)^9 - 40x^5(x^3 - 2x^2 + 1)^8
\end{align*}

Das ist die Ableitung von $f(x) = (x^3 - 2x^2 + 1)^{10}$.





\clearpage
\section{Unrelated:}
\subsection{partial stuff}
\subsubsection{1}
Problem: Find the partial derivatives of the function $f(x,y,z) = xy^2e^{xz}$ at the point $(1,1,0)$.

Solution: 
We begin by finding the first-order partial derivatives of $f$ with respect to each variable:
\begin{align*}
\frac{\partial f}{\partial x} &= y^2 e^{xz} + xz y^2 e^{xz} \\
\frac{\partial f}{\partial y} &= 2xy e^{xz} \\
\frac{\partial f}{\partial z} &= xy^2 x e^{xz} \\
\end{align*}
Next, we evaluate each of these partial derivatives at the point $(1,1,0)$:
\begin{align*}
\left.\frac{\partial f}{\partial x}\right|_{(1,1,0)} &= 1 \\
\left.\frac{\partial f}{\partial y}\right|_{(1,1,0)} &= 2e^0 = 2 \\
\left.\frac{\partial f}{\partial z}\right|_{(1,1,0)} &= 0 \\
\end{align*}
Finally, we can write the gradient of $f$ at $(1,1,0)$ as follows, using the nabla operator:
\[\nabla f(1,1,0) = \left(\left.\frac{\partial f}{\partial x}\right|_{(1,1,0)}, \left.\frac{\partial f}{\partial y}\right|_{(1,1,0)}, \left.\frac{\partial f}{\partial z}\right|_{(1,1,0)}\right) = (1,2,0)\]

\end{document}
