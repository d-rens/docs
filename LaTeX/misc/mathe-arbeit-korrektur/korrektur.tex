\documentclass{article}
\usepackage[a4paper,margin=3cm]{geometry}
\usepackage{amsmath}
\usepackage{multicol}

\setlength{\parindent}{0pt}


\author{Aaron, Daniel \& Vincent}
\title{Mathe Arbeit J1-2 1 Verbesserung}


\begin{document}
\maketitle

\textbf{Besprechung der Klausur (Gruppenarbeit)}

\begin{enumerate}
  \item Besprechen Sie alle Aufgaben. Klären und korrigieren Sie dabei die
    Fehler jeder Schülerin bzw. jeden Schülers der Gruppe.
  \item Einigen Sie sich auf eine Musterlösung der Klausur. Achten Sie dabei
    auf Vollständigkeit, eine saubere Darstellung und korrekte Anwendung der
    Fachsprache. Die Musterlösung muss abgeben werden.
  \item Jeder Schüler soll in den kommenden Wochen an einer seiner
    Schwierigkeiten arbeiten. Besprechen Sie in der Gruppe, an welcher
    Schwierigkeit jede Schülerinbzw. jeder Schüler arbeiten wird.
\end{enumerate}

Reflexion der Klausur (Einzelarbeit)Beantworten Sie ehrlich folgende Fragen.

\begin{itemize}
  \item Ich bin mit dem Ergebnis der Klausur zufrieden. 
    \subitem \textit{Nein, Joa, Ja}

  \item Was haben Sie in der in der Klausur leicht lösen können? Was hat Ihnen
    Schwierigkeiten bereitet?
    \subitem \textit{Ableitungen waren eine Schwierigkeit, auch extremwerte und punkte}

  \item Wie haben Sie sich zu Hause auf die Klausur vorbereitet?
    \subitem Die Aufschriebe im Heft noch einmal durchgelesen. \textit{Ja, Joa, Nein}
    \subitem Grundbegriffe wiederholt. \textit{Nein, nicht so, Ja}
    \subitem Übungsaufgaben gelöst. \textit{Ja, ne, Nein}
    \subitem Andersweitig ge\"ubt Wie? \textit{Ja, Ne, Ja}

  \item Wie lange haben Sie für die Klausur gelernt? \textit{10h, 0.5h, 1h}

  \item Haben Sie aktiv im Unterricht mitgearbeitet? (ständig / häufig / wenig / nie) \textit{Nein, Ja, Nein}
    \subitem Im Unterrichtsgespräch? \textit{N\"o, wenig, ne}
    \subitem In den Arbeitsphasen? \textit{Joa, Ja schon, ein wenig}

  \item Was hätten Sie im Vorfeld der Klausur anders machen können, um Ihre Note zu verbessern?
    \subitem \textit{Weis ich nicht, mehr \"ubungsaufgaben, sich \"uberzeugen das es die Zeit wert ist}

  \item An welcher Schwierigkeit werden Sie in den kommenden Wochen arbeiten?
    \subitem \textit{Alles anders, Hausaufgaben machen, weniger Linux mehr rechnen}
\end{itemize}


\begin{multicols}{2}

\section{}
\subsection{}
\begin{align*}
  f(x)&=2x^3-\sqrt{5}\cdot x^2+9\\
  f'(x)&=6x^2-\sqrt{5}\cdot 2x-1\\
\end{align*}

\subsection{}
\begin{align*}
  u(t)&=-3\sqrt{t}+\frac{1}{t}\\
  u'(t)&=-\frac{3}{2}t^{-\frac{1}{2}}-t^{-2}\\
\end{align*}

\subsection{}
\begin{align*}
  g(x)&=4x^2\cdot e^{3x-2}\\
  g'(x)&=8xe^{3x-2}+12x^2\cdot e^{3x-2}\\
\end{align*}

\subsection{}

\begin{align*}
  h(x)&=cos(-x+6)\cdot sin(x)\\
  h'(x)&=\sin(-x+6)\cdot \sin(x) + \cos(-x +6) \cdot \cos(x)\\
\end{align*}

\section{}
\subsection{}
Wendepunkte: 1(0,5;0,5), 2(-0,5;0,5)

\section{}
\subsection{}
Falsch, grafisch kann man pruefen, dass die Ableitung $g'$ bei $x_0=1$ ungefaer
1 ist, in dem intervall $[3;6]$ die mittlere aenderungsrate aber nur $~
\frac{1}{2}$, also um die haelfte weniger.

\subsection{}
Wahr, da es in deiesem Bereich keine positive Steigung gibt.
\end{multicols}

\section*{Teil 2, Aufgabe 1}
\begin{align*}
  f'(x)&=\lim_{h\to 0}\frac{f(x+h)-f(x)}{h}\\
       &=\lim_{h\to 0}\frac{(-(x+h)^2+4(x+h)+3)-(-x^2+4x+3)}{h}\\
       &=\lim_{h\to 0}\frac{-x^2-2xh-h^2+4x+4h+3+x^2-4x-3}{h}\\
       &=\lim_{h\to 0}\frac{-2xh-h^2+4h}{h}\\
       &=\lim_{h\to 0}(-2x-h+4)\\
       &=-2x+4\\
\end{align*}
Daher ist die Ableitungsfunktion von $f(x)$ gleich $f'(x)=-2x+4$.

\section*{Aufgabe 2}

Gegeben ist die Funktion $h(x) = 5x^2 \cdot e^{\frac{2}{3} x^3}$ für $-2 \leq x
\leq 2$. Wir sollen zeigen, dass $h'(x) = 10x \cdot (1+x^3) \cdot
e^{\frac{2}{3} x^3}$ eine Gleichung der ersten Ableitung von $h$ ist.

Zunächst berechnen wir die Ableitung $h'(x)$ von $h(x)$ nach der Produktregel
und der Kettenregel:

\begin{align*}
h'(x) &= \frac{d}{dx}(5x^2) \cdot e^{\frac{2}{3} x^3} + 5x^2 \cdot \frac{d}{dx}(e^{\frac{2}{3} x^3})\\
      &= 10x \cdot e^{\frac{2}{3} x^3} + 5x^2 \cdot \frac{2}{3} x^2 \cdot e^{\frac{2}{3} x^3}\\
      &= 10x \cdot e^{\frac{2}{3} x^3} + 10x^3 \cdot e^{\frac{2}{3} x^3}
\end{align*}

Dann faktorisieren wir den Term $h'(x)$ mit $10x$:
\begin{align*}
  h'(x) &= 10x \cdot e^{\frac{2}{3} x^3} + 10x^3 \cdot e^{\frac{2}{3} x^3}\\
        &= 10x \cdot (1 + x^3) \cdot e^{\frac{2}{3} x^3}
\end{align*}
Daher ist $h'(x) = 10x \cdot (1+x^3) \cdot e^{\frac{2}{3} x^3}$ eine Gleichung
der ersten Ableitung von $h(x)$, wie gefordert.

\subsection*{2.2\&3}
Da die Funktion $h(x) = 5x^2 \cdot e^{\frac{2}{3} x^3}$ auf dem Intervall $-2
\leq x \leq 2$ stetig ist, existieren nach dem Satz von Weierstraß globale
Extremwerte auf diesem Intervall.

Um die globalen Extremwerte zu ermitteln, suchen wir zuerst die kritischen
Punkte von $h(x)$, d.h. die Punkte, an denen die Ableitung von $h(x)$ gleich
Null ist oder nicht existiert. Wir berechnen zuerst die erste Ableitung von
$h(x)$:

$h'(x) = 10x \cdot (1+x^3) \cdot e^{\frac{2}{3} x^3}$

$h'(x)$ ist genau dann gleich Null, wenn $x=0$ oder $x=-1$ oder $x=1$. Um zu
zeigen, dass $h'(x)$ für alle $x \in [-2,0)$ negativ und für alle $x \in (0,2]$
positiv ist, nutzen wir die Vorzeichenregel für Ableitungen:

Für $x \in [-2,0)$ gilt: $h'(x) < 0$, da $10x < 0$ und $1+x^3 > 0$ für alle $x
\in [-2,0)$. Außerdem ist $e^{\frac{2}{3} x^3} > 0$ für alle $x$, daher ist
$h'(x) < 0$ für alle $x \in [-2,0)$.

Für $x \in (0,2]$ gilt: $h'(x) > 0$, da $10x > 0$ und $1+x^3 > 0$ für alle $x
\in (0,2]$. Außerdem ist $e^{\frac{2}{3} x^3} > 0$ für alle $x$, daher ist
$h'(x) > 0$ für alle $x \in (0,2]$.

Somit hat $h(x)$ bei $x=0$ ein lokales Maximum und bei $x=-1$ und $x=1$ jeweils
ein lokales Minimum auf dem Intervall $[-2,2]$. Da $h(x)$ für $x \rightarrow
-\infty$ und $x \rightarrow \infty$ gegen $0$ strebt, sind das globale Maximum
von $h(x)$ bei $x=0$ und das globale Minimum von $h(x)$ bei $x=-1$ und $x=1$.

Daher sind die globalen Extremwerte von $h(x)$ auf dem Intervall $[-2,2]$:
\begin{itemize}
  \item globales Maximum: $h(0) = 0$
  \item globales Minimum: $h(-1) = \frac{5}{e^{\frac{2}{3}}} \approx 0.7648$ und $h(1) = \frac{5}{e^{\frac{2}{3}}} \approx 0.7648$
\end{itemize}


\section*{Aufgabe 3}
\textit{Aufgabe 3 ist nicht siginifikant und wurde daher ausgelassen.}


\end{document}
