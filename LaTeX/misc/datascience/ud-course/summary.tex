\documentclass{article}
\usepackage[a4paper, right=3cm, left=3cm, bottom=2.5cm, top=3cm]{geometry}
\usepackage{amsmath}
\usepackage{booktabs}
\usepackage{tikz}
\usepackage{pgfplots}
\pgfplotsset{compat=1.18}
\usepackage{pstricks}
\usepackage{color}
\usepackage{multicol}
\setlength{\parindent}{0pt}


\begin{document}
\tableofcontents

\section{Combinatorics}
yeah, somebody should add it here :/.

\section{Probability}
\subsection{Introduction}
Probability is a fundamental concept in statistics and machine learning.
Understanding the basics of probability is crucial for mastering data science,
as it enables the extraction of important insights from data. Bayesian
Inference is a key component heavily used in many fields of mathematics to
succinctly express complicated statements. Through Bayesian Notation,
relationships between elements, sets, and events can be conveyed. This
understanding can aid in interpreting the mathematical intuition behind
sophisticated data analytics methods.

Distributions are the main way to classify sets of data. If a dataset complies
with certain characteristics, the likelihood of its values can be attributed to
a specific distribution. Many of these distributions have elegant relationships
between certain outcomes and their probabilities of occurring, making it
extremely convenient and useful to know key features of our data.

Overall, understanding the basics of probability and distributions is crucial
for anyone looking to dive into the world of statistics and machine learning,
as it enables the extraction of meaningful insights from data.

\subsubsection{What is probability?}
Probability is the measure of the likelihood of an event occurring, and it can
range from 0 to 1. The probability formula states that the probability of event
X occurring equals the number of preferred outcomes over the number of outcomes
in the sample space. Preferred outcomes are the outcomes we want to occur or
the outcomes we are interested in, while the sample space refers to all
possible outcomes that can occur.

\begin{align*}
  P(X)=\frac{\text{preferred outcomes}}{\text{sample space}}
\end{align*}

If two events are independent, the probability of them occurring simultaneously
equals the product of them occurring on their own. This concept is useful in
many areas, including statistics and machine learning, where probabilities are
used to make predictions and inferences.

\subsubsection{expected values}
\textbf{Trail} - Ovserving an event occur and recording the outcome.\\
\textbf{Experiment} - A collectoin of one or multiple trails.\\
\textbf{Experimental Probability} - The probability we assign an event, based on a experiment we conduct.\\
\textbf{Expected Value} - The specific outcome we expect to occur when we run an experiment.\\

\textbf{Example:} Trail\\
Flipping a coin and recording the outcome.

\textbf{Example:} Experiment\\
Flipping a coin 20 times and recording the 20 individual outcomes.

In this instance, the \textbf{experimental probability} for getting heads would
equal the number of heads we record over the course of the 20 outcomes, over 20
(the total number of trails).

The \textbf{expected value} can be numerical, Boolearn, categorical or other
depending on the type of the events we are interested in. For instance, the
expected value of the trail would be the more likely of the two outcomes,
whereas the expected value of the experiment will be the number of time we
expect get either heads or tails after 20 trails.

Expected value for \textbf{categorical} variables.
\[E(X)=n \cdot p\]

Expected value for \textbf{numeric} variables.
\[E(X)=\displaystyle\sum_{i=1}^{n}x_1\cdot p_i\]

\subsubsection{Probability Frequency Distribution}

\textbf{What is a probability frequency distribution?:}\\
A collection of the probabilities for each possible outcome of an 
event. 

\vspace{.5cm}
\textbf{Why do we need frequency distributions?:}\\
We need the probability frequency distribution to try and predict 
future events when the expected value is unattainable. 

\vspace{.5cm}
\textbf{What is a frequency?:}\\
Frequency is the number of times a given value or outcome 
appears in the sample space. 

\vspace{.5cm}
\textbf{What is a frequency distribution table?:}\\
The frequency distribution table is a table matching each distinct 
outcome in the sample space to its associated frequency.

\vspace{.5cm}
\textbf{How do we obtain the probability frequency distribution from the
frequency distribution table?:}\\
By dividing every frequency by the size of the sample space. 
(Think about the “favoured over all” formula.)

\begin{center}
\begin{tabular}{ccc}
\toprule
Sum & Frequency & Probability \\
\midrule
\vspace{4pt}
2   & 1         & $\frac{1}{36}$ \\
\vspace{4pt}
3   & 2         & $\frac{1}{18}$ \\
\vspace{4pt}
4   & 3         & $\frac{1}{12}$ \\
\vspace{4pt}
5   & 4         & $\frac{1}{9}$ \\
\vspace{4pt}
6   & 5         & $\frac{5}{36}$ \\
\vspace{4pt}
7   & 6         & $\frac{1}{6}$ \\
\vspace{4pt}
8   & 5         & $\frac{5}{36}$ \\
\vspace{4pt}
9   & 4         & $\frac{1}{9}$ \\
\vspace{4pt}
10  & 3         & $\frac{1}{12}$ \\
\vspace{4pt}
11  & 2         & $\frac{1}{18}$ \\
\vspace{4pt}
12  & 1         & $\frac{1}{36}$ \\
\bottomrule
\end{tabular}
\end{center}

\subsubsection{Complements}
The complement of an ecent is \textbf{everything} an event is \textbf{not}. We
denote the complement of an event with an apostrophe.
\[A\prime = \text{not} A\]
Where $A\prime$ is an complement, not the opposite and $A$ The original event.

\textbf{characteristics of complements:}
\begin{itemize}
  \item Can never occur simultaneously.
  \item Add up to the sample space. ($A + A\prime =$ sample space)
  \item Their probabilities add up to 1. $(P(A)+P(A\prime)=1)$
  \item The complement of a complement is the original event. ($(A\prime)\prime = A$)
\end{itemize}

\textbf{Example:}
\begin{itemize}
  \item Assume event A represents drawing a spade, so P(A) = 0.25.
  \item Then, $A\prime$ represents \textbf{not} drawing a spade, so drawing a
    club, a diamond or a heart. $P(A\prime)=1-P(A),\text{ so} P(A\prime)=0.75$.
\end{itemize}



\subsection{Bayesian thingies}
A \textbf{set} is a collection of elements, which hold vertain values.
Additionally, every event has a set of outcomes that satisfy it.
The \textbf{null-set} (or empty set), denoted $\emptyset$ , is an set which
contain no values.

An element is denoted in lower-case, e.g. $x$.\\
A set is written with upper-case, e.g. $A$.\\
To make sense out of it, it can be used the following ways:\\

\begin{table}[htb]
    \centering
    \begin{tabular}{c|c|c}
        \toprule
        \textbf{Notation} & \textbf{Interpretation} & \textbf{Example} \\
        \midrule
        $x \in A$ & Element $x$ is part of set $A$ & $2 \in$ All even numbers \\
        $A \ni x$ & Set $A$ contains element $x$ & All even numbers $\ni 2$ \\
        $x \notin A$ & Element $x$ is not part of set $A$ & $1 \notin$ All even numbers \\
        $\forall x:$ & For all $x$ such that... & $\forall x: x \in$ All even numbers \\
        $A \subseteq B$ & Set $A$ is a subset of set $B$ & Even numbers $\subseteq$ Integers \\
        \bottomrule
    \end{tabular}
    \caption{Notations and interpretations for set theory}
\end{table}

Remember every set has at least two subsets:\\
\begin{itemize}
  \item $A \subseteq A$
  \item $\emptyset \subseteq A$
\end{itemize}

There are three different types of multible events for now, one can imagine them as circles which are:
\begin{enumerate}
  \item Not touching at all.
  \item Intersect (partially overlap).
  \item One completely overlaps another.
\end{enumerate}

\paragraph{Intersection}
We denote the intersection of two sets with the "intersect" sign, which
resembles an upside down capital letter U: $A \cap B$.

\begin{center}
%LaTeX with PSTricks extensions
%%Creator: Inkscape 1.2.2 (b0a8486541, 2022-12-01)
%%Please note this file requires PSTricks extensions
\psset{xunit=.5pt,yunit=.5pt,runit=.5pt}
\begin{pspicture}(359.05511811,151.18110236)
{
\newrgbcolor{curcolor}{0 0 0}
\pscustom[linestyle=none,fillstyle=solid,fillcolor=curcolor]
{
\newpath
\moveto(131.57115546,74.2413234)
\curveto(131.57115546,45.01115572)(106.37661539,21.31542296)(75.29757522,21.31542296)
\curveto(44.21853505,21.31542296)(19.02399499,45.01115572)(19.02399499,74.2413234)
\curveto(19.02399499,103.47149107)(44.21853505,127.16722383)(75.29757522,127.16722383)
\curveto(106.37661539,127.16722383)(131.57115546,103.47149107)(131.57115546,74.2413234)
\closepath
}
}
{
\newrgbcolor{curcolor}{1 0 0}
\pscustom[linestyle=none,fillstyle=solid,fillcolor=curcolor]
{
\newpath
\moveto(183.62061508,73.37218517)
\curveto(183.62061508,53.80521233)(166.86277446,37.94304675)(146.19093046,37.94304675)
\curveto(125.51908647,37.94304675)(108.76124585,53.80521233)(108.76124585,73.37218517)
\curveto(108.76124585,92.93915802)(125.51908647,108.8013236)(146.19093046,108.8013236)
\curveto(166.86277446,108.8013236)(183.62061508,92.93915802)(183.62061508,73.37218517)
\closepath
}
}
{
\newrgbcolor{curcolor}{0 1 1}
\pscustom[linestyle=none,fillstyle=solid,fillcolor=curcolor]
{
\newpath
\moveto(124.31596724,97.42277291)
\curveto(122.92175622,96.22090583)(121.09824,94.52885669)(118.25445165,91.44190866)
\curveto(117.64376315,90.7790022)(117.03298772,90.1043074)(116.37901228,89.37723591)
\curveto(115.55914205,88.46573102)(113.86667717,86.56731591)(112.55690457,85.15545071)
\curveto(112.03481575,84.59212724)(111.58347213,83.97313134)(111.0401915,83.16251717)
\curveto(110.96119937,83.06723528)(110.82952063,82.9086463)(110.73352063,82.73136378)
\curveto(110.68344189,82.63884094)(110.6401663,82.53664252)(110.61363402,82.42484409)
\curveto(110.58604346,82.30820787)(110.5777285,82.18590236)(110.59473638,82.0592126)
\curveto(110.60834268,81.95803465)(110.62610646,81.85341732)(110.6841222,81.76939843)
\curveto(110.71322457,81.72740787)(110.75154898,81.69154016)(110.79675213,81.66776693)
\curveto(110.84195528,81.64395591)(110.89365921,81.63261732)(110.94460724,81.6360189)
\curveto(110.99555528,81.63946205)(111.04529386,81.6575622)(111.08686866,81.68738268)
\curveto(111.12844346,81.71686299)(111.16166551,81.75771969)(111.18479622,81.8032252)
\curveto(111.23101984,81.89427402)(111.23434583,82.00028976)(111.23434583,82.1024126)
\lineto(111.23434583,82.44120945)
\curveto(111.23303578,82.8682235)(110.59316643,82.8682235)(110.59185638,82.44120945)
\lineto(110.59185638,82.1024126)
\lineto(111.23142425,82.14565039)
\curveto(111.22575496,82.18839685)(111.2279622,82.23099213)(111.23860535,82.27634646)
\curveto(111.24918803,82.32132283)(111.26846362,82.37045669)(111.29839748,82.42544882)
\curveto(111.36034394,82.53985512)(111.44976756,82.64980157)(111.55642583,82.78129134)
\curveto(112.11068976,83.6059011)(112.54007811,84.19222299)(113.02790173,84.71857134)
\curveto(114.34204724,86.13514961)(116.04071055,88.0404926)(116.85657071,88.9475452)
\curveto(117.50978646,89.67377008)(118.11864945,90.34634079)(118.72686236,91.00656)
\curveto(121.55522646,94.07676472)(123.36113008,95.75148094)(124.73533606,96.93610583)
\curveto(125.05809535,97.21589281)(124.64013865,97.70074247)(124.31583874,97.42274268)
\closepath
}
}
{
\newrgbcolor{curcolor}{0 1 1}
\pscustom[linestyle=none,fillstyle=solid,fillcolor=curcolor]
{
\newpath
\moveto(126.76293165,94.43227087)
\curveto(126.71357102,94.32345827)(126.67124031,94.17972283)(126.6665537,93.98765102)
\curveto(126.66343937,93.86016756)(126.67675843,93.73094551)(126.70056945,93.58699087)
\curveto(126.71908913,93.47462551)(126.74482772,93.3515263)(126.7633474,93.23919874)
\curveto(126.76750488,93.21463181)(126.77128441,93.19055622)(126.77468598,93.16663181)
\curveto(126.7841348,93.09829795)(126.79131591,93.02610898)(126.78980409,92.95796409)
\curveto(126.78997795,92.91956409)(126.78564661,92.89590425)(126.77808756,92.87511685)
\curveto(126.77090646,92.85584126)(126.75956787,92.83448693)(126.74029228,92.81022236)
\curveto(126.7217726,92.78716724)(126.6961474,92.7611263)(126.65990173,92.73236409)
\curveto(126.62550803,92.7051515)(126.58397102,92.67703181)(126.53343874,92.64728693)
\curveto(126.43384819,92.58870425)(126.32692535,92.53881449)(126.1897663,92.47342866)
\curveto(126.07509543,92.41877669)(125.93083087,92.34938457)(125.80947024,92.26691528)
\lineto(126.16569071,92.27014299)
\curveto(126.18836787,92.25540283)(126.2113474,92.23537134)(126.23179465,92.20933039)
\curveto(126.25220409,92.18325165)(126.26694425,92.15475402)(126.27673323,92.12587843)
\curveto(126.28618205,92.09790992)(126.28996157,92.07179339)(126.29071748,92.05051465)
\curveto(126.29149984,92.02972724)(126.28959496,92.01230362)(126.28742173,91.99983118)
\curveto(126.28326425,91.97639811)(126.27683906,91.96203591)(126.2757052,91.95973039)
\curveto(126.27421228,91.95644598)(126.27445417,91.9576063)(126.27830551,91.96388787)
\curveto(126.28624252,91.97636031)(126.30022677,91.99563591)(126.32112756,92.02058079)
\curveto(126.34115906,92.04439181)(126.36338268,92.06922331)(126.38613543,92.09363906)
\curveto(126.40805669,92.11745008)(126.42907087,92.13888)(126.4480063,92.15902488)
\curveto(126.45254173,92.16393827)(126.45745512,92.16922961)(126.46199055,92.17414299)
\curveto(126.46652598,92.17905638)(126.47219528,92.18510362)(126.47786457,92.19190677)
\curveto(126.49184882,92.20891465)(126.49714016,92.21609575)(126.50280945,92.22441071)
\curveto(126.50772283,92.23159181)(126.51792756,92.24708787)(126.52813228,92.26806425)
\curveto(126.53342362,92.27864693)(126.54098268,92.29603276)(126.54778583,92.31848315)
\curveto(126.55116472,92.32982173)(126.55496693,92.34493984)(126.55723465,92.36221228)
\curveto(126.55979717,92.37959811)(126.56177008,92.40197291)(126.56019024,92.42808945)
\curveto(126.55865575,92.45454614)(126.55338709,92.48667213)(126.54053669,92.52163276)
\curveto(126.52730835,92.55716031)(126.50652094,92.5932548)(126.47738079,92.62564535)
\curveto(126.44827843,92.65777134)(126.4148674,92.68154457)(126.38175874,92.69794772)
\curveto(126.3492548,92.71382173)(126.31894299,92.72213669)(126.2940737,92.72667213)
\curveto(126.24679181,92.73498709)(126.20835402,92.73082961)(126.18805795,92.7274885)
\curveto(126.16575874,92.72397732)(126.14818394,92.71879559)(126.13714772,92.71501606)
\curveto(126.11560441,92.70783496)(126.09908787,92.69989795)(126.09118866,92.69611843)
\curveto(126.0821178,92.69158299)(126.07418079,92.68704756)(126.06888945,92.68402394)
\curveto(126.05830677,92.67797669)(126.04885795,92.67192945)(126.0420548,92.66701606)
\curveto(126.03449575,92.66210268)(126.02731465,92.65681134)(126.01975559,92.65152)
\curveto(126.00501543,92.64093732)(125.98762961,92.62846488)(125.96725795,92.61342236)
\curveto(125.9468485,92.59830425)(125.92292409,92.58054047)(125.89480441,92.55941291)
\curveto(125.45408504,92.23883339)(125.0598463,91.86578646)(124.68553323,91.47109795)
\curveto(124.32009071,91.08575244)(123.97165606,90.67740094)(123.6268422,90.29204409)
\curveto(123.19035969,89.82564661)(122.6778822,89.4237052)(122.15531339,89.00209512)
\curveto(121.61989039,88.5701178)(121.09091906,88.13029795)(120.64019528,87.59112945)
\curveto(117.57571276,84.3214526)(115.05986646,81.48183685)(113.17902236,78.10407685)
\curveto(112.43403213,76.74239244)(112.04261291,75.69385323)(111.80938961,74.90904189)
\curveto(111.68924259,74.4994672)(112.30225596,74.3173125)(112.42526362,74.72603717)
\curveto(112.64598803,75.46876724)(113.0201726,76.47514205)(113.74151433,77.79362268)
\curveto(115.57918488,81.09381921)(118.04571969,83.88341669)(121.12108346,87.1654526)
\curveto(121.53911811,87.66471685)(122.02292031,88.06976882)(122.55874394,88.50207118)
\curveto(123.0678652,88.91283024)(123.62177008,89.34635339)(124.10082142,89.85836598)
\curveto(124.46456315,90.2647748)(124.79435339,90.65217638)(125.15171906,91.02899528)
\curveto(125.51103874,91.4078778)(125.87527937,91.75075276)(126.27658583,92.04269102)
\curveto(126.30795591,92.06612409)(126.33048189,92.08301858)(126.34892598,92.09658709)
\curveto(126.36782362,92.11057134)(126.38105197,92.12002016)(126.39016063,92.12644535)
\curveto(126.37239685,92.11813039)(126.35765669,92.11094929)(126.33743622,92.10414614)
\curveto(126.3272315,92.10074079)(126.30984567,92.09583118)(126.28830236,92.09242961)
\curveto(126.26864882,92.08929638)(126.23047559,92.0852485)(126.18353386,92.09327622)
\curveto(126.15896693,92.0974337)(126.12854173,92.10574866)(126.09630236,92.12200063)
\curveto(126.06342047,92.1382526)(126.03016063,92.16187465)(126.00102047,92.19407622)
\curveto(125.97191811,92.2262022)(125.95128189,92.26214551)(125.93809134,92.29774866)
\curveto(125.92524094,92.33252031)(125.91994961,92.36441953)(125.9184378,92.39083843)
\curveto(125.9169411,92.41691717)(125.91883087,92.43910299)(125.92136693,92.45618646)
\curveto(125.92392189,92.47357228)(125.92741417,92.48793449)(125.93081575,92.49927307)
\curveto(125.93724094,92.52119433)(125.9448,92.53801323)(125.94971339,92.54810457)
\curveto(125.9591622,92.56775811)(125.96861102,92.58212031)(125.97239055,92.58748724)
\curveto(125.96596535,92.58431244)(125.94215433,92.55989669)(125.91558425,92.53124787)
\curveto(125.88874961,92.50252346)(125.8575685,92.46809197)(125.82706772,92.43154394)
\curveto(125.7968315,92.39526047)(125.7631937,92.35198488)(125.73431811,92.3062526)
\curveto(125.71957795,92.28281953)(125.70408189,92.25553134)(125.69024882,92.2254463)
\curveto(125.67702047,92.19596598)(125.66190236,92.15639433)(125.6535874,92.10986835)
\curveto(125.64942992,92.08567937)(125.64678425,92.0574463)(125.64791811,92.02615181)
\curveto(125.64910488,91.99440378)(125.65434331,91.95857386)(125.66681575,91.92149669)
\curveto(125.67966614,91.88351244)(125.69894173,91.84666205)(125.72513386,91.81302425)
\curveto(125.75159055,91.77938646)(125.78205354,91.75255181)(125.81365039,91.73191559)
\curveto(125.84086299,91.71415181)(125.86815118,91.69676598)(125.89797165,91.68425575)
\curveto(125.92782992,91.67178331)(125.9600315,91.6646022)(125.99234646,91.66498016)
\curveto(126.02485039,91.66527496)(126.05674961,91.67291717)(126.0863811,91.68614551)
\curveto(126.11586142,91.69899591)(126.14307402,91.71713764)(126.16983307,91.73531717)
\curveto(126.24428976,91.78588724)(126.34429606,91.83551244)(126.46554331,91.89330142)
\curveto(126.59193071,91.95354709)(126.72916535,92.01726992)(126.85853102,92.09335181)
\curveto(126.92780976,92.13409512)(126.99538772,92.17873134)(127.05861921,92.22892346)
\curveto(127.12570583,92.28217701)(127.18825701,92.34196913)(127.24249323,92.41018961)
\curveto(127.3011137,92.48392819)(127.34832,92.56537701)(127.38093732,92.65495181)
\curveto(127.41684283,92.75306835)(127.4323011,92.85352819)(127.4315074,92.95364787)
\curveto(127.43389984,93.06196913)(127.42168063,93.1726337)(127.40996409,93.25566992)
\curveto(127.40580661,93.28590614)(127.40089323,93.31572661)(127.3963578,93.34429984)
\curveto(127.37670425,93.46172976)(127.35319559,93.57398173)(127.33369323,93.69141165)
\curveto(127.31328378,93.81413291)(127.30610268,93.89977701)(127.30799244,93.97177701)
\curveto(127.3104378,94.0718211)(127.33104756,94.1310085)(127.34718614,94.16668724)
\curveto(127.52411834,94.55676455)(126.93900212,94.82216298)(126.76206992,94.43208567)
\closepath
}
}
{
\newrgbcolor{curcolor}{0 1 1}
\pscustom[linestyle=none,fillstyle=solid,fillcolor=curcolor]
{
\newpath
\moveto(129.55579843,85.93502362)
\curveto(128.71199244,84.61947591)(127.48164661,83.00539843)(124.86257386,80.52397228)
\curveto(124.3800189,80.06677417)(123.88830992,79.61372598)(123.35680252,79.12784504)
\curveto(122.51067969,78.35436094)(120.84559748,76.85990929)(119.55586394,75.54453921)
\curveto(118.39549984,74.41458142)(117.65886236,73.39696252)(116.37447307,71.83455496)
\curveto(115.59495307,71.00664945)(114.76806047,70.14841323)(113.99823874,69.2507263)
\curveto(113.19719811,68.31663874)(112.48806425,67.37479559)(111.90444094,66.35882835)
\curveto(111.58786772,65.77662236)(111.45308598,65.40375685)(111.38154331,65.05759748)
\curveto(111.29484098,64.63813562)(111.92403405,64.50808207)(112.01073638,64.92754394)
\curveto(112.06614425,65.19558803)(112.17333165,65.50834772)(112.46524346,66.04537701)
\curveto(113.01884976,67.00891843)(113.70126614,67.91744504)(114.48595276,68.83246488)
\curveto(115.24334362,69.71565354)(116.05543181,70.55843906)(116.85665764,71.41046929)
\curveto(118.17759874,73.01621669)(118.87740472,73.98704882)(120.00937701,75.08947654)
\curveto(121.28847874,76.39388598)(122.93499213,77.87174173)(123.79029921,78.65362016)
\curveto(124.3227326,79.14034394)(124.81774488,79.59641197)(125.30446488,80.05755591)
\curveto(127.95655181,82.57026142)(129.22150299,84.22378583)(130.09660346,85.58813102)
\curveto(130.32786006,85.94866756)(129.78705502,86.2955526)(129.55579843,85.93501606)
\closepath
}
}
{
\newrgbcolor{curcolor}{0 1 1}
\pscustom[linestyle=none,fillstyle=solid,fillcolor=curcolor]
{
\newpath
\moveto(130.96917921,79.12848378)
\curveto(130.69898079,78.81792)(130.44812598,78.58232693)(129.64153323,77.93369197)
\curveto(129.64487055,77.93637543)(129.58631433,77.88924472)(129.58964031,77.89192819)
\curveto(129.32065134,77.67524787)(128.7000378,77.18768504)(128.24513386,76.62985701)
\curveto(127.79315528,76.15595339)(127.05404976,75.4084989)(126.48482268,74.74523717)
\curveto(125.95121764,74.12348598)(125.40922961,73.4054589)(124.99306583,72.57224693)
\curveto(124.31414929,71.57848819)(123.53366551,70.7591811)(122.70430866,69.75728126)
\curveto(120.26857323,66.94346457)(118.2966652,64.74575622)(116.23386709,62.1585789)
\curveto(114.83710488,60.48487181)(113.93995465,59.63766047)(113.06885291,58.98115654)
\curveto(112.72798653,58.72319525)(113.1136503,58.21146475)(113.45554394,58.46806299)
\curveto(114.37749543,59.16288756)(115.30972346,60.0484422)(116.7316989,61.75248756)
\curveto(118.78834016,64.3318148)(120.74598425,66.51330142)(123.19466079,69.34219087)
\curveto(123.9980863,70.31264126)(124.83244724,71.19417449)(125.5364485,72.22869921)
\curveto(125.54476346,72.2407937)(125.55232252,72.25364409)(125.55874772,72.26672126)
\curveto(125.94314457,73.04181921)(126.45225827,73.72068283)(126.97246488,74.32682079)
\curveto(127.52745071,74.97348661)(128.2435011,75.69710362)(128.72695937,76.2054652)
\curveto(129.14697071,76.71903118)(129.7089222,77.16294425)(129.9927685,77.39159055)
\curveto(129.99515717,77.39351433)(130.04186457,77.43112441)(130.04424567,77.43301417)
\curveto(130.86250583,78.09102992)(131.14725543,78.35419465)(131.45399811,78.70679055)
\curveto(131.73512933,79.0299501)(131.2503898,79.4516471)(130.96925858,79.12848756)
\closepath
}
}
{
\newrgbcolor{curcolor}{0 1 1}
\pscustom[linestyle=none,fillstyle=solid,fillcolor=curcolor]
{
\newpath
\moveto(128.85510047,68.36849386)
\curveto(128.83658079,68.30405291)(128.82864378,68.23522772)(128.82864378,68.16039307)
\curveto(128.82888945,68.10354898)(128.83393512,68.04583559)(128.84111622,67.98642142)
\curveto(128.84754142,67.93460409)(128.85472252,67.88599937)(128.86152567,67.83607181)
\curveto(128.86757291,67.79154898)(128.8724863,67.75178835)(128.87475402,67.71713008)
\curveto(128.87694236,67.68387024)(128.8760315,67.66429228)(128.87470488,67.65461669)
\curveto(128.87376,67.64781354)(128.87305323,67.65915213)(128.88868913,67.6833411)
\curveto(128.89662614,67.69543559)(128.90758677,67.70828598)(128.92157102,67.72000252)
\curveto(128.93555528,67.7313411)(128.94916157,67.73890016)(128.96072693,67.74343559)
\curveto(128.98189228,67.75175055)(128.98491591,67.74699591)(128.95957795,67.74872693)
\curveto(128.92896378,67.75076409)(128.89831181,67.74842079)(128.86841575,67.74192378)
\curveto(128.79830551,67.72642772)(128.72898898,67.70715213)(128.66061732,67.68432378)
\curveto(128.41615748,67.60295055)(128.15599748,67.46692535)(127.89885354,67.25204031)
\curveto(127.64845984,67.04280567)(127.41828283,66.7727622)(127.22415874,66.45122268)
\curveto(126.98105953,66.08419276)(126.75816567,65.66741669)(126.50229165,65.28831874)
\curveto(126.38127118,65.10901795)(126.2531452,64.93920378)(126.11042646,64.78623118)
\curveto(125.97118866,64.63701543)(125.8180422,64.5037115)(125.6430274,64.3924422)
\curveto(125.31401953,64.16593512)(124.96559622,63.95192315)(124.62314457,63.71088)
\curveto(124.27950992,63.46902803)(123.94383118,63.20020157)(123.65391118,62.86674142)
\curveto(121.86133795,61.07163213)(120.26001638,59.54090079)(118.55362394,58.15323591)
\curveto(118.12772787,57.82721386)(117.68985449,57.53783433)(117.30762331,57.25788472)
\curveto(116.74944756,56.84906457)(116.38884283,56.52589228)(116.09348409,56.19391748)
\curveto(115.80876989,55.87391258)(116.28877745,55.44684108)(116.57349165,55.76684598)
\curveto(116.83076409,56.05601764)(117.15457512,56.34939969)(117.68726173,56.73954898)
\curveto(118.0547452,57.00868913)(118.49273575,57.29748283)(118.9516422,57.64896)
\curveto(120.6913663,59.0635578)(122.31118866,60.61285417)(124.12372535,62.42903811)
\curveto(124.38061984,62.72334992)(124.66963654,62.95791496)(124.99295244,63.18548787)
\curveto(125.3132674,63.41093669)(125.66008441,63.62410961)(125.99772472,63.85686425)
\curveto(126.2162948,63.99557291)(126.41057764,64.1661052)(126.58020661,64.34793449)
\curveto(126.7531578,64.53332031)(126.90210898,64.73222551)(127.03483465,64.92887811)
\curveto(127.31693858,65.34684094)(127.53860787,65.76251339)(127.76710299,66.1079622)
\curveto(127.93491402,66.38541732)(128.11988409,66.59945953)(128.31085606,66.75903874)
\curveto(128.50542614,66.92167181)(128.69448567,67.01842772)(128.86356283,67.07474268)
\curveto(128.89909039,67.08645921)(128.93465575,67.09704189)(128.97052346,67.1057348)
\curveto(129.04592504,67.10553449)(129.12155339,67.11631748)(129.19570772,67.14549543)
\curveto(129.24344315,67.16439307)(129.28811717,67.18994268)(129.32829354,67.22278677)
\curveto(129.36933921,67.25642457)(129.40229669,67.29463559)(129.42826205,67.33481197)
\curveto(129.48102425,67.41644976)(129.50196283,67.50133795)(129.51099591,67.56702614)
\curveto(129.52082268,67.63728756)(129.51931087,67.70513008)(129.51553134,67.7593285)
\curveto(129.51175181,67.81844031)(129.5038148,67.87747654)(129.49738961,67.92404031)
\curveto(129.48907465,67.98443717)(129.48378331,68.01883087)(129.47811402,68.06358047)
\curveto(129.47244472,68.10999307)(129.47055496,68.14064504)(129.47017701,68.16309543)
\curveto(129.47005606,68.19068598)(129.47295118,68.19597732)(129.47107276,68.18955213)
\curveto(129.58975775,68.60060714)(128.9736071,68.77926169)(128.85399685,68.36847496)
\closepath
}
}
{
\newrgbcolor{curcolor}{0 1 1}
\pscustom[linestyle=none,fillstyle=solid,fillcolor=curcolor]
{
\newpath
\moveto(126.97563591,55.7598274)
\curveto(126.97979339,55.68268724)(126.99188787,55.61393764)(127.00965165,55.5166148)
\curveto(127.02136819,55.45240063)(127.03119496,55.39604787)(127.03308472,55.35054236)
\curveto(127.03379906,55.33164472)(127.03287307,55.31917228)(127.03174299,55.31202898)
\curveto(127.03073764,55.30560378)(127.03003087,55.30598173)(127.03220409,55.31066457)
\curveto(127.04505449,55.32729449)(127.04505449,55.32578268)(127.03371591,55.32049134)
\curveto(127.0129285,55.30990866)(127.00423559,55.30575118)(126.99486236,55.3015937)
\curveto(126.95782299,55.28496378)(126.91318677,55.26908976)(126.85142929,55.25147717)
\curveto(126.78846236,55.23371339)(126.71041512,55.21594961)(126.61690961,55.1912315)
\curveto(126.53735055,55.17006614)(126.43594583,55.1408126)(126.33461669,55.09489134)
\curveto(126.23895685,55.05154016)(126.13536,54.99016063)(126.03894425,54.8976378)
\curveto(125.86833638,54.7559811)(125.70929386,54.58129134)(125.54948031,54.4249852)
\curveto(125.47400315,54.35113323)(125.4000378,54.28313953)(125.32365354,54.22308283)
\curveto(125.24923465,54.16453795)(125.17364409,54.11449701)(125.0935937,54.07450961)
\curveto(124.70737512,53.84717102)(124.31516598,53.69301165)(123.88060724,53.50364976)
\curveto(123.45623433,53.31871748)(123.0184063,53.11028787)(122.62889575,52.79733165)
\curveto(121.83739465,52.0795011)(121.11021354,51.19030299)(120.55028409,50.54164535)
\curveto(119.73178205,49.5813052)(119.07959811,48.78604346)(118.68218079,48.33115465)
\curveto(118.40036926,48.00858975)(118.88421682,47.58587227)(119.16602835,47.90843717)
\curveto(119.57002205,48.37085858)(120.22969323,49.17501732)(121.03795654,50.12334992)
\curveto(121.61546835,50.79236787)(122.30676283,51.63785197)(123.0460422,52.30906205)
\curveto(123.34991622,52.55246362)(123.7206085,52.73308724)(124.1372674,52.91464819)
\curveto(124.54270866,53.0913411)(124.9899515,53.26784504)(125.40064252,53.51058898)
\curveto(125.50805669,53.56338898)(125.62023307,53.63897953)(125.72084409,53.71808504)
\curveto(125.82296693,53.7984)(125.91541417,53.88423307)(125.99871496,53.96568189)
\curveto(126.19404094,54.15669921)(126.30538583,54.28391811)(126.46743307,54.41942929)
\curveto(126.51565984,54.46467024)(126.55349291,54.48870803)(126.59979213,54.50968441)
\curveto(126.65157165,54.53311748)(126.71162835,54.55171276)(126.78117165,54.57011906)
\curveto(126.85834961,54.5905285)(126.94267087,54.60965291)(127.02612283,54.63323717)
\curveto(127.10810079,54.65629228)(127.18490079,54.68244661)(127.25943307,54.7161222)
\curveto(127.27795276,54.72443717)(127.29685039,54.73350803)(127.31540787,54.74295685)
\curveto(127.3711937,54.76827969)(127.43253543,54.80879622)(127.48518425,54.85891276)
\curveto(127.54028976,54.9113726)(127.58141102,54.97010646)(127.61085354,55.03107024)
\curveto(127.64146772,55.09430173)(127.65787087,55.15673953)(127.66660157,55.21267654)
\curveto(127.67567244,55.27175055)(127.6768063,55.32723402)(127.67529449,55.37451591)
\curveto(127.6715263,55.47456)(127.65186142,55.57789228)(127.64203465,55.63201512)
\curveto(127.62351496,55.73266394)(127.61897953,55.76331591)(127.61746772,55.79389228)
\curveto(127.59453859,56.22160863)(126.95296379,56.18721493)(126.97589291,55.75949858)
\closepath
}
}
{
\newrgbcolor{curcolor}{1 1 1}
\pscustom[linestyle=none,fillstyle=solid,fillcolor=curcolor]
{
\newpath
\moveto(52.92643654,49.68554457)
\curveto(52.91245228,49.79507528)(52.91509795,49.90721386)(52.94760189,50.02547528)
\curveto(52.97292472,50.11814929)(53.01774992,50.21743748)(53.09281134,50.32715717)
\curveto(53.15778142,50.42213669)(53.24300976,50.52150047)(53.35613102,50.63273197)
\curveto(53.45817827,50.73307843)(53.57197984,50.83270677)(53.70887433,50.94881386)
\curveto(53.92241764,51.12992882)(54.24795591,51.39312)(54.49737071,51.67533354)
\curveto(54.61691717,51.81060283)(54.7342715,51.96571465)(54.83117858,52.1433789)
\curveto(54.91565102,52.29822614)(54.98564787,52.47208441)(55.02809197,52.66757669)
\curveto(55.49583496,53.52631181)(55.92251717,54.43040126)(56.32756157,55.39603276)
\curveto(56.82170079,56.57408882)(57.26650205,57.79787339)(57.78232063,59.31030803)
\curveto(58.40509984,61.13636787)(59.25847559,63.81778772)(60.23128441,66.05013543)
\curveto(61.73457638,69.7750148)(63.52854425,73.68178016)(65.78244283,78.14268472)
\curveto(66.14085543,78.77097449)(66.38384504,79.44744189)(66.62103685,80.15147339)
\curveto(66.84198803,80.80731213)(67.05561071,81.47975433)(67.35668787,82.10125606)
\curveto(69.56430236,86.41833071)(72.28296945,90.70745953)(74.4771137,95.41141039)
\curveto(75.30444094,97.09196976)(76.37410772,98.67896693)(77.80713071,99.92538331)
\curveto(77.82149291,99.93785575)(77.83434331,99.95146205)(77.8463622,99.96608882)
\curveto(77.92531654,100.06401638)(78.02014488,100.18715339)(78.1207937,100.31637543)
\curveto(78.22287874,100.44741165)(78.33392126,100.58850142)(78.45063307,100.72949669)
\curveto(78.56753386,100.87066205)(78.68855433,101.00986205)(78.81002835,101.13685417)
\curveto(78.93210709,101.26448882)(79.05104882,101.37602268)(79.16333858,101.46374551)
\curveto(79.21927559,101.50743685)(79.27181102,101.5438337)(79.32056693,101.57289827)
\curveto(79.36954961,101.60200063)(79.41229606,101.62233449)(79.44869291,101.63537386)
\curveto(79.48497638,101.6486022)(79.51192441,101.65313764)(79.53029291,101.65389354)
\curveto(79.54767874,101.65472882)(79.55788346,101.6523137)(79.56506457,101.64973606)
\curveto(79.57186772,101.64724913)(79.5832063,101.64179906)(79.59945827,101.62630299)
\curveto(79.61646614,101.61005102)(79.63955906,101.58162898)(79.66533543,101.53378016)
\curveto(79.69103622,101.48574236)(79.71817323,101.42115024)(79.74364724,101.33565732)
\curveto(79.76897008,101.2502022)(79.79225197,101.14683213)(79.81167874,101.02290142)
\curveto(80.28049134,99.28188472)(80.33221417,97.52824819)(80.3022652,95.77026898)
\curveto(80.27127307,93.94954961)(80.15618646,92.17691339)(80.29017071,90.38172094)
\curveto(80.52007937,80.37205039)(82.25796661,70.52650583)(84.12443339,61.69379906)
\curveto(85.05674079,57.50820661)(85.85572157,54.06671622)(86.78827087,50.63445921)
\curveto(87.08292283,49.72739528)(87.43642205,48.94544126)(87.69045921,48.23935748)
\curveto(87.84239622,47.81701417)(87.9585789,47.43806362)(88.04135055,47.06989984)
\curveto(88.13995843,46.63132346)(88.18304504,46.24200945)(88.18538835,45.88173732)
\curveto(88.18816,45.45341625)(88.83064189,45.45757373)(88.82787024,45.8858948)
\curveto(88.82519055,46.29567118)(88.77609071,46.73088378)(88.6681852,47.21081575)
\curveto(88.57789228,47.61234898)(88.4529411,48.0177978)(88.29499465,48.45681638)
\curveto(88.03084346,49.19101228)(87.6782022,49.97434205)(87.40374425,50.81802331)
\curveto(86.47974047,54.22031244)(85.68335622,57.65002205)(84.75227717,61.83000945)
\curveto(82.88900409,70.64770394)(81.16100031,80.44699843)(80.93166236,90.41300031)
\curveto(80.79964346,92.18561764)(80.91276472,93.8984315)(80.94451276,95.75927433)
\curveto(80.97474898,97.5302022)(80.92523717,99.35860535)(80.43901984,101.15639433)
\curveto(80.42352378,101.26857071)(80.39449701,101.40077858)(80.35923402,101.51911559)
\curveto(80.32408441,101.63703685)(80.2817537,101.74403528)(80.23073008,101.83863685)
\curveto(80.17974425,101.93320063)(80.11851591,102.01835339)(80.04458835,102.0894085)
\curveto(79.96994268,102.16118173)(79.8841474,102.21681638)(79.78810961,102.25238173)
\curveto(79.69233638,102.28790929)(79.59463559,102.30030614)(79.49958047,102.29573291)
\curveto(79.40565921,102.29119748)(79.31574425,102.27041008)(79.2321789,102.24028724)
\curveto(79.14872693,102.21042898)(79.06829858,102.17044157)(78.99195213,102.12497386)
\curveto(78.91545449,102.07943055)(78.84054425,102.02685732)(78.76790173,101.97012661)
\curveto(78.62329701,101.85715654)(78.48096,101.72230299)(78.34582299,101.58103181)
\curveto(78.21036472,101.43941291)(78.07879937,101.28781606)(77.95581732,101.13929197)
\curveto(77.83286929,100.99079433)(77.71717795,100.84369512)(77.61403465,100.71129071)
\curveto(77.51754331,100.58743559)(77.43450709,100.47964346)(77.36379213,100.39124031)
\curveto(75.8582211,99.07510677)(74.74858961,97.4174211)(73.89785953,95.68916409)
\curveto(71.72454047,91.03002709)(69.00972472,86.74488189)(66.78165921,82.38764598)
\curveto(66.4553726,81.71428535)(66.22704,80.99416063)(66.01226835,80.35668283)
\curveto(65.77589669,79.65506646)(65.54931024,79.03047307)(65.21669669,78.44666835)
\curveto(62.94894614,73.95923906)(61.1470337,70.03576063)(59.63898331,66.29877543)
\curveto(58.6557052,64.04270362)(57.78949039,61.32152315)(57.17431937,59.51778142)
\curveto(56.66054929,58.01134866)(56.2211452,56.8032)(55.73517732,55.64463496)
\curveto(55.3304315,54.67970268)(54.90570709,53.78209512)(54.44121071,52.93330016)
\curveto(54.42495874,52.90381984)(54.41362016,52.87165606)(54.40719496,52.83843402)
\curveto(54.38036031,52.69587024)(54.33103748,52.56827339)(54.26712567,52.45112315)
\curveto(54.19905638,52.32636094)(54.11280756,52.21048063)(54.01590047,52.10087433)
\curveto(53.80545638,51.86276409)(53.52802016,51.63798425)(53.29325102,51.43887118)
\curveto(53.15363528,51.32045858)(53.02513134,51.20843339)(52.9056,51.09089008)
\curveto(52.77191811,50.95943811)(52.65713386,50.8283263)(52.56249449,50.68996157)
\curveto(52.45130079,50.52744189)(52.37419843,50.36375055)(52.32805039,50.19576567)
\curveto(52.26954331,49.98290268)(52.26617953,49.78228535)(52.28919685,49.60352504)
\curveto(52.34389919,49.17870129)(52.98113509,49.26075483)(52.92643276,49.68557858)
\closepath
}
}
{
\newrgbcolor{curcolor}{1 1 1}
\pscustom[linestyle=none,fillstyle=solid,fillcolor=curcolor]
{
\newpath
\moveto(60.89907024,70.43176819)
\curveto(61.01846551,70.42383118)(61.14379465,70.42966299)(61.32500031,70.44915402)
\curveto(61.43487118,70.46087055)(61.57773732,70.47976819)(61.69335307,70.48619339)
\curveto(61.79759244,70.49186268)(61.89049323,70.48907717)(61.96872945,70.47031937)
\curveto(62.0035011,70.46200441)(62.03445543,70.45066583)(62.06249953,70.43592567)
\curveto(62.07723969,70.42798866)(62.09197984,70.4189178)(62.10649323,70.40833512)
\curveto(62.18215937,70.27064693)(62.28473575,70.14270992)(62.42008063,70.02927874)
\curveto(62.54862236,69.9215622)(62.69583496,69.83584252)(62.85883843,69.76433386)
\curveto(63.00348094,69.70087559)(63.15401953,69.65128819)(63.3079748,69.60600945)
\curveto(63.44830866,69.56473701)(63.58860472,69.528)(63.71624693,69.48948661)
\curveto(65.44838551,69.06611906)(67.08219969,68.93302677)(69.06941858,68.72576882)
\curveto(70.61732787,68.6035389)(72.29907024,68.43546331)(73.94878866,68.56389165)
\curveto(74.03688945,68.55595465)(74.12147528,68.55897827)(74.20269732,68.57145071)
\curveto(74.29431307,68.58505701)(74.37855874,68.61019087)(74.45600126,68.64171213)
\curveto(74.60865638,68.70380976)(74.73840756,68.79312)(74.84376567,68.87063811)
\curveto(74.96146016,68.95722709)(75.03852472,69.0203074)(75.13259717,69.08009953)
\curveto(75.17602394,69.10769008)(75.21627591,69.12983811)(75.25528063,69.14699717)
\curveto(75.29432315,69.16438299)(75.33200504,69.17647748)(75.36976252,69.18365858)
\curveto(76.15121008,69.38971843)(76.92354142,69.50359559)(77.66928,69.58071685)
\curveto(78.45093543,69.66156094)(79.30824945,69.71039244)(80.01802205,69.75956409)
\curveto(81.38802142,69.83379402)(82.36608756,69.85299402)(83.08660535,69.85511055)
\curveto(83.51459255,69.85671699)(83.51268767,70.49920644)(83.08470047,70.4976)
\curveto(82.35344882,70.49543433)(81.36441449,70.47605669)(79.97842772,70.4008063)
\curveto(79.2760063,70.35216378)(78.39922772,70.30212283)(77.6031874,70.21980472)
\curveto(76.83987024,70.14085039)(76.0313537,70.02258898)(75.22586835,69.80948409)
\curveto(75.15621165,69.7966337)(75.07313764,69.76874079)(74.99633764,69.73491402)
\curveto(74.9206337,69.70165417)(74.85143055,69.66257386)(74.78804787,69.62232189)
\curveto(74.66952189,69.54699591)(74.55114709,69.45292346)(74.46308409,69.38814236)
\curveto(74.3670085,69.3174652)(74.28975496,69.26765102)(74.21389984,69.23681008)
\curveto(74.17761638,69.22206992)(74.14258016,69.21224315)(74.10731717,69.20695181)
\curveto(74.07216756,69.20166047)(74.03399433,69.20052661)(73.99041638,69.20571591)
\curveto(73.96925102,69.20821795)(73.94808567,69.20862614)(73.92692031,69.20693669)
\curveto(72.33119622,69.07930205)(70.70038677,69.24170835)(69.1280126,69.36582803)
\curveto(67.12898646,69.57442016)(65.54799874,69.70345323)(63.88523339,70.10941228)
\curveto(63.75763654,70.14837921)(63.61775622,70.18492724)(63.48919559,70.22272252)
\curveto(63.34666961,70.26463748)(63.22561134,70.3053052)(63.11698772,70.35296504)
\curveto(62.99623181,70.40595402)(62.90442709,70.46192882)(62.83272945,70.52202331)
\curveto(62.74825701,70.59281386)(62.69197984,70.6692737)(62.65286173,70.75049575)
\curveto(62.64454677,70.76788157)(62.63585386,70.78488945)(62.62451528,70.80034772)
\curveto(62.61317669,70.81584378)(62.59957039,70.82945008)(62.58596409,70.84271622)
\curveto(62.57916094,70.84914142)(62.5723578,70.85556661)(62.5659326,70.86161386)
\curveto(62.50050898,70.92114142)(62.43145701,70.96827213)(62.36089323,71.0052737)
\curveto(62.27929323,71.04805795)(62.19731528,71.0764422)(62.11896567,71.09526425)
\curveto(61.95145701,71.13551622)(61.78617827,71.13510047)(61.65792378,71.12814614)
\curveto(61.51989543,71.12058709)(61.34070803,71.09753197)(61.25595213,71.08838551)
\curveto(61.09286551,71.07062173)(61.00843087,71.06873197)(60.9409285,71.0732674)
\curveto(60.51350444,71.10108471)(60.47177846,70.45994833)(60.89920252,70.43213102)
\closepath
}
}
{
\newrgbcolor{curcolor}{0.10196079 0.10196079 0.10196079}
\pscustom[linestyle=none,fillstyle=solid,fillcolor=curcolor]
{
\newpath
\moveto(146.85982488,89.55903874)
\curveto(146.7748611,89.18161512)(146.75233512,88.77751181)(146.78423433,88.29864945)
\curveto(146.81106898,87.89609197)(146.86538079,87.54785764)(146.9221115,87.09888378)
\curveto(146.96735244,86.74096252)(147.00813354,86.35457764)(146.99263748,85.98186331)
\curveto(146.97940913,85.66120819)(146.9244926,85.33296756)(146.78306268,85.00852157)
\curveto(146.62356661,84.82797354)(146.50024063,84.62625638)(146.4100989,84.40465134)
\curveto(146.32037291,84.18404031)(146.26655244,83.95114205)(146.23835717,83.70899906)
\curveto(146.21190047,83.48200063)(146.2084989,83.25060661)(146.21719181,83.01647244)
\curveto(146.22512882,82.80145512)(146.2428926,82.59031559)(146.26122331,82.37988283)
\curveto(146.18748472,80.8822526)(145.99782803,79.38398362)(145.75078299,77.75398677)
\curveto(145.61929323,76.8864189)(145.47408378,75.9984378)(145.33280126,75.08809323)
\curveto(145.23559181,74.46181795)(145.13343118,73.78044094)(145.04264693,73.08922961)
\curveto(144.46745575,68.80392945)(143.6446337,62.74471937)(143.62388409,56.95046929)
\curveto(143.62376693,56.9175874)(143.62879748,56.88489449)(143.63862425,56.8535622)
\curveto(143.67415181,56.73949606)(143.72377701,56.63344252)(143.78443843,56.53460787)
\curveto(143.84362583,56.43819213)(143.91271559,56.34975118)(143.98845732,56.268)
\curveto(144.13480063,56.11001575)(144.30722268,55.97554016)(144.48460724,55.85430803)
\curveto(144.65563087,55.73744504)(144.83500724,55.63044661)(145.00328693,55.52632063)
\curveto(145.16777197,55.42450016)(145.31823496,55.32751748)(145.4441726,55.22618835)
\curveto(145.76766236,54.9913663)(146.10441071,54.85545449)(146.45589165,54.79292976)
\curveto(146.79578457,54.73245732)(147.14295685,54.74190614)(147.48485669,54.7982211)
\curveto(147.81802205,54.85291087)(148.14754016,54.95201008)(148.46430236,55.07446677)
\curveto(148.77414803,55.1942022)(149.07185008,55.33631244)(149.34900283,55.48121197)
\curveto(149.74846488,55.69354583)(150.17143181,55.86922961)(150.61856882,56.03783055)
\curveto(151.05507402,56.20242898)(151.50907465,56.35776756)(151.96586835,56.53137638)
\curveto(152.41256315,56.70115276)(152.86767118,56.89096063)(153.30360189,57.12735496)
\curveto(153.73359118,57.36055181)(154.15130079,57.64233449)(154.53288945,58.00340787)
\curveto(154.95676346,58.38332976)(155.39328,58.77797291)(155.84798362,59.13578835)
\curveto(156.29199874,59.48520567)(156.75940913,59.80452661)(157.27105134,60.05674205)
\curveto(157.80191244,60.31877669)(158.25124913,60.68530772)(158.63433827,61.11502866)
\curveto(159.0093052,61.53563717)(159.31912441,62.01478299)(159.58064126,62.51225953)
\curveto(159.66866646,62.67190677)(159.78802394,62.82429732)(159.92869795,62.98055811)
\curveto(160.06736882,63.13461165)(160.22440819,63.28896756)(160.38142866,63.45753827)
\curveto(160.53593575,63.62338394)(160.69448693,63.80808945)(160.82482394,64.02104693)
\curveto(160.95226961,64.2292611)(161.05340976,64.46528504)(161.10171213,64.74114142)
\curveto(161.23970268,65.18144504)(161.40921449,65.67969638)(161.51247496,66.1785411)
\curveto(161.56531276,66.43388598)(161.60284346,66.69734929)(161.61157417,66.96551433)
\curveto(161.62026709,67.23287811)(161.60023559,67.50584693)(161.53806236,67.78095874)
\curveto(161.53314898,67.80363591)(161.52710173,67.82597291)(161.51727496,67.8468737)
\curveto(161.50744819,67.8676611)(161.49384189,67.88663433)(161.47985764,67.90496504)
\curveto(161.20667339,68.26269732)(160.89240567,68.59601386)(160.60840441,68.9169789)
\curveto(160.32119811,69.24156472)(160.05750047,69.56116913)(159.85170142,69.91700031)
\curveto(159.83696126,69.94232315)(159.81881953,69.96579402)(159.79772976,69.98631685)
\curveto(159.41768693,70.35644598)(158.99439118,70.64851276)(158.54451402,70.8815622)
\curveto(158.09593323,71.11392756)(157.62144378,71.28738142)(157.13749039,71.42150551)
\curveto(156.17502614,71.68826457)(155.15877921,71.8033474)(154.22259402,71.91031937)
\curveto(153.31631622,71.96330835)(152.43229606,72.06580913)(151.53271559,72.16128)
\curveto(150.63638173,72.25641071)(149.72961638,72.34390677)(148.81490646,72.36121701)
\curveto(148.18119307,72.3612737)(147.55498583,72.3612737)(146.93233512,72.3612737)
\curveto(146.8773052,72.3612737)(146.82212409,72.36109228)(146.76762331,72.35371465)
\curveto(146.71312252,72.34615559)(146.65918866,72.33141543)(146.61065953,72.30560126)
\curveto(146.56213039,72.27952252)(146.51930835,72.24248315)(146.4892611,72.19637291)
\curveto(146.45940283,72.15026268)(146.44243276,72.09553512)(146.44235717,72.04054299)
\curveto(146.44230047,71.98555087)(146.45898709,71.93074772)(146.48892094,71.88459969)
\curveto(146.51877921,71.83845165)(146.56160126,71.80114772)(146.6100926,71.77514457)
\curveto(146.65854614,71.74906583)(146.7124422,71.73425008)(146.76694299,71.72672882)
\curveto(146.82144378,71.71916976)(146.87662488,71.71879181)(146.9316548,71.71879181)
\curveto(148.05594331,71.71647496)(149.20789795,71.64225638)(150.36206362,71.65797921)
\curveto(151.52489197,71.67385323)(152.69132598,71.78141858)(153.83155654,72.15028157)
\curveto(156.27618142,72.86644157)(158.67208819,73.7387263)(160.87243465,74.97773102)
\curveto(161.22170079,75.2043515)(161.51658331,75.45328252)(161.78854677,75.71327622)
\curveto(162.06150425,75.9742148)(162.31697764,76.25190425)(162.5622085,76.52055307)
\curveto(162.81271559,76.79498457)(163.05141543,77.05872378)(163.30356283,77.3086148)
\curveto(163.55645102,77.55927307)(163.81542803,77.78828598)(164.09634142,77.98593638)
\curveto(164.11221543,77.99727496)(164.12733354,78.0097474)(164.14078866,78.02373165)
\curveto(164.34439181,78.23126551)(164.49746268,78.46228913)(164.61627969,78.7023685)
\curveto(164.73586394,78.9439937)(164.82158362,79.19649638)(164.88904819,79.44517795)
\curveto(165.02677417,79.95293102)(165.09620409,80.45721449)(165.20256,80.89276724)
\curveto(165.27493795,81.12018142)(165.31367811,81.37123276)(165.31103244,81.61515591)
\curveto(165.30836409,81.86260157)(165.26310803,82.10359559)(165.17239937,82.32613039)
\curveto(165.08010331,82.55256189)(164.94188598,82.75708724)(164.7584126,82.92759307)
\curveto(164.59744252,83.07714898)(164.40548031,83.19703559)(164.18612409,83.28362457)
\curveto(163.89237921,83.71098709)(163.55030929,84.09500976)(163.16475213,84.41840126)
\curveto(162.7351937,84.77870362)(162.25521638,85.06080378)(161.73148346,85.2469115)
\curveto(160.88646803,85.5660548)(159.94192252,85.7972674)(159.14531906,86.02536945)
\curveto(158.25706583,86.27969386)(157.46108976,86.5516611)(156.73123654,86.97715276)
\curveto(155.87161323,87.44711433)(155.0491578,87.88014236)(154.18323024,88.2112063)
\curveto(153.24226394,88.57094173)(152.31417449,88.78337764)(151.36346457,88.78447748)
\curveto(150.70481764,88.81811528)(150.07956661,88.81282394)(149.51253543,88.83912945)
\curveto(148.82384882,88.87125543)(148.26375307,88.94185701)(147.74094614,89.08683969)
\curveto(147.32819671,89.20130893)(147.15649278,88.58218452)(147.5692422,88.46771528)
\curveto(148.15505386,88.30527118)(148.76716724,88.23054992)(149.4827452,88.19732787)
\curveto(150.07162205,88.17011528)(150.69943181,88.17502866)(151.34682331,88.14241134)
\curveto(152.21615244,88.14100913)(153.06560504,87.95063811)(153.95377134,87.61108913)
\curveto(154.77567496,87.29685921)(155.56512756,86.88243402)(156.41535496,86.41774488)
\curveto(157.20809197,85.95545575)(158.06611654,85.66608378)(158.96844094,85.4077115)
\curveto(159.80053417,85.16945008)(160.69240063,84.95258835)(161.51041134,84.64368756)
\curveto(161.96288126,84.48283087)(162.37669039,84.24082016)(162.75185008,83.92615181)
\curveto(163.11222803,83.6238652)(163.43434583,83.2566085)(163.71037606,82.84007055)
\curveto(163.73002961,82.81021228)(163.75161071,82.78069417)(163.78010835,82.7588863)
\curveto(163.8084548,82.73696504)(163.84262173,82.72411465)(163.87663748,82.71273827)
\curveto(164.06013354,82.65162331)(164.20670362,82.56318236)(164.32103811,82.45693984)
\curveto(164.43355465,82.35239811)(164.51916094,82.22661543)(164.57740346,82.08363591)
\curveto(164.63500346,81.94231937)(164.66667591,81.78202961)(164.66856567,81.60821669)
\curveto(164.67041386,81.43673953)(164.64324283,81.25313008)(164.58409323,81.06551055)
\curveto(164.46122079,80.56527874)(164.39190425,80.06655874)(164.26895622,79.61337449)
\curveto(164.20837039,79.3900422)(164.13572787,79.17988157)(164.04044598,78.98733354)
\curveto(163.95056882,78.80576504)(163.84066016,78.63999496)(163.70134677,78.49354583)
\curveto(163.39199244,78.27316157)(163.11318425,78.02455559)(162.85128945,77.76500409)
\curveto(162.58521071,77.50130646)(162.33439748,77.22403654)(162.08766992,76.9537474)
\curveto(161.84177386,76.68434268)(161.59976693,76.42170709)(161.34456189,76.17773102)
\curveto(161.09186268,75.93614362)(160.82816882,75.71485606)(160.5395074,75.5268926)
\curveto(158.41632,74.33209701)(156.07192819,73.47613984)(153.64233827,72.76425449)
\curveto(152.57987906,72.42069543)(151.48527496,72.3158589)(150.3533178,72.30044976)
\curveto(149.23268409,72.28533165)(148.08441449,72.35895685)(146.9329663,72.36133795)
\lineto(146.93230488,71.7188485)
\lineto(148.80879118,71.7188485)
\curveto(149.6872063,71.70221858)(150.56991496,71.61740598)(151.46488441,71.52242646)
\curveto(152.35528063,71.42793827)(153.25957039,71.32309417)(154.16752252,71.27052094)
\curveto(155.09495811,71.16405165)(156.06027591,71.05342488)(156.96586583,70.80243402)
\curveto(157.41723591,70.67733165)(157.84836661,70.51866709)(158.2489474,70.31115213)
\curveto(158.63738835,70.10993008)(158.99699528,69.8628548)(159.31871244,69.55574551)
\curveto(159.55406362,69.15853606)(159.84348094,68.81194205)(160.12717984,68.49131717)
\curveto(160.41249638,68.16888567)(160.68218079,67.88276409)(160.92537827,67.57179213)
\curveto(160.96279559,67.37952756)(160.97560819,67.18446992)(160.96918299,66.98647181)
\curveto(160.96200189,66.76393323)(160.93051843,66.53804598)(160.88308535,66.30885921)
\curveto(160.78659402,65.84267339)(160.62845858,65.38167685)(160.48175244,64.91214614)
\curveto(160.47721701,64.89778394)(160.47381543,64.88304378)(160.47116976,64.86800126)
\curveto(160.44017764,64.67721071)(160.37142803,64.5117052)(160.2764485,64.35652535)
\curveto(160.17939024,64.19793638)(160.05477921,64.04996787)(159.9108926,63.89551748)
\curveto(159.76708157,63.74112378)(159.60709417,63.58408441)(159.45078803,63.41048693)
\curveto(159.29382425,63.23613732)(159.13882583,63.04247433)(159.01452094,62.81681764)
\curveto(158.76749102,62.34701102)(158.48584063,61.91447055)(158.15434961,61.54263307)
\curveto(157.8167622,61.16396598)(157.43142047,60.85264252)(156.98643402,60.63301417)
\curveto(156.42190488,60.35472756)(155.91597354,60.00723402)(155.45026016,59.64075591)
\curveto(154.97592567,59.26748976)(154.52268472,58.85747906)(154.09723087,58.47598488)
\curveto(153.7585852,58.15570772)(153.38943874,57.90506457)(152.99691969,57.69220157)
\curveto(152.59737827,57.47552126)(152.1725178,57.29745638)(151.73721827,57.1320189)
\curveto(151.29401953,56.96356535)(150.84028346,56.80830236)(150.39147969,56.63906646)
\curveto(149.9334274,56.46634205)(149.48148661,56.27952)(149.04899906,56.04963024)
\curveto(148.78753134,55.91296252)(148.51271055,55.78219087)(148.23227339,55.67383181)
\curveto(147.94518047,55.56286488)(147.65924787,55.47809008)(147.38035654,55.43232)
\curveto(147.09462425,55.38541606)(146.8227137,55.38046488)(146.56800756,55.4258948)
\curveto(146.30774929,55.47219402)(146.06223496,55.57144441)(145.83356976,55.73706331)
\curveto(145.68692409,55.8555515)(145.50667843,55.97048693)(145.3409915,56.07302551)
\curveto(145.16051906,56.18471055)(145.00185449,56.27912315)(144.84670488,56.38517669)
\curveto(144.68936315,56.49270425)(144.5601411,56.59626331)(144.45941669,56.70500031)
\curveto(144.4081663,56.7603326)(144.36576,56.81547591)(144.33159307,56.87114835)
\curveto(144.30513638,56.91423496)(144.28340409,56.95796409)(144.26632063,57.0027515)
\curveto(144.29164346,62.73167622)(145.10474835,68.72470299)(145.67927811,73.00506331)
\curveto(145.76926866,73.68995528)(145.87056,74.36571969)(145.96742929,74.98997291)
\curveto(146.1081789,75.89677228)(146.25353953,76.78569071)(146.38575874,77.65813417)
\curveto(146.63520756,79.30402772)(146.82924094,80.83381039)(146.90375811,82.37085732)
\curveto(146.90445732,82.38521953)(146.90418142,82.39995969)(146.90292283,82.41428409)
\curveto(146.88364724,82.63372346)(146.86663937,82.83655559)(146.85896693,83.04053669)
\curveto(146.85102992,83.25377764)(146.85480945,83.4505663)(146.87635276,83.63513575)
\curveto(146.89902992,83.83110425)(146.94117165,84.00586961)(147.00508346,84.16305638)
\curveto(147.0735685,84.33143433)(147.16741417,84.48012094)(147.29032441,84.61220787)
\curveto(147.31337953,84.63715276)(147.33265512,84.66568819)(147.34694173,84.69668031)
\curveto(147.54737008,85.13188913)(147.61804724,85.56287244)(147.6343748,85.95564472)
\curveto(147.65213858,86.3864126)(147.60489449,86.81894929)(147.55931339,87.17988283)
\curveto(147.50005039,87.64890331)(147.45019843,87.96438425)(147.42510236,88.34175874)
\curveto(147.39637795,88.77578835)(147.41829921,89.11569638)(147.48644409,89.41837228)
\curveto(147.58052909,89.8362417)(146.95372468,89.97736926)(146.85963969,89.55949984)
\closepath
}
}
{
\newrgbcolor{curcolor}{0.10196079 0.10196079 0.10196079}
\pscustom[linestyle=none,fillstyle=solid,fillcolor=curcolor]
{
\newpath
\moveto(122.09632252,119.40464013)
\curveto(122.10085795,117.34487962)(122.08233827,111.75016441)(122.0868737,109.69040504)
\curveto(122.09016189,109.62596409)(122.09745638,109.58011843)(122.11030677,109.53476409)
\curveto(122.1126085,109.52682709)(122.11522016,109.51851213)(122.11786583,109.51057512)
\curveto(122.1348737,109.45981606)(122.15849575,109.41121134)(122.18956346,109.36581921)
\curveto(122.21866583,109.32341291)(122.25226583,109.28629795)(122.28904063,109.2539452)
\curveto(122.32343433,109.22370898)(122.36009575,109.19800819)(122.39683276,109.17627591)
\curveto(122.4644863,109.1362885)(122.53546583,109.10782866)(122.60119181,109.08530268)
\curveto(122.66139969,109.06451528)(122.72406425,109.0467137)(122.77780913,109.03049953)
\curveto(122.83053354,109.01462551)(122.87520756,109.00026331)(122.91375874,108.9843515)
\curveto(122.95042016,108.96923339)(122.97370205,108.95600504)(122.9884422,108.94481764)
\curveto(123.00091465,108.93536882)(123.00507213,108.92894362)(123.00733984,108.92516409)
\curveto(123.00915402,108.92183433)(123.0156548,108.90891213)(123.01754457,108.87757984)
\curveto(123.01861795,108.85943811)(123.02008441,108.84129638)(123.02434772,108.82364598)
\curveto(123.0285052,108.8058822)(123.03530835,108.78887433)(123.0421115,108.7722822)
\curveto(123.10477606,108.62295307)(123.15345638,108.45975307)(123.1997178,108.27271559)
\curveto(123.24276661,108.09855496)(123.28180913,107.90912504)(123.32996031,107.70952441)
\curveto(123.37520126,107.52183307)(123.43079811,107.31740976)(123.51073512,107.11724976)
\curveto(123.58840441,106.92279307)(123.69154772,106.72533165)(123.83921386,106.5394885)
\curveto(123.8516863,106.52399244)(123.8652926,106.50963024)(123.88014614,106.49674205)
\curveto(123.95335559,106.43317039)(124.03299024,106.3827137)(124.11825638,106.34559874)
\curveto(124.2020863,106.30893732)(124.28780598,106.28712945)(124.37337449,106.27666016)
\curveto(124.5396737,106.25625071)(124.70075717,106.28023181)(124.84518425,106.31921764)
\curveto(124.98627402,106.35731528)(125.1216189,106.41268535)(125.24211024,106.46605228)
\curveto(125.36150551,106.51892787)(125.47107402,106.57169008)(125.56110236,106.60922079)
\curveto(127.38362079,107.57220661)(128.94186331,108.81627591)(130.59836976,110.0422715)
\curveto(131.93146961,111.02413606)(133.16326677,112.05429165)(134.54762835,112.8826885)
\curveto(135.2009348,113.14574362)(135.8822022,113.35079811)(136.56816,113.5912766)
\curveto(137.26793197,113.83659817)(137.95734803,114.11402683)(138.59609575,114.51715956)
\curveto(138.76069417,114.59968176)(138.88806425,114.72267893)(138.97622551,114.86037846)
\curveto(139.02252472,114.9326846)(139.05858142,115.01023672)(139.08307276,115.09003389)
\curveto(139.10763969,115.17055294)(139.12132157,115.25543735)(139.12049008,115.34110035)
\curveto(139.11948472,115.42787074)(139.1034822,115.5173246)(139.06738772,115.60298381)
\curveto(139.03034835,115.69057436)(138.9738822,115.77068523)(138.89742236,115.83611641)
\curveto(138.82005543,115.90232617)(138.72900661,115.94787704)(138.63066331,115.97444334)
\curveto(138.55106646,115.99594885)(138.4675389,116.00483074)(138.38208378,116.00347011)
\curveto(133.96941732,117.29806639)(130.99416567,117.94121613)(127.4387263,119.38739452)
\curveto(126.15524409,119.88579137)(125.10262677,120.28982665)(123.97728756,120.61263987)
\curveto(122.53291087,121.02697096)(121.35795024,121.18426923)(120.30917291,121.18346041)
\curveto(119.88084478,121.18312781)(119.88134368,120.54063534)(120.30967181,120.54096794)
\curveto(121.29031181,120.5417246)(122.40524598,120.39518778)(123.80012598,119.99505524)
\curveto(124.89342236,119.68143383)(125.92022173,119.28782211)(127.20140598,118.79035956)
\curveto(130.80563528,117.32429216)(133.85482961,116.66387036)(138.25146331,115.37207735)
\curveto(138.26998299,115.36667263)(138.28850268,115.36164586)(138.30758929,115.35960491)
\curveto(138.32648693,115.35755565)(138.3459515,115.35850318)(138.36507591,115.3598812)
\curveto(138.41462551,115.36344718)(138.44569323,115.35876057)(138.46258772,115.35421191)
\curveto(138.47883969,115.34978986)(138.48072945,115.34672844)(138.47921764,115.34801348)
\curveto(138.47506016,115.35292687)(138.47745638,115.34685619)(138.47761134,115.33365128)
\curveto(138.4777663,115.32027175)(138.47562709,115.3014875)(138.46854047,115.27852687)
\curveto(138.46173732,115.25596309)(138.45039874,115.23136214)(138.43490268,115.20677254)
\curveto(138.4042885,115.15873474)(138.35780031,115.11504718)(138.30144756,115.08834104)
\curveto(138.28935307,115.08267175)(138.27801449,115.07628435)(138.26667591,115.06917883)
\curveto(137.68595528,114.70037254)(137.04587717,114.4396543)(136.35533858,114.19757065)
\curveto(135.70007811,113.96785474)(134.96060976,113.74333531)(134.28336378,113.46896088)
\curveto(134.26824567,113.46280025)(134.25350551,113.45550576)(134.23948346,113.44715301)
\curveto(132.81109039,112.59549052)(131.5143685,111.51505587)(130.21649764,110.55913776)
\curveto(128.53317165,109.31329587)(127.02784252,108.11096013)(125.28704504,107.18968894)
\curveto(125.19917102,107.15453934)(125.08843087,107.10079446)(124.98165921,107.05347477)
\curveto(124.87065449,107.00430312)(124.77076157,106.96465587)(124.67752063,106.93952202)
\curveto(124.58231433,106.91382123)(124.50865134,106.90739603)(124.45097575,106.91457713)
\curveto(124.42225134,106.91806186)(124.39730646,106.92478186)(124.37436472,106.93498658)
\curveto(124.35735685,106.94216769)(124.33997102,106.95199446)(124.3217537,106.96560076)
\curveto(124.23202772,107.08435351)(124.1633537,107.21520076)(124.10718992,107.35584076)
\curveto(124.04410961,107.51374942)(123.9969789,107.6835258)(123.95434583,107.86038879)
\curveto(123.90967181,108.04558564)(123.87021354,108.23701871)(123.82323402,108.42714028)
\curveto(123.77716157,108.61354658)(123.72436157,108.79677808)(123.65334425,108.97479761)
\curveto(123.64124976,109.06849209)(123.61399937,109.15477871)(123.57155528,109.23271257)
\curveto(123.51841512,109.33033776)(123.44788913,109.40313146)(123.37592693,109.45755666)
\curveto(123.30215055,109.51334249)(123.22467024,109.55140233)(123.15830173,109.57872831)
\curveto(123.08872063,109.60745272)(123.01838362,109.6292606)(122.9633537,109.64585272)
\curveto(122.89717417,109.66588422)(122.85476787,109.67760076)(122.80971591,109.693248)
\curveto(122.76436157,109.70874406)(122.73873638,109.72083855)(122.72369386,109.72953146)
\curveto(122.7233537,109.72348422)(122.72603717,109.71781493)(122.72747339,109.71290154)
\curveto(122.7266948,112.82372863)(122.74107969,116.27329776)(122.73881197,119.4060703)
\curveto(122.73695765,119.83346228)(122.09633041,119.83207846)(122.09632252,119.40468246)
\closepath
}
}
{
\newrgbcolor{curcolor}{0.10196079 0.10196079 0.10196079}
\pscustom[linestyle=none,fillstyle=solid,fillcolor=curcolor]
{
\newpath
\moveto(201.08362205,59.73674079)
\curveto(202.18614425,61.03639559)(203.78528882,62.78157732)(205.65849449,65.02180535)
\curveto(206.96096126,66.57947339)(208.51338709,68.5486715)(209.80920945,70.77887244)
\curveto(209.8167685,70.79210079)(209.82432756,70.80570709)(209.83113071,70.81946457)
\curveto(210.16663937,71.48937827)(210.54311055,72.2545663)(210.98018646,72.92169449)
\curveto(211.18515024,73.23452598)(211.40858079,73.53468472)(211.65796157,73.80916157)
\curveto(211.89146079,74.06616945)(212.14891465,74.30203843)(212.43840378,74.50669984)
\curveto(212.73513449,74.68694551)(212.98175244,74.91486992)(213.19239685,75.16116661)
\curveto(213.40193386,75.40615559)(213.57905764,75.6731452)(213.73648252,75.93529701)
\curveto(213.74668724,75.95192693)(213.75651402,75.9689348)(213.76634079,75.98552693)
\curveto(213.90727937,76.2233348)(214.02818646,76.44765732)(214.16374677,76.68933921)
\curveto(214.29130583,76.91679118)(214.41807118,77.13408378)(214.5508989,77.3222022)
\curveto(214.5645052,77.3414778)(214.57735559,77.36162268)(214.58910992,77.3822211)
\curveto(215.08790929,78.25515591)(215.68573984,79.41289323)(216.27288567,80.32573606)
\curveto(216.56050772,80.77287685)(216.87266646,81.20889449)(217.22259402,81.61206425)
\curveto(217.5571578,81.99754961)(217.9293052,82.3566274)(218.35453984,82.67113701)
\curveto(219.08582929,83.25435591)(219.69953386,83.91964346)(220.26870047,84.64958362)
\curveto(220.80726047,85.34026961)(221.28597543,86.06207244)(221.76960378,86.7928063)
\curveto(221.81885102,86.8728189)(221.85154394,86.93397165)(221.87932346,87.00026457)
\curveto(221.90691402,87.06583937)(221.93257701,87.14388661)(221.94731717,87.23210079)
\curveto(221.96205732,87.32114646)(221.9662148,87.42410079)(221.94694299,87.53468976)
\curveto(221.92728945,87.64694173)(221.88597921,87.75189921)(221.82671622,87.84498898)
\curveto(221.76960756,87.93475276)(221.7020674,88.00312441)(221.64189732,88.05210709)
\curveto(221.58316346,88.09991811)(221.52745323,88.13295118)(221.48652094,88.15472126)
\curveto(221.40212409,88.19958425)(221.35510677,88.21504252)(221.3557115,88.21477795)
\curveto(221.35601386,88.21464567)(221.36893984,88.20986457)(221.39894929,88.19247874)
\curveto(221.42502803,88.17736063)(221.48111622,88.14202205)(221.53981228,88.07954646)
\curveto(221.60296819,88.01230866)(221.66858079,87.91396535)(221.70233197,87.78436535)
\curveto(221.73559181,87.65627717)(221.72652094,87.54183307)(221.7061115,87.45785197)
\curveto(221.68721386,87.3792)(221.65826268,87.32522835)(221.64363591,87.30047244)
\curveto(221.62700598,87.27250394)(221.61869102,87.26343307)(221.61869102,87.26343307)
\curveto(221.61916346,87.26400378)(221.65308472,87.30126614)(221.71661858,87.35792126)
\curveto(221.75214614,87.38966929)(221.78453669,87.42451654)(221.81337449,87.46231181)
\curveto(222.81714142,88.77826016)(223.97474268,90.00305008)(225.30323906,91.07604661)
\curveto(225.31306583,91.08398362)(225.3228926,91.09229858)(225.33234142,91.1009915)
\curveto(225.93586016,91.64196661)(226.46747717,92.23916976)(226.96231937,92.84596913)
\curveto(227.4556611,93.45092787)(227.91523654,94.06827969)(228.37750299,94.66118551)
\curveto(228.83982614,95.25417071)(229.29994961,95.81691591)(229.79890772,96.32746205)
\curveto(230.28758929,96.82748976)(230.80804157,97.27160315)(231.39312378,97.63743874)
\curveto(231.49664504,97.69020094)(231.59736945,97.75184504)(231.68539465,97.80838677)
\curveto(231.78562772,97.87278992)(231.8913411,97.9449789)(231.98363717,98.0073789)
\curveto(232.08054425,98.07291591)(232.1673978,98.13093165)(232.24846866,98.18101039)
\curveto(232.33090016,98.23192063)(232.39193953,98.26487811)(232.43498835,98.28407811)
\curveto(232.45426394,98.29277102)(232.46598047,98.29655055)(232.46446866,98.29579465)
\curveto(232.46274142,98.29529197)(232.44708283,98.29012535)(232.41034583,98.28785764)
\curveto(232.37481827,98.28580535)(232.31279622,98.28729449)(232.23735685,98.31129071)
\curveto(232.15764661,98.33661354)(232.07997732,98.38204346)(232.01557417,98.44618205)
\curveto(231.95491276,98.5065789)(231.92078362,98.57022614)(231.90483402,98.61044031)
\curveto(231.88896,98.65069228)(231.88820409,98.66913638)(231.88820409,98.66615055)
\curveto(231.88887685,98.66010331)(231.89181354,98.57389228)(231.87006236,98.4321978)
\curveto(231.8609915,98.37384189)(231.85910173,98.3145789)(231.86401512,98.25573165)
\curveto(231.96700724,97.05561449)(232.11966236,95.8776)(232.18656,94.73832189)
\curveto(232.25489386,93.5744315)(232.23474898,92.46970205)(232.01655685,91.40097638)
\curveto(232.0025726,91.33611969)(231.97713638,91.27390866)(231.94402772,91.2167622)
\curveto(231.90963402,91.15753701)(231.86363717,91.09770709)(231.80082142,91.03326614)
\curveto(231.66464504,90.8935748)(231.48700724,90.76760315)(231.23971654,90.5968252)
\curveto(231.00988346,90.43808504)(230.71275969,90.23595591)(230.45850709,89.97954898)
\curveto(230.32542992,89.84533795)(230.19741732,89.69033953)(230.08856693,89.50869921)
\curveto(229.98462992,89.33529449)(229.90182047,89.14374803)(229.84713071,88.93186394)
\curveto(229.3377978,87.47371087)(229.05160819,85.9874948)(228.85412031,84.57018331)
\curveto(228.65304945,83.12727307)(228.51709984,81.51248504)(228.3756737,80.2009852)
\curveto(228.11511307,77.00363339)(228.05815559,73.93153512)(227.84550047,71.04070677)
\curveto(227.60807055,67.81340598)(227.20101165,64.94581795)(226.35223181,62.21489764)
\curveto(225.99223181,61.11227717)(225.61320189,60.28339654)(225.27559559,59.65640315)
\curveto(224.80794453,58.78418062)(226.11425357,58.08079447)(226.58500535,58.9513474)
\curveto(226.9636611,59.65457386)(227.37792756,60.56485417)(227.76918425,61.76345953)
\curveto(228.66841701,64.65645354)(229.08719244,67.64927244)(229.3286589,70.93159559)
\curveto(229.54515024,73.87446047)(229.60263685,76.94766614)(229.85610331,80.06090835)
\curveto(230.00161512,81.40780346)(230.12996787,82.95049323)(230.3270589,84.3649474)
\curveto(230.51917228,85.74384378)(230.79109039,87.13377638)(231.2617852,88.47190299)
\curveto(231.27047811,88.49646992)(231.27765921,88.52137701)(231.28370646,88.54666205)
\curveto(231.30147024,88.62058961)(231.32834268,88.68450142)(231.36405921,88.74410457)
\curveto(231.40034268,88.80435024)(231.44879622,88.86618331)(231.51448441,88.93243843)
\curveto(231.65421354,89.07333921)(231.83154898,89.1982526)(232.08481512,89.37316157)
\curveto(232.30958362,89.52842457)(232.61124661,89.73418205)(232.86567685,89.99518488)
\curveto(232.99671307,90.12962268)(233.1237052,90.28643528)(233.2308926,90.47143937)
\curveto(233.33815559,90.65663622)(233.42172094,90.8635389)(233.47179969,91.0949178)
\curveto(233.72782488,92.34854173)(233.74332094,93.59593323)(233.67113197,94.82553449)
\curveto(233.60181543,96.00576)(233.45229732,97.16969197)(233.35293354,98.29951748)
\curveto(233.37636661,98.48320252)(233.38505953,98.66427969)(233.36578394,98.83392378)
\curveto(233.35444535,98.93529071)(233.33139024,99.04705134)(233.28671622,99.1593789)
\curveto(233.24147528,99.27333165)(233.16992882,99.39518362)(233.06470677,99.49995213)
\curveto(232.95578079,99.60842457)(232.82640756,99.68435528)(232.68879496,99.7281978)
\curveto(232.55549102,99.77067969)(232.42955717,99.7785411)(232.32459969,99.77253165)
\curveto(232.22039811,99.76648441)(232.12643906,99.74645291)(232.04801386,99.7233978)
\curveto(231.96887055,99.70034268)(231.89520756,99.67154268)(231.82948157,99.64225134)
\curveto(231.70010835,99.58457575)(231.57553512,99.51321827)(231.4669115,99.44613165)
\curveto(231.35704063,99.37825134)(231.24758551,99.30481512)(231.15041386,99.23908913)
\curveto(231.04874457,99.17033953)(230.96279811,99.11164346)(230.88150047,99.05941039)
\curveto(230.79910677,99.00649701)(230.74294299,98.97425764)(230.70654614,98.95668283)
\curveto(230.68349102,98.94534425)(230.66073827,98.93287181)(230.63881701,98.91964346)
\curveto(229.91573291,98.47223811)(229.29367559,97.93828535)(228.73520504,97.36684346)
\curveto(228.17753953,96.79622929)(227.67705071,96.18156094)(227.20455307,95.57552882)
\curveto(226.72874457,94.9652485)(226.27537512,94.35683906)(225.80968063,93.78578268)
\curveto(225.34451906,93.21538016)(224.87179087,92.68723276)(224.3534022,92.22074079)
\curveto(222.95666646,91.09263496)(221.73107528,89.7986948)(220.67235024,88.41866457)
\curveto(220.60008567,88.35180472)(220.53322583,88.28415118)(220.47611717,88.21566614)
\curveto(220.43945575,88.17182362)(220.39961953,88.11883465)(220.36333606,88.05745512)
\curveto(220.3278085,87.99717165)(220.28574236,87.91235906)(220.26023055,87.80672126)
\curveto(220.23339591,87.69564094)(220.22432504,87.55795276)(220.26287244,87.40954583)
\curveto(220.30195276,87.25953638)(220.37811024,87.14373165)(220.45574173,87.06111118)
\curveto(220.52898898,86.98313953)(220.60348346,86.93449701)(220.65065197,86.90705764)
\curveto(220.69868976,86.87908913)(220.7416252,86.86000252)(220.76365984,86.85044031)
\curveto(220.81449449,86.82851906)(220.82277165,86.82322772)(220.78822677,86.8417474)
\curveto(220.77235276,86.85006236)(220.74030236,86.86858205)(220.70258268,86.89912063)
\curveto(220.66354016,86.93086866)(220.6144063,86.97973795)(220.57173543,87.04682457)
\curveto(220.52702362,87.11712378)(220.49622047,87.19604031)(220.48182047,87.27820724)
\curveto(220.46783622,87.3585978)(220.4719937,87.42727181)(220.48018772,87.47712378)
\curveto(220.48850268,87.52595528)(220.50097512,87.55970646)(220.50740031,87.57527811)
\curveto(220.51458142,87.59228598)(220.51722709,87.59530961)(220.51344756,87.58926236)
\curveto(220.04112,86.87663622)(219.59303811,86.20237228)(219.0955011,85.56429732)
\curveto(218.57452724,84.89616378)(218.0408315,84.32362583)(217.44800126,83.85030425)
\curveto(216.94236472,83.47696252)(216.49178835,83.03961071)(216.09905008,82.58711811)
\curveto(215.68942866,82.11517606)(215.33510173,81.61774866)(215.02169197,81.13048819)
\curveto(214.41739465,80.19098079)(213.75907276,78.93092409)(213.31425638,78.14974488)
\curveto(213.14262803,77.90290394)(212.99156031,77.64048756)(212.8662463,77.41708724)
\curveto(212.7158589,77.14896756)(212.61555024,76.96166551)(212.48655496,76.74399874)
\curveto(212.47786205,76.72963654)(212.46954709,76.71527433)(212.46085417,76.7011389)
\curveto(212.32513134,76.47512315)(212.19617386,76.28543622)(212.06152441,76.12799622)
\curveto(211.92444094,75.96770646)(211.79049449,75.85175055)(211.65350929,75.77067969)
\curveto(211.63725732,75.76123087)(211.62176126,75.75102614)(211.60641638,75.74044346)
\curveto(211.20437669,75.45966236)(210.85834205,75.1416378)(210.55658835,74.80951937)
\curveto(210.2425852,74.46388157)(209.9721978,74.09818961)(209.73555402,73.73698772)
\curveto(209.24638866,72.99034961)(208.83234898,72.14819528)(208.51112693,71.50647307)
\curveto(207.28111748,69.39368693)(205.79754331,67.50760063)(204.5169411,65.97608315)
\curveto(202.67865071,63.77761134)(201.06905197,62.01956409)(199.94888315,60.69910677)
\curveto(199.30751274,59.94306049)(200.44158266,58.98100442)(201.08295307,59.73705071)
\closepath
}
}
{
\newrgbcolor{curcolor}{0.10196079 0.10196079 0.10196079}
\pscustom[linestyle=none,fillstyle=solid,fillcolor=curcolor]
{
\newpath
\moveto(211.71608693,74.01877795)
\lineto(221.68740283,74.01877795)
\lineto(229.66393701,74.01877795)
\curveto(230.65538225,74.01877795)(230.65538225,75.50594646)(229.66393701,75.50594646)
\lineto(221.68740283,75.50594646)
\lineto(211.71608693,75.50594646)
\curveto(210.72464169,75.50594646)(210.72464169,74.01877795)(211.71608693,74.01877795)
\closepath
}
}
{
\newrgbcolor{curcolor}{0.10196079 0.10196079 0.10196079}
\pscustom[linestyle=none,fillstyle=solid,fillcolor=curcolor]
{
\newpath
\moveto(244.00762961,59.59557543)
\curveto(244.16474457,60.05319685)(244.28296819,60.60969071)(244.29540283,61.3017978)
\curveto(244.3044737,61.80746835)(244.25722961,62.33046047)(244.16353512,62.92192252)
\curveto(244.08185953,63.43754835)(243.98703118,63.88219465)(243.88052409,64.43856756)
\curveto(243.79310362,64.89512693)(243.70870677,65.37660094)(243.67068472,65.84526614)
\curveto(243.66919181,65.86378583)(243.66701102,65.88192756)(243.66425953,65.90025827)
\curveto(243.62453669,66.15375118)(243.60348472,66.43199244)(243.61459654,66.71526803)
\curveto(243.62177764,66.8978948)(243.64218709,67.08097512)(243.67911307,67.25817071)
\curveto(243.69460913,67.3326652)(243.71312882,67.40776441)(243.73576819,67.48244787)
\curveto(243.80043591,67.69761638)(243.8957178,67.90796598)(244.03314142,68.09090646)
\curveto(244.1484548,68.24443087)(244.29959811,68.38718362)(244.50444472,68.49951496)
\curveto(244.70014866,68.59256693)(244.87971402,68.70663307)(245.04182551,68.84284724)
\curveto(245.20298457,68.97826772)(245.34074835,69.1303937)(245.45817449,69.29586898)
\curveto(245.68018394,69.60881386)(245.81700283,69.95184)(245.91387969,70.28859969)
\curveto(246.00390803,70.60162016)(246.06290646,70.92295559)(246.1201285,71.21059276)
\curveto(246.17421354,71.48256756)(246.22565291,71.71854614)(246.29719937,71.92451528)
\curveto(246.44641512,72.26104441)(246.6321411,72.59324598)(246.84920315,72.94555465)
\curveto(247.06176378,73.29058772)(247.30036913,73.6470274)(247.54231937,74.02717606)
\curveto(247.77442016,74.39186268)(248.01642331,74.79019087)(248.22557858,75.21595843)
\curveto(248.4270652,75.62613165)(248.60492598,76.07498079)(248.7206022,76.56898016)
\curveto(249.03037228,77.59988031)(249.2622463,78.66001512)(249.59366929,79.63991055)
\curveto(249.75630236,80.12079496)(249.94195276,80.58004157)(250.17101102,81.00869669)
\curveto(250.39619528,81.43012157)(250.66368378,81.82259528)(250.99400315,82.17796913)
\curveto(251.55131339,82.76100661)(252.00647811,83.4119811)(252.40565669,84.07882583)
\curveto(252.79880315,84.73560189)(253.14405543,85.4191937)(253.48133669,86.07305575)
\curveto(253.81922646,86.72806299)(254.14991622,87.35501102)(254.52756283,87.93914457)
\curveto(254.90116913,88.51703811)(255.31455496,89.04294425)(255.81222425,89.49754205)
\curveto(256.71377008,90.32317228)(257.70034394,91.05672945)(258.73517858,91.78747465)
\curveto(259.74631181,92.50148031)(260.81793638,93.22230047)(261.83470866,93.99025134)
\curveto(262.70982047,94.53946205)(263.55282898,95.07689575)(264.43706835,95.49446551)
\curveto(264.88185827,95.70449386)(265.3253178,95.87817827)(265.77216756,96.00265701)
\curveto(266.22165165,96.12787276)(266.67252283,96.2027452)(267.12956976,96.21536882)
\curveto(267.14733354,96.21585638)(267.16471937,96.21697512)(267.18248315,96.21872126)
\curveto(267.27750047,96.22817008)(267.33990047,96.22089071)(267.3796989,96.21154016)
\curveto(267.41847685,96.20246929)(267.44984693,96.18848504)(267.48038551,96.16822677)
\curveto(267.5132674,96.14668346)(267.55223433,96.11319685)(267.59917606,96.05941417)
\curveto(267.64725165,96.00438425)(267.69816189,95.93431181)(267.75625323,95.84367874)
\curveto(267.8716422,95.66369764)(267.99912567,95.42035654)(268.17174425,95.15977323)
\curveto(268.25844661,95.02888819)(268.35981354,94.89051969)(268.48083402,94.75822866)
\curveto(268.59096945,94.63785071)(268.71883087,94.52091213)(268.86898016,94.4178822)
\curveto(269.41229858,93.94025953)(270.00722646,93.54388913)(270.58422803,93.19131969)
\curveto(271.16313071,92.8375937)(271.77530835,92.49894047)(272.35421102,92.14521071)
\curveto(272.93070236,91.79322331)(273.44493354,91.44345071)(273.89156409,91.04582173)
\curveto(274.33789606,90.64845732)(274.70550047,90.21262866)(274.98475087,89.70209386)
\curveto(275.53034457,88.35932598)(275.83750677,86.95625953)(276.0393222,85.5951874)
\curveto(276.25781669,84.12176126)(276.3713915,82.53567118)(276.48309921,81.18941102)
\curveto(276.94868409,73.53929197)(276.33240945,58.35496819)(276.30289134,55.04616945)
\curveto(276.29407244,54.05476452)(277.78118048,54.04153618)(277.78999937,55.0329411)
\curveto(277.81834583,58.19255811)(278.43676724,73.56986079)(277.96635213,81.29613354)
\curveto(277.85583874,82.63014047)(277.73784189,84.27955276)(277.5104126,85.81336441)
\curveto(277.29717165,87.25144441)(276.96323528,88.79656063)(276.34596283,90.3025663)
\curveto(276.33613606,90.32637732)(276.32517543,90.34939465)(276.31308094,90.37192063)
\curveto(275.93485228,91.07735055)(275.44171087,91.65677102)(274.88028094,92.15660598)
\curveto(274.33771465,92.63964094)(273.73791874,93.0427389)(273.12903685,93.41451213)
\curveto(272.54170583,93.7731137)(271.94679685,94.10175496)(271.35946583,94.46036787)
\curveto(270.79875024,94.80298205)(270.27516472,95.15658709)(269.81797417,95.56381228)
\curveto(269.78887181,95.58989102)(269.75769071,95.61339969)(269.72484661,95.63445165)
\curveto(269.67805606,95.66430992)(269.63035843,95.70478866)(269.57789858,95.76212409)
\curveto(269.52509858,95.81983748)(269.47086236,95.89130835)(269.4113726,95.98110992)
\curveto(269.29526551,96.15640441)(269.17239307,96.39002457)(269.0079874,96.64644283)
\curveto(268.92774803,96.7715452)(268.8327685,96.90757039)(268.71911811,97.03768441)
\curveto(268.60531654,97.16800252)(268.46770394,97.29843402)(268.29909165,97.40962772)
\curveto(268.12942866,97.52157732)(267.93603024,97.6089222)(267.71676472,97.65998362)
\curveto(267.50839937,97.70851276)(267.2902526,97.72109858)(267.06337134,97.70118047)
\curveto(266.47856882,97.68266079)(265.91531339,97.58635843)(265.37287559,97.43525291)
\curveto(264.82758047,97.2833537)(264.30428598,97.07650016)(263.80182425,96.8392063)
\curveto(262.81083969,96.37122898)(261.87657449,95.77223055)(261.01670551,95.23277102)
\curveto(260.99818583,95.22105449)(260.98042205,95.20896)(260.96314961,95.19573165)
\curveto(259.96413732,94.43969386)(258.93540283,93.74947654)(257.87712756,93.00218079)
\curveto(256.82825953,92.26152567)(255.78000756,91.48468913)(254.80831748,90.59482205)
\curveto(254.19175559,90.03161953)(253.69957795,89.39782677)(253.27845921,88.74645921)
\curveto(252.85710614,88.09472504)(252.49705323,87.40922079)(252.15943937,86.75471622)
\curveto(251.81255433,86.08220976)(251.49256063,85.44918425)(251.12942362,84.8425474)
\curveto(250.76594646,84.2353663)(250.37532472,83.68308661)(249.91159181,83.19785953)
\curveto(249.47350677,82.72665827)(249.13466079,82.22505827)(248.85915591,81.70948535)
\curveto(248.58034016,81.18772157)(248.36451402,80.64795213)(248.18469165,80.1162822)
\curveto(247.8138822,79.01995087)(247.58012976,77.93787591)(247.28956724,76.97503748)
\curveto(247.28465386,76.95878551)(247.28049638,76.94253354)(247.27671685,76.92628157)
\curveto(247.19296252,76.56072567)(247.05852472,76.2133115)(246.89063811,75.87155528)
\curveto(246.71764913,75.51941669)(246.51012283,75.1752378)(246.28756913,74.82556346)
\curveto(246.05841638,74.46552567)(245.81506772,74.10234331)(245.5828989,73.72549417)
\curveto(245.34168945,73.33395402)(245.11355339,72.92928378)(244.92418772,72.49689071)
\curveto(244.91738457,72.48139465)(244.91095937,72.46552063)(244.90529008,72.4497222)
\curveto(244.78793575,72.12264189)(244.71688063,71.78004283)(244.66128378,71.50061858)
\curveto(244.59756094,71.18030362)(244.55269795,70.93707213)(244.48440189,70.69957795)
\curveto(244.41500976,70.45833071)(244.33730268,70.28637354)(244.24496882,70.15625197)
\curveto(244.19719559,70.08890079)(244.14443339,70.03137638)(244.08486803,69.98133543)
\curveto(244.02261921,69.92902677)(243.94789795,69.88068661)(243.85563969,69.83782677)
\curveto(243.84316724,69.83215748)(243.8306948,69.82573228)(243.81860031,69.81930709)
\curveto(243.39540661,69.59321575)(243.07849323,69.29630362)(242.84389795,68.9839748)
\curveto(242.57747906,68.62926614)(242.4136063,68.25044787)(242.31141543,67.91050583)
\curveto(242.27550992,67.79095937)(242.24640756,67.67375622)(242.22293669,67.56074835)
\curveto(242.16722646,67.29274205)(242.13838866,67.02702992)(242.1284485,66.77357858)
\curveto(242.11333039,66.39211843)(242.14054299,66.02634331)(242.19062173,65.69760756)
\curveto(242.23654299,65.15300031)(242.3324674,64.61529449)(242.41985008,64.1588863)
\curveto(242.53278236,63.56888315)(242.61793512,63.17345008)(242.69465953,62.68916787)
\curveto(242.77935874,62.1545915)(242.81552882,61.72224756)(242.80842331,61.32842457)
\curveto(242.79897449,60.7984063)(242.70989102,60.39561449)(242.60100283,60.07840252)
\curveto(242.28210902,59.14216603)(243.68417895,58.66082729)(244.00758425,59.59551496)
\closepath
}
}
{
\newrgbcolor{curcolor}{0.10196079 0.10196079 0.10196079}
\pscustom[linestyle=none,fillstyle=solid,fillcolor=curcolor]
{
\newpath
\moveto(291.71579717,95.40184441)
\curveto(291.63907276,95.19567118)(291.58392945,94.97354835)(291.55505386,94.73186646)
\curveto(291.53048693,94.52467276)(291.5259515,94.3123389)(291.53917984,94.09095307)
\curveto(291.56223496,93.70529386)(291.63461291,93.33817701)(291.71798929,92.93812913)
\curveto(291.78700346,92.60711811)(291.85975937,92.26952693)(291.88666961,91.95407622)
\curveto(291.89914205,91.80939591)(291.90178772,91.66875969)(291.89082709,91.53250016)
\curveto(291.8859137,91.47032693)(291.87797669,91.40868283)(291.86701606,91.34749228)
\curveto(291.70759559,91.23558047)(291.53453102,91.11573165)(291.35349543,90.98276787)
\curveto(291.12105449,90.81204661)(290.86067528,90.60761197)(290.62250835,90.36704126)
\curveto(290.39717291,90.13940031)(290.17539402,89.86359307)(290.0065285,89.52564283)
\curveto(289.85289071,89.21817827)(289.74611906,88.86549921)(289.71524031,88.45916598)
\curveto(289.35380409,86.92931528)(289.18265953,85.38995906)(289.0297285,83.74810961)
\curveto(289.02443717,83.6921726)(288.98195528,83.22733606)(288.97673953,83.17138394)
\curveto(288.83908913,81.66215433)(288.67793008,79.76205354)(288.26694803,78.00824315)
\curveto(287.71960819,75.2289789)(287.01316535,72.25684535)(286.44100157,69.05252787)
\curveto(285.93576189,66.22298835)(285.52728567,63.09847181)(285.6206211,59.87270551)
\curveto(285.61608567,58.65220157)(285.64027465,57.06698457)(285.58052031,55.67538898)
\curveto(285.57754205,55.60622362)(285.57744,55.53588661)(285.59601638,55.46925354)
\curveto(285.61491402,55.40258268)(285.65149984,55.34263937)(285.69008882,55.28515276)
\curveto(285.86356913,55.02670866)(286.08629669,54.82272)(286.31852976,54.6559937)
\curveto(286.54443213,54.49381417)(286.79040378,54.35911181)(287.0284422,54.24060472)
\curveto(287.26337764,54.12362835)(287.4974778,54.01878425)(287.71357228,53.91575433)
\curveto(287.92840063,53.81332913)(288.11643969,53.71714016)(288.27532346,53.61577323)
\curveto(288.28628409,53.60859213)(288.29762268,53.60216693)(288.30896126,53.59574173)
\curveto(288.7209222,53.36674016)(289.15309984,53.20337008)(289.5972737,53.09633764)
\curveto(290.03348031,52.99122898)(290.47830425,52.94115024)(290.92383874,52.93491402)
\curveto(291.79396535,52.92281953)(292.66724409,53.07804472)(293.48839937,53.31769701)
\curveto(296.19056882,53.97001701)(299.43801071,54.54076724)(302.33049449,55.5174652)
\curveto(306.03066709,56.67720945)(309.57410646,58.47031181)(312.8566148,60.74727685)
\curveto(313.2888189,61.00621228)(313.75655811,61.23859276)(314.2553537,61.47880441)
\curveto(314.74847622,61.71627213)(315.28072441,61.96541102)(315.78901039,62.23959685)
\curveto(316.29888378,62.51463307)(316.80886677,62.82729071)(317.27722961,63.20893606)
\curveto(317.74556976,63.5905663)(318.17480315,64.04311559)(318.5243263,64.60004409)
\curveto(318.55078299,64.64218583)(318.5723263,64.68727559)(318.58850268,64.73433071)
\curveto(318.80635465,65.36828976)(319.01286047,66.05354835)(319.13928567,66.76040315)
\curveto(319.26525732,67.46473323)(319.31442898,68.20454929)(319.21037858,68.95133102)
\curveto(319.20244157,69.00726803)(319.19299276,69.06346961)(319.17095811,69.1155137)
\curveto(319.14903685,69.1675578)(319.11577701,69.2138948)(319.08130772,69.25860661)
\curveto(318.89724472,69.49732157)(318.68537575,69.69636283)(318.46873701,69.86797984)
\curveto(318.25330394,70.03866331)(318.0260863,70.18757669)(317.80961764,70.3219389)
\curveto(317.59146331,70.45735937)(317.37860409,70.58178142)(317.1820422,70.70476346)
\curveto(316.9836926,70.82884535)(316.80699969,70.94801386)(316.65137008,71.0740989)
\curveto(316.63549606,71.08694929)(316.61886614,71.09942173)(316.60148031,71.11076031)
\curveto(315.11778898,72.09484724)(313.58783244,73.13429669)(311.97928063,74.0148737)
\curveto(310.3601537,74.90124472)(308.64408189,75.63692598)(306.77459528,76.01083843)
\curveto(305.61229606,76.27052976)(304.4331515,76.58096126)(303.23700661,76.85522646)
\curveto(302.0308422,77.13177449)(300.80708787,77.37144945)(299.5517178,77.48650583)
\curveto(298.79828787,77.57483339)(298.03495181,77.71063181)(297.27870236,77.8789115)
\curveto(300.3225411,78.15666898)(303.31614992,78.57803717)(306.24394583,79.32015496)
\curveto(308.13903874,79.78899402)(310.00431118,80.19255307)(311.88579402,80.72403024)
\curveto(313.7884611,81.26149417)(315.62937827,81.91005732)(317.36319874,82.83110173)
\curveto(317.73011528,82.96591748)(318.04629165,83.18792693)(318.29177953,83.44957228)
\curveto(318.55317165,83.72816126)(318.75056126,84.0687685)(318.84161008,84.43394646)
\curveto(318.88817386,84.62065512)(318.90775181,84.81754583)(318.89066835,85.01703307)
\curveto(318.87328252,85.21961575)(318.81829039,85.42129512)(318.71956913,85.61146205)
\curveto(318.61960063,85.80399118)(318.48024945,85.97433449)(318.3070337,86.11768063)
\curveto(318.14133921,86.25480189)(317.94903685,86.36372787)(317.73661228,86.44665071)
\curveto(317.33525291,86.64575622)(316.98320126,86.85233764)(316.68780472,87.09197858)
\curveto(316.53246614,87.21798803)(316.39712126,87.34981795)(316.28155465,87.48942236)
\curveto(316.16261291,87.6331578)(316.06415622,87.78569953)(315.98637354,87.94973858)
\curveto(315.97768063,87.96788031)(315.96823181,87.98602205)(315.95764913,88.00340787)
\curveto(315.60815622,88.59133228)(315.17353701,89.0708863)(314.70066142,89.46573354)
\curveto(314.21009764,89.8753474)(313.67303811,90.1985915)(313.13588031,90.46176378)
\curveto(312.0145474,91.01114835)(310.74934299,91.36104189)(309.77667402,91.63119496)
\curveto(306.37220031,92.50400882)(302.73999874,92.72113134)(299.82806173,92.92336252)
\curveto(295.96529008,93.19163339)(293.07860031,93.44799496)(290.14793575,94.24708157)
\curveto(289.31172824,94.47508778)(288.96971879,93.22077596)(289.8059263,92.99276976)
\curveto(292.87386709,92.15625449)(295.87688315,91.89453354)(299.73798803,91.62638362)
\curveto(302.68361197,91.4217978)(306.1872189,91.20929008)(309.44130142,90.37515969)
\curveto(310.42182425,90.10269354)(311.56882394,89.78177008)(312.56387528,89.29425638)
\curveto(313.03267276,89.06457449)(313.47580724,88.7947389)(313.86737386,88.46778331)
\curveto(314.23591559,88.16005417)(314.56306772,87.79829669)(314.82584693,87.3628611)
\curveto(314.95019339,87.10706268)(315.10333984,86.87376)(315.27988157,86.6604548)
\curveto(315.4572548,86.44615559)(315.65590677,86.25489638)(315.86859213,86.08233827)
\curveto(316.27650142,85.75140283)(316.73340094,85.49144693)(317.18841071,85.26730205)
\curveto(317.2076863,85.25785323)(317.2276422,85.24916031)(317.24786268,85.24160126)
\curveto(317.35773354,85.20048)(317.43060283,85.15531465)(317.47803591,85.11604535)
\curveto(317.52403276,85.07798551)(317.54992252,85.04249575)(317.56553197,85.0124485)
\curveto(317.58102803,84.98259024)(317.59161071,84.94781858)(317.59501228,84.90594142)
\curveto(317.59865197,84.86349732)(317.59536378,84.81077291)(317.57989417,84.7484863)
\curveto(317.54852409,84.62300598)(317.47127055,84.47541543)(317.34344693,84.3391748)
\curveto(317.21910047,84.20666457)(317.06179654,84.10110236)(316.89115465,84.0431622)
\curveto(316.85751685,84.03182362)(316.82497512,84.01746142)(316.79368063,84.00090709)
\curveto(315.16001008,83.12785512)(313.40144126,82.5032126)(311.53213606,81.97517102)
\curveto(309.67606299,81.45087118)(307.87675087,81.06345827)(305.92788283,80.58130394)
\curveto(301.9402885,79.57056756)(297.80417008,79.16612787)(293.50490457,78.89970142)
\curveto(293.32416756,78.88836283)(293.13783685,78.87173291)(292.97840882,78.78586205)
\curveto(292.89869858,78.74292661)(292.82726551,78.68377701)(292.77363402,78.61079433)
\curveto(292.72000252,78.53784945)(292.68466394,78.45137386)(292.67324976,78.36153449)
\curveto(292.66191118,78.27169512)(292.67445921,78.17913449)(292.70802142,78.09511559)
\curveto(292.74165921,78.01109669)(292.7961222,77.93592189)(292.8625663,77.87446677)
\curveto(292.99549228,77.75148094)(293.17173165,77.68870299)(293.34392693,77.63272819)
\curveto(295.26391937,77.0083578)(297.30364346,76.44069165)(299.41647496,76.19353701)
\curveto(300.60425575,76.08445984)(301.7636674,75.85912441)(302.94605102,75.58802646)
\curveto(304.11518362,75.31994457)(305.29648252,75.00886677)(306.50505449,74.73900094)
\curveto(308.21745638,74.39634898)(309.80623748,73.72211906)(311.35460031,72.87449197)
\curveto(312.89450457,72.03149102)(314.36215937,71.03596346)(315.85766173,70.04379213)
\curveto(316.06519559,69.87809764)(316.28324031,69.73334173)(316.49208945,69.60267591)
\curveto(316.71099969,69.46570583)(316.92517795,69.34056567)(317.12354646,69.21742866)
\curveto(317.32382362,69.09312)(317.50270488,68.97440504)(317.66105197,68.84896252)
\curveto(317.76793701,68.76426331)(317.86200945,68.67918614)(317.94372283,68.59142551)
\curveto(317.99459528,68.07162709)(317.95695118,67.5355389)(317.85913701,66.98939717)
\curveto(317.75383937,66.40072441)(317.58103937,65.81111433)(317.38360063,65.23005354)
\curveto(317.12889827,64.84021039)(316.81538646,64.5100422)(316.45562835,64.21689449)
\curveto(316.07368063,63.90565039)(315.64082646,63.63713008)(315.17143559,63.38392441)
\curveto(314.69704063,63.1280126)(314.20293921,62.89682268)(313.69091528,62.65024252)
\curveto(313.18252346,62.40540472)(312.66123591,62.14756913)(312.16787906,61.85048315)
\curveto(312.15578457,61.84330205)(312.14406803,61.83574299)(312.1323515,61.82742803)
\curveto(308.94155339,59.61175559)(305.50976126,57.87651024)(301.92765354,56.7536126)
\curveto(299.0911748,55.79598992)(296.0059389,55.26300472)(293.1531326,54.57354331)
\curveto(292.3903937,54.35168504)(291.64926992,54.22499528)(290.94152315,54.23485984)
\curveto(290.5811074,54.23977323)(290.2332926,54.28025197)(289.90141228,54.36022677)
\curveto(289.5700989,54.44005039)(289.25419087,54.55944567)(288.95595213,54.72332598)
\curveto(288.73250646,54.86418898)(288.49267654,54.98437795)(288.27250016,55.08937323)
\curveto(288.03741354,55.20143622)(287.81046425,55.30337008)(287.6073411,55.40447244)
\curveto(287.39629228,55.50954331)(287.22055181,55.60849134)(287.0761663,55.71216378)
\curveto(287.00148283,55.76579528)(286.93957417,55.8176126)(286.88847496,55.86795591)
\curveto(286.93738205,57.23303811)(286.91606551,58.75623685)(286.91984504,59.88937701)
\curveto(286.82996787,63.00769134)(287.2217915,66.03256441)(287.72025827,68.82408945)
\curveto(288.2833852,71.97780661)(288.9886715,74.94771024)(289.53706583,77.73450331)
\curveto(289.96798488,79.57157291)(290.1355389,81.56959748)(290.27086488,83.05339087)
\curveto(290.25839244,82.91543811)(290.33621291,83.76557858)(290.32362709,83.62762583)
\curveto(290.47771843,85.28211402)(290.64473575,86.75936504)(290.99184378,88.2105222)
\curveto(291.00091465,88.24793953)(291.00658394,88.28645291)(291.0084737,88.32504189)
\curveto(291.02245795,88.57384819)(291.08357291,88.77415559)(291.16876346,88.94460094)
\curveto(291.26090835,89.12900409)(291.38880756,89.29401827)(291.54565795,89.45242961)
\curveto(291.71373354,89.62220598)(291.91151622,89.78011465)(292.1224252,89.93506772)
\curveto(292.34927244,90.10170709)(292.58811969,90.26286614)(292.81829291,90.4291011)
\curveto(292.87487244,90.4699578)(292.93058268,90.51383811)(292.97143937,90.57045543)
\curveto(293.01233386,90.62703496)(293.03648504,90.69370583)(293.05753701,90.76026331)
\curveto(293.13059528,90.99145701)(293.16925984,91.2163389)(293.18622992,91.42951937)
\curveto(293.2039937,91.65296504)(293.19832441,91.86698457)(293.18131654,92.06465386)
\curveto(293.14616693,92.47549606)(293.05334173,92.89971402)(292.98995906,93.20359559)
\curveto(292.90624252,93.60522331)(292.85291339,93.88820787)(292.83620787,94.16853165)
\curveto(292.82713701,94.32307654)(292.83091654,94.45713638)(292.84527874,94.57777512)
\curveto(292.86190866,94.71739087)(292.8928252,94.83909165)(292.9335685,94.94854677)
\curveto(293.23583987,95.76086768)(292.01735798,96.21427492)(291.71508661,95.40195402)
\closepath
}
}
{
\newrgbcolor{curcolor}{0.10196079 0.10196079 0.10196079}
\pscustom[linestyle=none,fillstyle=solid,fillcolor=curcolor]
{
\newpath
\moveto(128.67892535,115.90810696)
\curveto(128.8249285,116.19963326)(128.92002142,116.53334249)(128.95460409,116.91971414)
\curveto(128.98143874,117.22142249)(128.97010016,117.53215106)(128.93381669,117.85888365)
\curveto(128.8753474,118.38818041)(128.74714583,118.95377575)(128.68614425,119.44312403)
\curveto(128.65893165,119.65997443)(128.64808441,119.84605757)(128.65817575,120.00229946)
\curveto(128.66762457,120.1470138)(128.69370331,120.2416834)(128.72390173,120.30450293)
\curveto(128.73788598,120.33375647)(128.75451591,120.35941191)(128.77469858,120.38344214)
\curveto(128.79435213,120.40683742)(128.8225474,120.43423899)(128.86457575,120.46407458)
\curveto(128.94606236,120.52190513)(129.0953537,120.59956309)(129.36327307,120.66259427)
\curveto(129.45647622,120.68451553)(129.54578646,120.7206138)(129.62802898,120.76958891)
\curveto(130.8081411,121.47224769)(132.0398211,122.25375383)(133.28122961,123.11833852)
\curveto(135.23253921,124.4773372)(137.78751874,126.44131049)(140.2524548,129.03479357)
\curveto(143.86390299,132.06965216)(148.21044661,134.44884624)(153.31840252,136.2171462)
\curveto(157.91430425,137.80818028)(162.89572157,138.82672139)(168.15789732,139.46786419)
\curveto(172.26497008,139.88445165)(175.74341669,140.10203376)(180.2863748,140.47982589)
\curveto(180.94527118,140.48310312)(181.58453669,140.35361235)(182.24186079,140.13276321)
\curveto(182.91099591,139.90794557)(183.56928378,139.59953877)(184.28698961,139.25394784)
\curveto(184.98058205,138.91996611)(185.73762142,138.54710854)(186.52501417,138.25048365)
\curveto(187.29846425,137.95911231)(188.13403465,137.72756258)(189.04417512,137.65785109)
\curveto(193.46738268,136.69672554)(197.72175118,135.76754608)(201.86255622,134.36653266)
\curveto(201.88598929,134.35859565)(201.90968693,134.35160353)(201.93364913,134.34555628)
\curveto(203.91499087,133.84569109)(205.76234079,133.43757883)(207.55285795,132.79023383)
\curveto(208.45302803,132.46478627)(209.29751811,132.09183609)(210.08040945,131.64262715)
\curveto(210.88160504,131.18291414)(211.61145827,130.64715099)(212.26423559,130.0095035)
\curveto(212.27897575,129.9951035)(212.29409386,129.98119483)(212.30989228,129.96777751)
\curveto(220.70808945,122.78338696)(227.28262677,114.77953398)(232.2689726,107.11141153)
\curveto(232.56759307,106.63421216)(232.8042822,106.24042318)(232.96549039,105.96837657)
\curveto(233.60278588,104.89539344)(235.21268748,105.84937258)(234.57757984,106.92365216)
\curveto(234.41026016,107.20598287)(234.16552819,107.61314381)(233.84856567,108.11938507)
\curveto(228.77426646,115.92333128)(222.09254551,124.06084422)(213.55127811,131.37198009)
\curveto(212.77843276,132.12286677)(211.92599433,132.74422715)(211.01320063,133.26797102)
\curveto(210.11493165,133.78338066)(209.16695433,134.19934602)(208.19018835,134.55248617)
\curveto(206.29685669,135.23700397)(204.32538709,135.67555994)(202.42798866,136.15350539)
\curveto(198.16187339,137.59401071)(193.79477291,138.54326627)(189.3770948,139.50315515)
\curveto(189.33219402,139.51290633)(189.28661291,139.51936932)(189.24076724,139.52243074)
\curveto(188.54088567,139.56930671)(187.86778205,139.74714935)(187.18583055,140.00405178)
\curveto(186.49352315,140.26485506)(185.82123969,140.59507049)(185.10017764,140.94228019)
\curveto(184.38700724,141.28568769)(183.63227717,141.64248907)(182.8388863,141.90905461)
\curveto(182.02957228,142.18097008)(181.16679685,142.3640572)(180.23297764,142.35330973)
\curveto(180.21067843,142.35305272)(180.18830362,142.35199635)(180.1660422,142.35014249)
\curveto(175.65016063,141.97423824)(172.1069178,141.75188636)(167.95021606,141.33007257)
\curveto(162.57798047,140.67571087)(157.46003528,139.63382476)(152.7056126,137.98791231)
\curveto(147.39644976,136.14995868)(142.82655874,133.65522406)(139.00535811,130.43423093)
\curveto(138.97852346,130.41170494)(138.95323843,130.3877314)(138.92920063,130.36237833)
\curveto(136.57011024,127.87338293)(134.10914268,125.97833688)(132.21051969,124.65603137)
\curveto(131.05161071,123.84890192)(129.89813291,123.11440781)(128.78578394,122.44901518)
\curveto(128.40087307,122.344224)(128.06490709,122.19425235)(127.78014614,121.99212321)
\curveto(127.61350677,121.87382778)(127.46799496,121.74015345)(127.34345575,121.59271294)
\curveto(127.21555654,121.44132775)(127.1142652,121.28079987)(127.03523528,121.11639194)
\curveto(126.87124157,120.77527446)(126.80823685,120.42852094)(126.78843213,120.12378293)
\curveto(126.76726677,119.79991899)(126.79296756,119.48241373)(126.82679433,119.21138003)
\curveto(126.90676913,118.56951269)(127.02809197,118.04505146)(127.07140535,117.65310652)
\curveto(127.09824,117.40878652)(127.10088567,117.22883981)(127.08841323,117.08681991)
\curveto(127.07291717,116.91449235)(127.0358022,116.81142841)(127.00360063,116.74715376)
\curveto(126.44523838,115.63047172)(128.12076342,114.79142341)(128.67912567,115.90810545)
\closepath
}
}
\end{pspicture}

\end{center}

\paragraph{Union}
The union is two or more events expresses the set of outcome that satify at
least one of the events. Graphically, is the area that includes both sets.
Noted: $A \cup B$, $A \cup B = A+B-A\cap B$.

\paragraph{Mutually exclusive Sets}
Sets with no overlapping elements are called mutually exclusive. Graphically,
their circles never touch.

\textbf{If $A \cap B = \emptyset$, then the two sets are mutually exclusive.}
Remember: All complements are mutually exclusive, but not all mutually
exclusive sets are complements.

Example: Dogs and cats are mutually exclusive sets, since no species is
simultaneously a feline and a canine, but the two are not complements, since
there exist other types of animals as well.


\subsubsection{Independent and Dependent Events}
If the likelihood of event A occurring (P(A)) is affected event B occurring,
then we say that A and B are \textbf{dependent} events.
Alternatively, if it isn't – the two events are \textbf{independent}.

We express the probability of event A occurring, given event B has occurred the
following way \textbf{P(A|B)}. We call this the conditional probability.

\textbf{Independent:}
\begin{itemize}
  \item All the probabilities we have examined so far.
  \item The outcome of A does not depend on the outcome of B.
  \item $P(A|B)=P(A)$
  \item \textbf{Example:}
  \item A -> Hearts
  \item B -> Jacks
\end{itemize}

\textbf{Dependent:}
\begin{itemize}
  \item New concept
  \item The outcome of A depends on the outcome of B.
  \item $P(A|B)\neq P(A)$
  \item \textbf{Example:}
  \item A -> Hearts
  \item B -> Red
\end{itemize}

\subsubsection{Conditional Probability}
For any two events A and B, such that the likelihood of B occuring is greater
than 0 (P(B)>0), the conditional probability formula states the following.
Also writable as: $$\forall A, B \quad P(B)>0:\quad P(A|B)=\frac{P(A\cap B)}{P(B)}.$$

\textbf{Intuition behind the formula:}
\begin{itemize}
  \item Only interest in the outcome where B is satisfied.
  \item Oly the elements in the intersectino would satisfly A as well.
  \item Paralll to the "favored over all" formula.
    \subitem Intersectino = "preferred outcomes"
    \subitem B = "sample space"
\end{itemize}

\textbf{Remember:}
\begin{itemize}
  \item Unlike the union or the intersection, changing the order of A and B in
    the conditional probability alters its meaning.
  \item P(A|B) is not the same as P(B|A), even if P(A|B)=P(B|A) numerically.
  \item the two conditional probabilities posess \textbf{different meanings}
    even if they have equal values.
\end{itemize}

\subsubsection{Law of total probability}
The \textbf{law of total probability} dictates that for any set A, which is a
union of many mutually exclusive sets $B_1, B_2, ..., B_n$ its probability
equals the following sum.
\[P(A)=P(A|B_1)\cdot P(B_1)+P(A|B_2)\cdot P(B_2) + \dots + P(A|B_n) \cdot P(B_n)\]
Meaning of the opperators:\\
$P(A|B_1)$= Conditional probability of A, given $B_1$ has occurred.\\
$P(B_1)\quad$= Probability of $B_1$ occurring.\\
$P(A|B_2)$= Conditional of A, given $B_2$ has occurred.\\
$P(B_2)\quad$= Probability of $B_2$ occuring.

\textbf{Intution behind the formula:}
\begin{itemize}
  \item Since P(A) is the union of mutually exclusive sets, so it equals to the
    sum of the individual sets.

  \item The \textbf{intersection} of a union and one of its subsets is the entire subset.

  \item We can rewrite the conditional probability formula 
    \subitem $P(A|B)=\frac{P(A \cap B)}{P(B)} \quad \text{to get} \quad P(A\cap B) \cdot P(B)$.

  \item Another way to express the law of probability is:
    \[P(A)=P(A \cap B_1)+P(A \cap B_2)+ \dots + P(A\cap B_n)\]
\end{itemize}

\subsubsection{Additive Law}
The additive law calculates the probability of the union based on the
probability of the individual sets it accounts for.
\[P(A\cup B) = P(A) + P(B) - P(A \cap B)\]
Meaning of the operators:\\
$P(A \cup B)$= Probability of the union.\\
$P(A \cap B)$= Probability of the intersection.

\textbf{Intution behind the formula:}
\begin{itemize}
  \item Recall the formula for finding the size of the uunion using the size of the intersction:
    \subitem $A\cup B = A+B - A \cap B$
  \item The probability of each oen is simply its size over the size of the sample space.
  \item this holds true for any events A and B.
\end{itemize}


\subsubsection{The Multiplication Rule}
The multiplication rule calculates the probability of the intersection based on the conditional probability.
\[ P(A\cap B) = P(A|B)\cdot P(B) \]

meaning of the operators:\\
$P(A\cap B)$= Probability of the intersection\\
$P(A|B)$= Conditional Probability\\
$P(B)$= Probability of event B

Intuition behind the formula:
\begin{itemize}
  \item We can multiply both sides of the conditional probability formula
    $P(A|B)=\frac{P(A\cap B)}{P(B)}$ by $P(B)$ to get $P(A\cap B)=P(A|B)\cdot
    P(B).$ \item If event B occurs in 40\% of the time ($P(B)=0.4$) and event A
    occurs
    in 50\% of the time of the time mB occurs ($P(A|B)=0.5$), then they would
    simantaneously occur 20\% of the time ($P(A|B)\cdot P(B)=0,5\cdot 0,4 =
    0,2$).
\end{itemize}

\subsubsection{Bayes' Law}
Bayes' Law helps us understand the relationship between two events by computing
the different conditional probabilities. We also call it Bayes' Rue or Bayes'
Theorem.
\[ P(A|B)\frac{P(B|A)\cdot P(A)}{P(B)} \]
Intuition behind the formula
\begin{itemize}
  \item According to the multiplication rule $P(A\cap B)= P(A|B) \cdot P(B)
    \text{, so } P(B \cap A) = P(B|A)\cdot P(A)$.
  \item Since $P(A \cap B) = P(B\cap A)$, we plug in $P(B|A) \cdot P(A)$ for
    $P(A\cap B)$ in the conditional probability formula $P(A|B)=\frac{P(A\cap
    B)}{P(B)}$
  \item Bayes' Law is often used in medical or business analysis to determine
    which of two symption affects the other one more.
\end{itemize}








\end{document}
