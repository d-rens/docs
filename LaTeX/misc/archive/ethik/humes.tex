\documentclass{article}

\usepackage[a4paper, left=3cm, right=3cm, top=3cm, bottom=3cm]{geometry}
\usepackage{tcolorbox}
\usepackage[ngerman]{babel}

\usepackage{csquotes}
\setlength{\parindent}{0pt}

\usepackage{biblatex}
\bibliography{humes.bib}


\begin{document}

\begin{tcolorbox}[colback=white!20,colframe=gray!75!black,title=Arbeitsauftrag:]
Stellen Sie Hume's gefühlethischen Ansatz, insbesondere seiner Begründung, dar
und nehmen Sie kritisch dazu Stellung.
\end{tcolorbox}

%\textbf{Quellen wo man recherchieren kann:}
%\begin{itemize}
%\item ``An Inquiry Concerning the Prniciples of Morals''
%\item Das obere ist eine fortsetzung seiner Untersuchung über den Verstand: ``An
  %Inquiry Concerning Human Understanding"
%\end{itemize}

%\textbf{Kritik an Humes}
%\begin{itemize}
  %\item Hume's Theory of Morals, von J.L. Mackie, dieser Artikel ist eine
    %wichtige kritische Darstellung von Humes Ethik und stellt Alternativen zur
    %Humschen Gefühls- und Empfindungsethik vor.
  %\item Hume's Moral Philosophy, von Terence Irwin, dieser Artikel beietet
    %einen umfassenden Überblick über Humes Ethik und stellt seine Ansichten in
    %den Kontext der philosophischen Disussionen seiner Zeit.
  %\item Emotion and Value in Hume's Moral Psychology, von Anette C. Baier
    %dieser Artikel diskutiert die Bedeutung von Emotionen in Humes
    %Moralpsychologie und argumentiert für eine stärkere Betonung der Rolle von
    %Emphatie und Mitgefühl in moralischen Entscheidungen.
  %\item  Hume on Reason and Virtue, von Philip Stratton-Lake, dieser Artikel
    %setzt sich mit Humes Verständnis von Vernunft und Tugend auseinander und
    %untersucht seine Ansichten im Hinblick auf die Diskussionen der
    %zeitgenösischen Ethik. 
%\end{itemize}
%Citation, dass ich eine Fehlermeldung mehr bekomm \cite{hume2000}.

%\section{Darstellung Hume's gefühlethischem Ansatz}
%David Hume vertrat eine gefühlsbasierte Ethik, die darauf basiert, dass
%moralische Urteile auf Gefühlen und Emotionen beruhen. Seine Begründung für
%diese Ansicht beruht auf seiner empiristischen Philosophie, die besagt, dass
%alle menschlichen Erkenntnisse auf Erfahrung und Beobachtung beruhen.
%
%Hume argumentiert, dass moralische Urteile nicht auf rein vernünftigen
%Überlegungen oder logischen Prinzipien beruhen können, sondern dass sie auf
%einem subjektiven Gefühl von Gut oder Schlecht basieren. Er behauptet, dass
%moralische Urteile nicht aus Fakten abgeleitet werden können, sondern dass sie
%auf subjektiven Einschätzungen basieren, die von individuellen Emotionen und
%Empfindungen beeinflusst werden.
%
%Humes Ansatz beruht auf der Vorstellung, dass moralische Bewertungen aufgrund
%der emotionalen Reaktionen entstehen, die wir auf Handlungen oder Ereignisse
%empfinden. Nach Hume können moralische Urteile nicht objektiv begründet werden,
%sondern sind das Ergebnis subjektiver Empfindungen. So können Handlungen, die
%für eine Person moralisch sind, für eine andere unmoralisch sein, je nach den
%individuellen Empfindungen und Vorlieben.
%
%Kritiker von Humes Ansatz behaupten, dass sein Ansatz zu einem moralischen
%Relativismus führt, der es schwierig macht, objektive moralische Maßstäbe zu
%finden. Wenn moralische Urteile allein auf subjektiven Empfindungen basieren,
%dann gibt es keine Möglichkeit, moralische Fragen objektiv zu klären oder
%moralische Normen für eine Gesellschaft zu etablieren. Darüber hinaus wird Humes
%Ansatz oft kritisiert, weil er moralischen Skeptizismus fördert und die
%Möglichkeit ausschließt, moralische Fortschritte zu machen. Wenn moralische
%Urteile ausschließlich auf subjektiven Empfindungen beruhen, gibt es keine
%Möglichkeit, moralischen Fortschritt zu messen oder moralische Fehler zu
%korrigieren.
%
%Zusammenfassend lässt sich sagen, dass Humes gefühlethischer Ansatz für die
%Ethik eine wichtige Perspektive darstellt, die die Rolle von Emotionen und
%subjektiven Empfindungen bei moralischen Urteilen betont. Allerdings ist dieser
%Ansatz auch umstritten, da er zu moralischem Relativismus und Skeptizismus
%führen kann. Es ist wichtig, kritisch zu hinterfragen, ob moralische Urteile
%tatsächlich ausschließlich auf subjektiven Empfindungen basieren oder ob es
%objektive Kriterien für moralische Urteile gibt, die unabhängig von
%individuellen Emotionen und Empfindungen sind.



\subsubsection*{Darstellung Hume's gefühlsethischer Ansatz}
Humes gefühlsethischer Ansatz geht davon aus, dass moralische Urteile nicht auf
vernünftigen Überlegungen beruhen, sondern auf Emotionen und Gefühlen. Er
argumentiert, dass die Moralität von Handlungen nicht durch die Handlungen
selbst bestimmt wird, sondern durch die Empfindungen, die sie bei den Menschen
auslösen.

Hume geht davon aus, dass menschliche Empfindungen und Gefühle durch Erfahrung
und Gewohnheit geprägt werden. So entwickeln sich unsere moralischen Urteile und
Werte aus unseren Erfahrungen und Beobachtungen der Handlungen anderer Menschen.
Dabei spielen Gefühle wie Empathie, Sympathie, Mitleid und Missbilligung eine
entscheidende Rolle.

Hume argumentiert weiter, dass moralische Urteile subjektiv sind und von Mensch
zu Mensch unterschiedlich ausfallen können. Es gibt also keine objektive Moral,
sondern nur individuelle moralische Vorstellungen, die auf unseren Emotionen und
Gefühlen basieren.

Auch betont Hume, dass die Vernunft lediglich dazu dienen kann, moralische
Entscheidungen zu begründen und zu erklären, aber nicht, um moralische Urteile
selbst zu fällen. Die Vernunft kann nur dabei helfen, unsere moralischen
Überzeugungen zu analysieren und zu reflektieren.

Insgesamt geht Hume davon aus, dass moralische Urteile nicht durch objektive
Faktoren bestimmt werden, sondern durch subjektive Empfindungen und Gefühle. Die
Vernunft kann dabei helfen, unsere moralischen Überzeugungen zu begründen und zu
erklären, aber letztendlich sind es unsere Emotionen und Gefühle, die unser
moralisches Urteilsvermögen bestimmen.


\subsubsection*{kritische Auseinandersetzung}
%David Humes gefühlethischer Ansatz in der Ethik basiert auf der Annahme, dass
%moralische Urteile und Werte auf Emotionen und Gefühlen basieren. Dabei
%argumentiert Hume, dass menschliche Empfindungen und Gefühle durch Erfahrung und
%Gewohnheit geprägt werden. So entwickeln sich unsere moralischen Urteile und
%Werte aus unseren Erfahrungen und Beobachtungen der Handlungen anderer Menschen.
%Dabei spielen Gefühle wie Empathie, Sympathie, Mitleid und Missbilligung eine
%entscheidende Rolle.

Jedoch gibt es einige kritische Ansätze gegenüber Humes gefühlethischem Ansatz.
Einer der Hauptkritikpunkte ist der Subjektivismus, der daraus folgt. Wenn
moralische Urteile lediglich auf persönlichen Gefühlen und Emotionen basieren,
dann gibt es keine Möglichkeit, objektive moralische Standards zu etablieren,
die für alle Menschen gelten. Ein weiterer kritischer Punkt ist, dass Humes
Ansatz Schwierigkeiten bei der Erklärung der moralischen Verpflichtung aufweist.
Wenn moralische Urteile lediglich auf Emotionen und Gefühlen basieren, dann gibt
es keine objektive Grundlage für moralische Verpflichtungen.

Ein weiterer Kritikpunkt an Humes Ansatz ist der Relativismus, der daraus folgt.
Wenn moralische Urteile auf individuellen Empfindungen und Gefühlen basieren,
dann gibt es keine allgemein gültigen moralischen Standards, die für alle
Menschen gelten. Dieser Relativismus könnte dazu führen, dass moralische
Entscheidungen nicht konsistent sind und dass moralische Regeln nicht
durchsetzbar sind.

Schließlich gibt es auch den Kritikpunkt, dass Humes Ansatz keine ausreichende
Rechtfertigung für moralische Werte bietet. Hume argumentiert, dass moralische
Urteile auf Empfindungen und Gefühlen basieren, aber er gibt keine ausreichende
Begründung dafür, warum bestimmte Emotionen und Gefühle als moralisch richtig
angesehen werden sollten.

Insgesamt kann Humes gefühlethischer Ansatz als problematisch angesehen werden,
da er Schwierigkeiten bei der Erklärung der moralischen Verpflichtung, der
objektiven moralischen Standards und der ausreichenden Rechtfertigung
moralischer Werte aufweist. Trotzdem bleibt sein Ansatz ein wichtiger Beitrag
zur Diskussion der ethischen Theorien und hat viele wichtige Impulse für die
moderne Ethik gegeben.



 \subsubsection*{Verlgeich mit Kant}
Eine (weitere Kritik) an Humes Ansatz der Gefühlsethik ist, dass er zu einem
moralischen Relativismus führen kann. Wenn moralische Werte lediglich auf
Emotionen und Empfindungen beruhen, dann gibt es keine objektiven Standards für
moralische Werte. Moralische Urteile sind dann letztendlich nur subjektive
Empfindungen und Meinungen, die von Person zu Person unterschiedlich sein
können.

Kants Ethik hingegen beruht auf der Idee, dass es objektive moralische Werte
gibt, die auf der Vernunft basieren. Diese Werte sind universell und gelten für
alle Menschen. Kant argumentiert, dass moralische Handlungen auf der Idee des
kategorischen Imperativs beruhen sollten, der besagt, dass eine Handlung nur
dann moralisch ist, wenn sie als allgemeines Gesetz gelten kann.

Eine Kritik an Kants Ethik ist, dass sie oft als zu rigide angesehen wird. Kant
betont die Bedeutung der Vernunft und des Gesetzes, aber vernachlässigt dabei
oft den Kontext und die individuellen Umstände, die bei moralischen
Entscheidungen berücksichtigt werden sollten.

Insgesamt sind Humes Ansatz der Gefühlsethik und Kants Ethik grundlegend
verschiedene Ansätze, die unterschiedliche Vorstellungen von moralischen Werten
und deren Begründung haben. Beide Ansätze haben jedoch ihre eigenen Vor- und
Nachteile, und es gibt keinen eindeutigen Sieger im Vergleich zwischen den
beiden Ansätzen.


test \cite{foot1996}


\printbibliography
\end{document}
