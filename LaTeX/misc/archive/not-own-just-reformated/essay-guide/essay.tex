\documentclass{article}
\usepackage[a4paper,margin=3cm]{geometry}
\usepackage{ulem}
\usepackage{url}
\usepackage{soul}
\usepackage{color}

\setlength{\parskip}{3pt} % set the parskip to 10pt
\setlength{\parindent}{0pt} % disable paragraph indentation
%\setlength{\parskip}{0.6\baselineskip} % set the parskip to the height of a line

\author{Jordan B. Peterson}
\title{Essay Writing Guide}
\date{}


\begin{document}
\maketitle

You can use this word document to write an excellent essay from
beginning to end, using a ten-step process. Most of the time, students
or would-be essay writers are provided only with basic information about
how to write, and most of that information concentrates on the details
of formatting. These are necessary details, but writing is obviously far
more than mere formatting. If you write your essay according to this
plan, and you complete every step, you will produce an essay that is at
least very good. You will also learn exactly how to write an essay,
which is something very valuable to learn.

To start writing your essay, go to the next page, for Part One:
Introduction.

\textit{Jordan B Peterson}

\tableofcontents
\clearpage

\section{Introduction}

\subsection{What is an essay?}

An essay is a relatively short piece of writing on a particular topic.
However, the word \emph{essay} also means attempt or try. An essay is,
therefore, a short piece written by someone attempting to explore a
topic or answer a question.

\subsection{Why bother writing an essay?}

Most of the time, students write essays only because they are required
to do so by a classroom instructor. Thus, students come to believe that
essays are important primarily to demonstrate their knowledge to a
teacher or professor. This is simply, and dangerously, wrong (even
though such writing for demonstration may be practically necessary).

\textbf{The primary reason to write an essay is so that the writer can
formulate and organize an informed, coherent and sophisticated set of
ideas about something important.}

Why is it important to bother with developing sophisticated ideas, in
turn? It's because there is no difference between doing so and thinking,
for starters. It is important to think because action based on thinking
is likely to be far less painful and more productive than action based
upon ignorance. So, if you want to have a life characterized by
competence, productivity, security, originality and engagement rather
than one that is nasty, brutish and short, you need to think carefully
about important issues. There is no better way to do so than to write.
This is because writing extends your memory, facilitates editing and
clarifies your thinking.

You can write down more than you can easily remember, so that your
capacity to consider a number of ideas at the same time is broadened.
Furthermore, once those ideas are written down, you can move them around
and change them, word by word, sentence by sentence, and paragraph by
paragraph. You can also reject ideas that appear substandard, after you
consider them more carefully. If you reject substandard ideas, then all
that you will have left will be good ideas. You can keep those, and use
them. Then you will have good, original ideas at your fingertips, and
you will be able to organize and communicate them.

Consider your success over the course of a lifetime. Here is something
to think about: the person who can formulate and communicate the best
argument almost always wins. If you want a job, you have to make a case
for yourself. If you want a raise, you have to convince someone that you
deserve it. If you are trying to convince someone of the validity of
your idea, you have to debate its merits successfully, particularly if
there are others with other competing ideas.

If you sharpen your capacity to think and to communicate as a
consequence of writing, you are better armed. The pen is mightier than
the sword, as the saying goes. This is no cheap cliché. Ideas change the
world, particularly when they are written. The Romans built buildings,
and the Romans and the buildings are both gone. The Jews wrote a book,
and they are still here, and so is the book. So it turns out that words
may well last longer than stone, and have more impact than whole
empires.

If you learn to write and to edit, you will also be able to tell the
difference between good ideas, intelligently presented, and bad ideas
put forth by murky and unskilled thinkers. That means that you will be
able to separate the wheat from the chaff (look it up). Then you can be
properly influenced by profound and solid ideas instead of falling prey
to foolish fads and whims and ideologies, which can range in their
danger from trivial to mortal.

Those who can think and communicate are simply more powerful than those
who cannot, and powerful in the good way, the way that means ``able to
do a wide range of things competently and efficiently.'' Furthermore,
the further up the ladder of competence you climb, with your
well-formulated thoughts, the more important thinking and communicating
become. At the very top of the most complex hierarchies (law, medicine,
academia, business, theology, politics) nothing is more necessary and
valuable. If you can think and communicate, you can also defend
yourself, and your friends and family, when that becomes necessary, and
it will become necessary at various points in your life.

Finally, it is useful to note that your mind is organized verbally, at
the highest and most abstract levels. Thus, if you learn to think,
through writing, then you will develop a well-organized, efficient mind
-- and one that is well-founded and certain. This also means that you
will be healthier, mentally and physically, as lack of clarity and
ignorance means unnecessary stress. Unnecessary stress makes your body
react more to what could otherwise be treated as trivial affairs. This
makes for excess energy expenditure, and more rapid aging (along with
all the negative health-related consequences of aging).

So, unless you want to stay an ignorant, unhealthy lightweight, learn to
write (and to think and communicate). Otherwise those who can will ride
roughshod over you and push you out of the way. Your life will be
harder, at the bottom of the dominance hierarchies that you will
inevitably inhabit, and you will get old fast.

Don't ever underestimate the power of words. Without them, we would
still be living in trees. So when you are writing an essay, you are
harnessing the full might of culture to your life. That is why you write
an essay (even if it has been assigned). Forget that, and you are doing
something stupid, trivial and dull. Remember it, and you are conquering
the unknown.

\subsection{A note on technology}

If you are a student, or anyone else who is going to do a lot of
writing, then you should provide yourself with the right technology,
especially now, when it is virtually costless to do so. Obviously, you
need a computer. It doesn't have to be that good, although a digital
hard drive is a good investment for speed. Less obviously, you need two
screens, one set up beside the other. They don't have to be bigger than
19'' diagonal. Even 17'' monitors will do well. High resolution is
better. You need the two screens so that you can present your reference
material on one screen, and your essay (or even two versions of your
essay, side by side) on the other.

Having this extra visual real estate really matters. It will make you
less cramped and more efficient. A good keyboard (such as the Microsoft
Natural Ergonomic keyboard) is also an excellent investment. Standard
keyboards will hurt your hands if you use them continually, and the less
said about a notebook keyboard the better. Use a good mouse, as well,
and not a touchpad, which requires too much finicky movement for someone
who is really working. Set up the keyboards so you are looking directly
at their centers when you are sitting up straight. Use a decent chair,
and sit so that your feet can rest comfortably on the floor when your
knees are bent 90 degrees. These are not trivial issues. You may spend
hours working on your writing, so you have to set up a workspace that
will not annoy you, or you will have just one more good reason to avoid
your tasks and assignments.

\subsection{A note on use of time}

People's brains function better in the morning. Get up. Eat something.
You are much smarter and more resilient after you have slept properly
and ate. There is plenty of solid research demonstrating this. Coffee
alone is counter-productive. Have some protein and some fat. Make a
smoothie with fruit and real yogurt. Go out and buy a cheap breakfast,
if necessary. Eat by whatever means necessary. Prepare to spend between
90 minutes and three hours writing. However, even 15 minutes can be
useful, particularly if you do it every day.

Do not wait for a big chunk of free time to start. You will never get
big chunks of free time ever in your life, so don't make your success
dependent on their non-existent. The most effective writers write every
day, at least a bit.

Realize that when you first sit down to write, your mind will rebel. It
is full of other ideas, all of which will fight to dominate. You could
be looking at Facebook, or Youtube, or watching or reading online porn,
or cleaning the dust bunnies from under your bed, or rearranging your
obsolete CD collection, or texting an old flame, or reading a book for
another course, or getting the groceries you need, or doing the laundry,
or having a nap, or going for a walk (because you need the exercise), or
phoning a friend or a parent -- the list is endless. Each part of your
mind that is concerned with such things will make its wants known, and
attempt to distract you. Such pesky demons can be squelched, however,
with patience. If you refuse to be tempted for fifteen minutes (25 on a
really bad day) you will find that the clamor in your mind will settle
down and you will be able to concentrate on writing. If you do this day
after day, you will find that the power of such temptations do not
reduce, but the duration of their attempts to distract you will
decrease. You will also find that even on a day where concentration is
very difficult, you will still be able to do some productive writing if
you stick it out.

Don't kid yourself into thinking you will write for six hours, either.
Three is a maximum, especially if you want to sustain it day after day.
Don't wait too late to start your writing, so you don't have to cram
insanely, but give yourself a break after a good period of sustained
concentration. Three productive hours are way better than ten hours of
self-deceptive non-productivity, even in the library.

\textbf{\hfill\break}

\section{Levels of Resolution}

\subsection{Words, sentences, paragraphs and more}

An essay, like any piece of writing, exists at multiple levels of
resolution, simultaneously. \ul{First} is the selection of the word­.
\ul{Second} is the crafting of the sentence. Each word should be
precisely the right word, in the right location in each sentence. The
sentence itself should present a thought, part of the idea expressed in
the paragraph, in a grammatically correct manner. Each sentence should
be properly arranged and sequenced inside a paragraph, the \ul{third}
level of resolution. As a rule of thumb, a paragraph should be made up
of at least 10 sentences or 100 words. This might be regarded as a
stupid rule, because it is arbitrary. However, you should let it guide
you, until you know better. You have very little right to break the
rules, until you have mastered them.

Here's a little story to illustrate that idea, taken in part from a
document called the \emph{Codex Bezae}.

\emph{Christ is walking down the road on the Sabbath, when good Jews of
that time were not supposed to work. In the ditch, he sees a shepherd,
trying to rescue a sheep from a hole that it has fallen into. It is very
hot and, clearly, the sheep will not be in very good shape if it spends
a whole day in the desert sun. On the other hand, it is the Sabbath.
Christ looks at the shepherd and says, ``Man, if indeed thou knowest
what thou doest, thou art blessed: but if thou knowest not, thou art
cursed, and a transgressor of the law.'' Then he walks on down the
road.}

The point is this: There is a rest day for a reason. Otherwise people
would work all the time. Then they would be chronically unhappy and
exhausted. They would compete each other to death. So if it's time for
everybody to rest, then rest, and don't be breaking the rule. However,
it is also not good to let a sheep die in the hot sun, when a few
minutes of labor might save it. So, if you are respectful of the rule,
and conscious of its importance, and realize that it serves as a bulwark
against the chaos of the unknown, and you still decide to break it,
carefully, because the particularities of the circumstances demand it --
well, then, more power to you. If you are just a careless, ignorant,
antisocial narcissist instead, however, then look out. You break a rule
at your peril, whether you know it or not.

Rules are there for a reason. You are only allowed to break them if you
are a master. If you're not a master, don't confuse your ignorance with
creativity or style. Writing that follows the rules is easier for
readers, because they know roughly what to expect. So rules are
conventions. Like all conventions, they are sometimes sub-optimal. But
not very often. So, to begin with, use the conventions. For example, aim
to make your paragraphs about 10 sentences or 100 words long.

A paragraph should present a single idea, using multiple sentences. If
you can't think up 100 words to say about your idea, it's probably not a
very good idea, or you need to think more about it. If your paragraph is
rambling on for 300 words, or more, it's possible that it has more than
one idea in it, and should be broken up.

All of the paragraphs have to be arranged in a logical progression, from
the beginning of the essay to the end. This is the \ul{fourth} level of
resolution. Perhaps the most important step in writing an essay is
getting the paragraphs in proper order. Each of them is a stepping stone
to your essay's final destination.

The \ul{fifth} level of resolution is the essay, as a whole. Every
element of an essay can be correct, each word, sentence, and paragraph
-- even the paragraph order -- and the essay can still fail, because it
is just not interesting or important. It is very hard for competent but
uninspired writers to understand this kind of failure, because a critic
cannot merely point it out. There is no answer to their question,
``exactly where did I make a mistake?'' Such an essay is just not good.
An essay without originality or creativity might fall into this
category. Sometimes a creative person, who is not technically proficient
as a writer, can make the opposite mistake: their word choice is poor,
their sentences badly constructed and poorly organized within their
paragraphs, their paragraphs in no intelligible relationship to one
another -- and yet the essay as a whole can succeed, because there are
valuable thoughts trapped within it, wishing desperately to find
expression.

\subsection{Additional levels}

You might think that there could not possibly be anything more to an
essay than these five levels of resolution or analysis, but you would be
wrong. This is something that was first noticed, perhaps, by those
otherwise entirely reprehensible and destructive scholars known as
post-modernists. An essay necessarily exists within a context of
interpretation, made up of the reader (level six), and the culture that
the reader is embedded in (level seven), which is made up in part of the
assumptions that he or she will bring to the essay. Levels six and seven
have deep roots in biology and culture. You might think, ``Why do I need
to know this?'' but if you don't you are not considering your audience,
and that's a mistake. Part of the purpose of the essay is to set your
mind straight, but the other part, equally important, is to communicate
with an audience.

For the essay to succeed, brilliantly, it has to work at all of these
levels of resolution simultaneously. That is very difficult, but it is
in that difficulty that the value of the act of writing exists.

\subsection{Considerations of Aesthetics and Fascination}

This is not all that has to be properly managed when you write an essay.
You should also strive for brevity, which is concise and efficient
expression, as well as beauty, which is the melodic or poetic aspect of
your language (at all the requisite levels of analysis). Finally, you
should not be bored, or boring. If you are bored while writing, then,
most importantly, you are doing it wrong, and you will also bore your
reader. Think of it this way: you get bored for a reason, and sometimes
for a good reason. You may be bored while writing your essay because you
are actually lying to yourself in a very deep way about what you are
doing and why you are doing it. Your mind, independent of your ego,
cannot be hoodwinked into attending to something that you think is
uninteresting or useless. It will automatically regard such a thing as
unworthy of attention, and make you bored by it.

If you are bored by your essay, you have either chosen the wrong topic
(one which makes no difference to you and, in all likelihood, to anyone
else) or you are approaching a good topic in a substandard manner.
Perhaps you are resentful about having to write the essay, or afraid of
its reception, or lazy, or ignorant, or unduly and arrogantly skeptical,
or something of the kind.

You have to place yourself in the correct state of mind to write
properly. That state of mind is partly aesthetic. You have to be trying
to produce something of worth, beauty and elegance. If you think that is
ridiculous, then you are far too stupid at the moment to write properly.
You need to meditate long and hard on why you would dare presume that
worth, beauty and elegance are unworthy of your pursuit. Do you plan to
settle for ugly and uncouth? Do you want to destroy, instead of build?

You must choose a topic that is important to you. This should be
formulated as a question that you want to answer. This is arguably the
hardest part of writing an essay: choosing the proper question. Perhaps
your instructor has provided you with a list of topics, and you think
you are off the hook as a consequence. You're not. You still have to
determine how to write about one of those topics in a manner that is
compelling to you. It's a moral, spiritual endeavour.

If you properly identify something of interest to you, then you have put
yourself in alignment with the deeper levels of your psyche, your
spirit. If these deeper levels do not want or need an answer to the
question you have posed, you will not possibly be interested in it. So
the fact of your interest is evidence of the importance of the topic.
You, or some part of you, needs the answer -- and such needs can be deep
enough so that life itself can depend upon them. Someone desperate, for
example, might find the question ``why live?'' of extreme interest, and
absolutely require an answer that makes life's suffering worth bearing.
It is not necessary to ensure that every question you try or essay to
answer of that level of importance, but you should not waste your time
with ideas that do not grip you.

So, the proper attitude is interested and aesthetically sensitive.

Having said all that, here is something to remember: \emph{finished}
beats \emph{perfect}. Most people fail a class or an assignment or a
work project not because they write badly, and geta D's or F's, but
because they don't write at all, and get zeroes. Zeroes are very bad.
They are the black holes of numbers. Zeroes make you fail. Zeroes ruin
your life. Essays handed in, no matter how badly written, can usually
get you at least a C. So don't be a completely self-destructive idiot.
Hand something in, regardless of how pathetic you think it is (and no
matter how accurate you are in that opinion).

\textbf{\hfill\break}

\section{The Topic and the Reading List}

The central question that you are trying to answer with the essay is the
topic question. Here are some potentially interesting topic questions:

\begin{itemize}
  \item Does evil exist?
  \item Are all cultures equally worthy of respect?
  \item How should a man and a woman treat each other in a relationship?
  \item What, if anything, makes a person good?
  \item These are very general, abstract topics. That makes them philosophical.
    Good topics do not have to be so general. Here are some good, more specific
    topics:
  \item What were the key events of Julius Caesar's rule?
  \item What are the critical elements of Charles Darwin's theory of evolution?
  \item Is ``The Sun Also Rises,'' by Ernest Hemingway, an important book?
  \item How might Carl Jung and Sigmund Freud's theory of the psyche be
    contrasted?
  \item How did Newton and Einstein differ in their conceptualization of time?
  \item Was the recent Iraq war just or unjust?
\end{itemize}

You can begin your essay writing process two different ways. You can
either list the topics you have been assigned, or list ten or so
questions that you might want to answer, if you are required to choose
your own topic, or you can start to create and finalize your reading
list. If you think you can already identify several potential topics of
interest, start with Topics. If you are unsure, then start constructing
your Reading List.

\emph{CHOICE BETWEEN TOPICS and READING LIST}

\subsection{Topics}

Put these in question form, as in the examples above.

1.

2.

3.

4.

5.

6.

7.

8.

9.

10.

If you can't do this, then you have to do some more reading (which you
will likely have to do to complete the essay anyway). There is, by the
way, no such thing as reader's block. If you can't write, it is because
you have nothing to say. You have no ideas. In such a situation, don't
pride yourself on your writer's block. Read something. If that doesn't
work, read something else -- maybe something better. Repeat until the
problem is solved.

\subsection{Reading List}

Indicate here what you have to or want to read. These should be books or
articles, generally speaking. If you don't know what articles or books
might be appropriate or useful, then you could start with Wikipedia
articles or other encyclopedic sources, and look at their reference
lists for ideas about further reading. These sources are fine as a
beginning.

If you find someone whose writing is particularly interesting and
appropriate, it is often very useful to see if you can find out what
authors they admired and read. You can do this by noting who they refer
to, in the text of their writings or in the reference list. You can
meander productively through wide bodies of learning in this manner.

Assume you need 5-10 books or articles per thousand words of essay,
unless you have been instructed otherwise. A double-spaced page of
typing usually contains about 250 words. List your sources now, even if
you have to do it badly. You can always make it better later.

Reading 1.

Notes: (see next section for Notes on Notes):

Reading 2.

Notes:

Reading 3.

Notes:

Reading 4.

Notes:

Reading 5.

Notes:

Reading 6.

Notes:

Reading 7.

Notes:

Reading 8.

Notes:

Reading 9.

Notes:

Reading 10 (repeat if necessary).

Notes (repeat if necessary):

\subsection{A Psychological Note and some Notes on Notes.}

While you are reading, see if you can notice anything that catches your
attention. This might be something you think is important, or something
that you seriously disagree with, or something that you might want to
know more about. You have to pay careful attention to your emotional
reactions to do this.

You also want to take some notes. You can place your notes below the
readings you listed above.

When you are taking notes, don't bother doing stupid things like
highlighting or underlining sentences in the textbook. There is no
evidence that it works. It just looks like work. What you need to do is
to read for understanding. Read a bit, then write down what you have
learned or any questions that have arisen in your mind. Don't ever copy
the source word for word. The most important part of learning and
remembering is the recreation of what you have written in your own
language. This is not some simplistic ``use your own words.'' This is
the dialog you are having with the writer of your sources. This is your
attempt to say back to the author ``this is what I understand you are
saying.'' This is where you extract the gist of the writing.

If someone asks you about your day, you don't say, ``Well, first I
opened my eyes. Then I blinked and rubbed them. Then I placed my left
leg on the floor, and then my right.'' You would bore them to death.
Instead, you eliminate the extra detail, and concentrate on
communicating what is important. That is exactly what you are supposed
to be doing when you take some notes during or after reading (after is
often better, with the book closed, so that you are not tempted to copy
the author's writing word for word so that you can fool yourself into
thinking you did some work).

If you find note-taking in this manner difficult, try this. Read a
paragraph. Look away. Then say to yourself, out loud, even in a whisper
(if you are in a library), what the paragraph meant. Listen to what you
said, and then quickly write it down.

Take about two to three times as many notes, by word, as you will need
for your essay. You might think that is inefficient, but it's not. In
order to write intelligibly about something, or to speak intelligently
about it, you need to know far more than you actually communicate. That
helps you master level six and seven, described previously -- the
context within which the essay is to be understood. Out of those notes
you should be able to derive 8-10 topic questions. Do so. Remember, they
can be edited later. Just get them down.

\textbf{\hfill\break}

\section{The Outline}

At this point you have prepared a list of topics, and a reading list.
Now it's time to choose a topic.

\emph{ENTER TOPIC HERE}

1.

Here's another rule. When you write your first draft, it should be
longer than the final version. This is so that you have some extra
writing to throw away. You want to have something to throw away after
the first draft so that you only have to keep what is good. It is NOT
faster to try to write exactly as many words as you need when you first
sit down to write. Trying to do so merely makes you too aware of what
you are writing. This concern will slow you down. Aim at producing a
first draft that is 25\% longer than the final draft is supposed to be.
If your final work is to be 1000 words, then write that (or four pages)
below. The word document will automatically add 25\% to the length you
specify.

Now specify the length of your essay.

\emph{WORDS:}

\emph{PAGES:}

\emph{ADD 25\% TO THE ABOVE LENGTHS}

Now you have to write an outline. \ul{This is the most difficult part of
writing an essay}, and it's not optional. The outline of an essay is
like the skeleton of a body. It provides its fundamental form and
structure. Furthermore, the outline is basically the argument (with the
sentences themselves and the words serving that argument).

A thousand-word essay requires a ten-sentence outline. However, the
fundamental outline of an essay should not get much longer than fifteen
sentences, even if the essay is several thousand words or more in
length. This is because it is difficult to keep an argument of more than
that length in mind at one time so that you can assess the quality of
its structure. So, write a ten to fifteen sentence outline of your
essay, and if it is longer than a thousand words, then make sub-outlines
for each primary outline sentence. Here is an example of a good simple
outline:

\begin{itemize}
  \item Topic: Who was Abraham Lincoln?
  \item Why is Abraham Lincoln worthy of remembrance?
  \item What were the crucial events of his childhood?
  \item Of his adolescence?
  \item Of his young adulthood?
  \item How did he enter politics?
  \item What were his major challenges?
  \item What were the primary political and economic issues of his time?
  \item Who were his enemies?
  \item How did he deal with them?
  \item What were his major accomplishments?
  \item How did he die?
\end{itemize}

Here is an example of a good longer outline (for a three thousand word
essay):

\begin{itemize}
  \item Topic: What is capitalism?
  \item How has capitalism been defined?
    \subitem Author 1
    \subitem Author 2
    \subitem Author 3
  \item Where and when did capitalism develop?
    \subitem Country 1
    \subitem Country 2
  \item How did capitalism develop in the first 50 years after its origin?
    \subitem How did capitalism develop in the second 50 years after its origin?
    \subitem (Repeat as necessary)
  \item Historical precursors?
    \subitem (choose as many centuries as necessary)
  \item Advantages of capitalism?
    \subitem Wealth generation
    \subitem Technological advancement
    \subitem Personal freedom
  \item Disadvantages of capitalism?
    \subitem Unequal distribution
    \subitem Pollution and other externalized costs
  \item Alternatives to capitalism?
    \subitem Fascism
    \subitem Communism
  \item Consequences of these alternatives?
  \item Potential future developments?
  \item Conclusion
\end{itemize}

Beware of the tendency to write trite, repetitive and clichéd
introductions and conclusions. It is often useful to \emph{write} a
stock intro (what is the purpose of this essay? How is it going to
proceed?) and a stock conclusion (How did this essay proceed? What was
its purpose?) but they should usually then be thrown away. Write your
outline here. Try for one outline heading per 100 words of essay length.
You can add subdivisions, as in the example regarding capitalism, above.

Write outline here:

1. Outline sentence 1:

2. Outline sentence 2:

3. Outline sentence 3:

4. Outline sentence 4:

5. Outline sentence 5:

6. Outline sentence 6:

7. Outline sentence 7:

8. Outline sentence 8:

9. Outline sentence 9:

10. Outline sentence 10 (repeat if necessary):

\textbf{\hfill\break}

\section{Paragraphs}

So, now you have your outline. Copy it here:

\emph{OUTLINE COPIED HERE}

Now, write ten to fifteen sentences per outline heading to complete your
paragraph. You may find it helpful to add additional subdivisions to
your outline, and to work back and forth between the outline and the
sentences, editing both. Use your notes, as well. Use single spacing at
this point, so that you can see more writing on the paper at once. You
will format your essay properly later.

Don't worry too much about how well you are writing at this point. It is
also best at this point not to worry too much about the niceties of
sentence structure and grammar. That is all best left for the second
major step, which is editing. You should think of the essay writing
process as twofold. The first major step is the first draft, which can
be relatively quick and dirty. For the first draft you can use your
notes, extensively, and rough out the essay. If you get stuck writing
anywhere, just move to the next outline sentence. You can always go
back.

The second major step is editing. Production (the first major step) and
editing (the second) are different functions, and should be treated that
way. This is because each interferes with the other. The purpose of
production is to produce. The function of editing is to reduce and
arrange. If you try to do both at the same time then the editing stymies
the production. It's not faster to combine them, nor is it better, and
it is bound to be frustrating.

Here is an example of writing associated with an outline question:
(note: places where references are necessary are indicate as (REFERENCE,
19XX). How to format these references will be discussed later.

\textit{\ul{Outline sentence: How has capitalism been defined?}}

\begin{quote}
\emph{Something as complex as capitalism cannot be easily defined.
Different authors have each offered their opinion. Liberal or
conservative thinkers stress the importance of private property and the
ownership rights that accompany such property as key to capitalism
(REFERENCE, 19XX). Such private property (including valuable goods and
the means by which they are produced) can be traded, freely, with other
property owners, in a market where the price is set by public demand,
rather than by any central agency. Liberal and conservative thinkers
stress efficiency of production, as well as quality, and consider profit
the motive for efficiency. They believe that lower cost is a desirable
feature of production, and that fair competition helps ensure desirably
lower prices.}

\emph{The World Socialist Movement (REFERENCE, 19XX), an international
consortium of far left political parties, defines capitalism, by
contrast, as ownership of the means of production by a small minority of
people, the capitalist class, who profitably exploit the working class,
the genuine producers, who must sell their ability to work for a salary
or wage. Such socialists believe that it is profit that solely motivates
capitalism, and that the profit motive is essentially corrupt. Modern
environmentalists tend to add the natural world itself to the list of
capitalist targets of exploitation (REFERENCE, 19XX). Thinkers on the
right tend to regard problems emerging from the capitalist system as
real, but trivial in comparison to those produced by other economic and
political systems, real and hypothetical. Thinkers on the far left
regard capitalism as the central cause of problems as serious as
poverty, inequality and environmental degradation, and believe that
there are other political and economic systems whose implementation
would constitute an improvement.}
\end{quote}

It took two paragraphs to begin to address the first outline sentence.
Notice that the essay begins without referring to itself. It is better
to tell the reader what the essay will be about and how the topic will
be addressed than to meander around stupidly at the beginning of an
essay, but it is still better to grab the reader's attention immediately
without beating around the bush.

Once you have completed ten to fifteen sentences for each outline
heading, you have finished your first draft. Now it is time to move to
editing.

\section{Editing and Arringing of Sentences Within Paragraphs}

Copy the first paragraph of your first draft here:

Paragraph 1:

Now, place each sentence on its own line, so it looks like this (this
example is taken from the first paragraph on capitalism, above):

\emph{Something as complex as capitalism cannot be easily defined.}

\emph{Different authors have each offered their opinion.}

\emph{Liberal or conservative thinkers stress the importance of private
property and the ownership rights that accompany such property as key to
capitalism (REFERENCE, 19XX).}

\emph{Such private property (including valuable goods and the means by
which they are produced) can be traded, freely, with other property
owners, in a market where the price is set by public demand, rather than
by any central agency.}

\emph{Liberal and conservative thinkers stress efficiency of production,
as well as quality, and consider profit the motive for efficiency.}

\emph{They believe that lower cost is a desirable feature of production,
and that fair competition helps ensure desirably lower prices.}

\hl{Now, write another version of each sentence, under each sentence,
like this:}

\emph{Liberal and conservative thinkers stress efficiency of production,
as well as quality, and consider profit the motive for efficiency.}

Liberal and conservative thinkers alike stress the importance of quality
and efficiency, and see them as properly rewarded by profit.

In this example, the meaning of the sentence has been changed slightly,
during the rewrite. It may be that the second sentence flows better than
the first, and is also more precise and meaningful. See if you can make
each sentence you have written better, in a similar manner:

\begin{itemize}
\item Better would mean shorter and simpler (as all unnecessary words should be
  eliminated). There is almost nothing a novice writer can do that will improve
  his or her writing more rapidly than writing very short sentences. See if you
  can cut the length of each sentence by 15-25\%. Remember, earlier, you tried
  to make your essay longer than necessary. Here you can start cleaning it up.

\item Better would mean that each word is precisely and exactly the right word.
  Don't be tempted to use any word that you would be uncomfortable to use in
  spoken conversation. Often, new writers try to impress their readers with
  their vocabulary. This often backfires when words are used that are
  technically correct but whose connotation is not, or that are unsuitable
  within the context of the sentence, paragraph or full essay. An expert writer
  will spot such flaws immediately, and see them for what they are: forms of
  camouflage and deception. Write clearly at a vocabulary level you have
  mastered (with maybe a bit of stretching, to produce improvement).
\end{itemize}

Read each sentence aloud, and listen to how it sounds. If it's awkward,
see if you can say it a different, better way. Listen to what you said,
and then write it down. Rewrite each sentence. Once you have done this
with all the sentences, read the old versions and the new versions, and
replace the old with the new if the new is better. Then copy the new
paragraph here:

New paragraph 1:

Repeat for each paragraph:

New paragraph 2:

New paragraph 3:

New paragraph 4:

New paragraph 5 (etc.):

Now you are going to try to improve each of those paragraphs. Copy them
again here, unchanged (you are doing this so that you can easily compare
the improved paragraphs to the originals, so that you can be sure they
are truly improved, before you keep them):

New paragraph 1 (copy):

New paragraph 2 (copy):

New paragraph 3 (copy):

New paragraph 4 (copy):

New paragraph 5 (copy) (etc.):

Start with paragraph 1. Break it up into single sentences, as you did
before. Now check to see if the sentences are in the best possible
order, within each paragraph. Drag and drop them, or cut and paste them,
into better order.

You can also eliminate sentences that are no longer necessary. When you
are satisfied with the first paragraph (which means that the sentences
are necessary, short and punchy, and in the correct order) then go ahead
to the next paragraph and do the same thing.

\section{Re-Ordering the Paragraphs}

Now, copy all of the new, improved paragraphs that you have edited here:

New improved paragraph 1:

New improved paragraph 2:

New improved paragraph 3:

New improved paragraph 4:

New improved paragraph 5 (etc.):

Now you are going to try to improve the order of those new, improved
paragraphs. Copy them here, again, unchanged.

New improved paragraph 1 (copy):

New improved paragraph 2 (copy):

New improved paragraph 3 (copy):

New improved paragraph 4 (copy):

New improved paragraph 5 (copy) (etc.):

Now look at the order of the paragraphs themselves (as you just did with
the sentences within each paragraph). It may well be that by now in the
editing process, you will find that the order of the subtopics within
your original outline is no longer precisely appropriate, and that some
re-ordering of those sub-topics is called for. So, move around the new
improved paragraph (copies) above, until they are ordered more
appropriately than they were.

\textbf{\hfill\break}

\section{Generating a new Outline}

So now you should have produced a pretty decent second draft. You have
identified the appropriate sources, written the proper notes, outlined
your argument, roughed in a first draft (paragraph by paragraph),
rewritten your sentences to make them more elegant, and re-ordered those
sentences, as well as the paragraphs themselves. This is much farther
than most writers ever get. You may even think you're finished -- but
you're not.

The next step will take you from a ``B'' essay to an ``A'' essay. It may
even help you write something that is better than you have ever produced
(better meaning richer in information, precise, coherent, elegant and
beautiful). Copy what you have written so far here:

\emph{FULL RE-ORDERED ESSAY HERE}

Read it. Then go to the next page.

This part of the process will probably strike you as unnecessary, or
annoying, or both, but what do you know? This is the step that separates
the men from the boys, or the women from the boys, or the men from the
girls, or whatever version of this saying is acceptably non-sexist and
politically correct.

\hl{\hfill\break}

You have just finished reading your essay. Try now to write a new
outline of ten to fifteen sentences. \textbf{Don't look back at your
essay while you are doing this.} If you have to, go back and re-read the
whole thing, and then return to this page, but don't look at your essay
while you are rewriting the outline. If you force yourself to
reconstruct your argument from memory, you will likely improve it.
Generally, when you remember something, you simplify it, while retaining
most of what is important. Thus, your memory can serve as a filter,
removing what is useless and preserving and organizing what is vital.
What you are doing now is distilling what you have written to its
essence.

Write new outline here:

1. New outline sentence 1:

2. New outline sentence 2:

3. New outline sentence 3:

4. New outline sentence 4:

5. New outline sentence 5:

6. New outline sentence 6:

7. New outline sentence 7:

8. New outline sentence 8:

9. New outline sentence 9:

10. New outline sentence 10 (repeat if necessary):

Now that you have a new outline, you can cut and paste material from
your previous essay. To do this, open up a new Word document beside this
one. Then cut and paste the new outline that you have written into the
new Word document. Return to the original document, and scroll up to the
full, re-ordered essay you copied and pasted into Part Eight, above.
Then cut and paste from the re-ordered essay into your new outline.

You may find that you don't need everything you wrote before. Don't be
afraid to throw unnecessary material away. You are trying to get rid of
what is substandard, and leave only what is necessary.

Once you have finished cutting and pasting your old material into the
new outline, then copy the new essay, and paste it into a new word
document. That will be your final essay. Don't forget to put a title
page on it.

\emph{PASTE NEWLY OUTLINED ESSAY HERE:}

\section{Repeat}

Now you have a third draft, and it's probably pretty good. If you really
want to take it to the next level, then you can repeat the process of
sentence rewriting and re-ordering, as well as paragraph re-ordering and
re-outlining. Often it is a good idea to wait a few days to do this, so
that you can look at what you have produced with fresh eyes. Then you
will be able to see what you have written, instead of seeing what you
think you wrote (which is the case when you try to edit immediately
after producing).

You are not genuinely finished until you cannot edit so that your essay
improves. Generally, you can tell if this has happened when you try to
rewrite a sentence (or a paragraph) and you are not sure that the new
version is an improvement over the original.

\section{References and Bibliogrpahy}

When you write a sentence that contains what is supposed to be a fact or
at least an informed opinion, and you have picked it up from something
you read, then you have to refer to that source. Otherwise, following
convention, people may accuse you of plagiarism, which is a form of
theft (of intellectual property). There are a large number of
conventions that you can follow to properly structure your references
and your bibliography (which is a list of books and articles that you
have read to obtain relevant background information, but from which you
may not have drawn any ideas specific enough to require a reference).

The conventions of the American Psychological Association (APA) are
commonly used by essay writers. This convention generally requires the
use of the last names of the authors of the source in parentheses after
the sentence requiring a reference. For example:

\begin{quote}
It is necessary to add a reference after a sentence containing an
opinion which is not your own, or a fact that you have acquired from
some source material (Peterson, 2014).
\end{quote}

This sentence could also be constructed like this:

\begin{quote}
Peterson (2014) claims that it is necessary to add a reference
after a sentence containing an opinion which is not your own, or a fact
that you have acquired from some source material.
\end{quote}

There are also many conventions covering the use of a direct quote,
which have to be followed when you directly quote someone, rather than
paraphrasing them. Here is an example, adding the specific (fictional)
number of the page containing the quoted material in the original
manuscript:

\begin{quote}
Peterson (2014, p. 19) claims that ``the conventions of the
American Psychological Association (APA) are commonly used by essay
writers.''
\end{quote}

In the Reference List, at the end of the essay, Peterson's paper might
be listed, as follows (this is a fictional reference):

\emph{Peterson, J.B. (2014). Essay writing for writers. Journal of Essay
Writing, 01, 15-24.}

Different conventions hold for different types of source material such
as webpages, books, and articles. All the details regarding APA style
can be found at \url{http://www.apastyle.org/}

Your instructor may have recommended, or demanded, use of a different
set of conventions. Information about other techniques and rules can be
found at \url{http://www.easybib.com/reference/guide/mla/general}.

It is necessary to master at least one convention. The rules are finicky
and annoying. However, they are necessary, so that readers know what
writers are up to. Furthermore, you only have to learn them once, so
bite the bullet and do it.

Copy your essay here again.

Add references where they are necessary. Then, add your reference list
to the end of your essay. Make sure you construct both according to APA
convention, or some other set of rules.

\textbf{YOUR COMPLETED ESSAY}

Now your essay is completed. Now you need to copy it into a new Word
document, and format it properly.

That generally means double-spaced, with a title page, with a five space
tab indent at the beginning of each paragraph. If you want to add
subtitles, or section headers, their use is discussed in detail at
\url{https://owl.english.purdue.edu/owl/resource/560/01/} . Additional
useful information for style, including examples, can be found at
\url{http://bit.ly/ZC5eFV} . A video discussing such matters is
available at \url{http://bit.ly/ZpX1nR} .

If you got this far, good work. If you write a number of essays using
this process, you will find that your thinking will become richer and
clearer, and so will your conversation. There is nothing more vital to
becoming educated, and there is nothing more vital than education to
your future, and the future of those around you

Good luck with your newly organized and refreshed mind.

\end{document}
