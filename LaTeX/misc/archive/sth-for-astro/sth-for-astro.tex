\documentclass[11p]{article}
\usepackage[a4paper,left=2.5cm,right=2.5cm,top=3cm,bottom=3cm]{geometry}
\usepackage[utf8]{inputenc}
\usepackage[T1]{fontenc}
\usepackage{textcomp}
\usepackage[german]{babel} %i guess for a german dictionary
\usepackage{amsmath, amssymb} %math symbols
\usepackage{soul} %to strike stuff thru
\usepackage{cancel} %another one to cancel, hopefully this works
\usepackage{graphicx}
\usepackage{tikz}
\usepackage{pgfplots}
\pgfplotsset{compat=1.18}

% figure support
\usepackage{import}
\usepackage{xifthen}
\pdfminorversion=7
\usepackage{pdfpages}
\usepackage{transparent}

\begin{document}
\section*{So, i still have to learn \LaTeX and also need to practice writing limits
here, so i can also just explain to you how limits work.}	

A limit basically measures the slope at a curve at a point (of a function), so
the tangent. a limit is written.
\[\lim_{n \rightarrow 0}\]
n approaching 0 basically, that means the variable n approaches 0, meaning two
points of a secant are getting so close, it could nearly be a tangent. I hope
you know tangent/secant, if not just look them up it's pretty easy. Tangent is
the curve of a function shown with a line that crosses the function once, a
secant crosses a function twice. (that probably makes no sense)
\\
this can be used e.g. here:
\[\lim_{h \rightarrow 0} \frac{f(x_0+h)-f(x_0)}{h}\] 
you might know this thing for normally getting the slope of a curve, then
written:
\[m=\frac{y_2 -y_1}{x_2 -x_1}\] 
i have no idea how to explain this here: basically the top of the fractions are
the same because f(x) is the same as a y, the +h is the difference between $y_1$ 
and $y_2$ in the fractions right above, and the h as a denomenator (i hope it's
spelled this way) is the difference between the ys which is the same as $x_2$ -
$x_1$. the $\lim_{h\rightarrow 0}$ saying $x\rightarrow h$, means h is getting closer and closer to 0,
to the point that it is this close, it could even be zero.
Goal is always to plug 0 in, which cant be dont while h is in the denomenator.
so most times the first step is to plug the values and functions you have in and
cross out the denomenator.
\\
\\
\Large{EX}\\
\normalsize
Find the equation of the tangent line to y=$\frac{3}{x}$ at (3,1).\\
Here we already have $x_0=3, f(x)=\frac{3}{x}$ and $f(x+h)=\frac{3}{3+h}$ 

\begin{center}
	\[\lim_{h \rightarrow 0} \frac{f(x_0+h)-f(x_0)}{h}\] \\
	$\downarrow$ here you just enter the given variables\\
	\[\lim_{h \rightarrow 0} \frac{f(3+h)-f(3)}{h}\] here you solve the functions \\
	$\downarrow$\\
	\[\lim_{h \rightarrow 0} \frac{\frac{3}{3+h}-1}{h}\]\\
	$\downarrow$\\
	\[\lim_{h \rightarrow 0} \frac{\frac{3}{3+h}-\frac{3+h}{3+h}}{h}\] the 3+h
	fraction is exactly the same as 1 because nominator and dominator are the same\\
	$\downarrow$\\
	\[\lim_{h \rightarrow 0} \frac{\frac{3-3-h}{3+h}}{h}\]\\
	that above is just some algebra, i will not explain it because i would not
	know how.\\
	$\downarrow$\\
\[\lim_{h \rightarrow 0} \frac{\frac{-h}{3+h}}{\frac{h}{1}}\]\\
	again some basic algebra, h and $\frac{h}{1}$ are the same.\\
	$\downarrow$\\
	\[\lim_{h \rightarrow 0}\frac{\cancel{-h}}{3+h}*\frac{1}{\cancel{h}} \]\\
	here just some crossing out\\
	$\downarrow$\\
	\[\lim_{h \rightarrow 0}\frac{-1}{3+h} \]\\
	now you have the h not alone in the denomenator and you can insert the h
	(=0)\\
	$\downarrow$\\
	\[\frac{-1}{3+0} = -\frac{1}{3}\]\\
soo, here you know have the slope of a curve at a point, so exactly what the
goal was, go get the tangent equation you now just set in your slope into the
line formula.\\
\clearpage
	\[y-y_1 =m(x-x_1)		\]
	\[y-1 =-\frac{1}{3}(x-3)		\]
	\[y-1 =-\frac{1}{3}x+1\]
	\[y=-\frac{1}{3}x+2\]
	So that's your formula, let me illustrate it:

%%%%%%%%%%%%%%%%%%%%%%%%%%%%%%GRAPHS%%%%%%%%%%%%%%%%%%%%%%%%%%%
\begin{tikzpicture}
\begin{axis}
[
    xlabel={x}, 
    ylabel={y}
]
\addplot[grid, no marks, smooth, domain=-10:10, samples=50] {3/x};
\addlegendentry{f(x)=3/x}

\addplot[no marks, domain=-10:10, samples=50] {-0.333333333*x+2};
\addlegendentry{g(x)=-1/3x+2}

\end{axis}
\end{tikzpicture}




\end{center}
%%%%%%%%%%%%%%%%%%%%%%%%%%%%%%%%%%%%%%%%%%%%%%%%%%%%%%%%%%%%%%%
So that is the exact slope of the curve at the point (3,1). That is all limits
are there for, at least i think and hope so. On the pic you see: red graph (the
main function), blue graph (the tangent, which was solved for) and lastly green
point is the point where the slope is searched.

\end{document}
