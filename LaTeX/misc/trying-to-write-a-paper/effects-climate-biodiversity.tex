\documentclass{article}
\usepackage{biblatex}
\usepackage[a4paper,margin=3.5cm]{geometry}

\addbibresource{bio.bib}

\title{What are the effects of climate change on biodiversity?}
\author{Daniel Renschler}
\begin{document}

I got this question from~\cite{chatgpt}

These sources can provide you with some background information and scientific
evidence related to climate change and its effects on biodiversity. To write a
scientific paper on this topic, you may want to consider the following
sections:

1. Introduction: Introduce the topic of climate change and its potential impact
  on biodiversity. Provide a clear research question and objectives.

2. Literature review: Summarize and evaluate the key findings from the sources
  listed above, along with other relevant literature on the topic.

3. Methods: Describe the methods used to conduct your own research, such as
  data collection, analysis, and interpretation.

4. Results: Present the findings of your research, along with relevant data and
  statistics.

5. Discussion: Interpret your findings and draw conclusions regarding the
  effects of climate change on biodiversity. Discuss the limitations of your
  study and areas for future research.

6. Conclusion: Summarize the main findings of your paper and emphasize their
  implications for biodiversity conservation efforts.

I hope this helps you get started on writing your scientific paper on climate
change and biodiversity!

These sources can provide you with some background information and scientific
evidence related to climate change and its effects on biodiversity. To write a
scientific paper on this topic, you may want to consider the following
sections:

1. Introduction: Introduce the topic of climate change and its potential impact
  on biodiversity. Provide a clear research question and objectives.

2. Literature review: Summarize and evaluate the key findings from the sources
  listed above, along with other relevant literature on the topic.

3. Methods: Describe the methods used to conduct your own research, such as
  data collection, analysis, and interpretation.

4. Results: Present the findings of your research, along with relevant data and
  statistics.

5. Discussion: Interpret your findings and draw conclusions regarding the
  effects of climate change on biodiversity. Discuss the limitations of your
  study and areas for future research.

6. Conclusion: Summarize the main findings of your paper and emphasize their
  implications for biodiversity conservation efforts.

I hope this helps you get started on writing your scientific paper on climate
change and biodiversity!

\printbibliography

\end{document}
