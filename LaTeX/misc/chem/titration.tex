\documentclass{report}
\usepackage[a4paper, left=2.5cm, right=2.5cm, top=2cm, bottom=3cm]{geometry}
\usepackage{chemfig}
\setlength{\parindent}{0pt}


\begin{document}

\begin{titlepage}
\centering

\LARGE{Labor Bericht: Neutralisations-Titration}\\
\vspace{20pt}
\large{Till Severin und Linus Blum}\\
\vspace{20pt}
\large{Abgabe Datum: 03.03.2023}\\
\vspace{20pt}
\large{Probe-Nummer 6}\\

\end{titlepage}

\section{R/S S\"atze:} % (fold)
\label{sec:R/S S"atze:}
\textbf{Natronlauge:}

R 35: Verursacht schwere Verätzungen

S1/2: Unter Verschluss und für Kinder unzugänglich aufbewahren.

S 26: Bei Berührung mit den Augen gründlich mit Wasser abspülen und Arzt konsultieren.

S 27: Beschmutzte, getränkte Kleidung sofort ausziehen. S 29: Nicht in die Kanalisation gelangen lassen.

S 35: Abfälle und Behälter müssen in gesicherter Weise beseitigt werden.

S 36: Bei der Arbeit geeignete Schutzkleidung tragen.

S 37: Geeignete Schutzhandschuhe tragen.

S 39: Schutzbrille/Gesichtsschutz tragen.

\vspace{20pt}
\textbf{Salzs\"aure}

R 34: Verursacht Verätzungen.

R 35: Verursacht schwere Verätzungen.

R 36: Reizt die Augen.

R 37: Reizt die Atmungsorgane.

S1/2: Unter Verschluss und für Kinder unzugänglich aufbewahren.

S 26: Bei Berührung mit den Augen gründlich mit Wasser abspülen und Arzt konsultieren.

S 27: Beschmutzte, getränkte Kleidung sofort ausziehen.) S 29: Nicht in die Kanalisation gelangen lassen.

S 35: Abfälle und Behälter müssen in gesicherter Weise beseitigt werden.

S 36: Bei der Arbeit geeignete Schutzkleidung tragen.

S 37: Geeignete Schutzhandschuhe tragen.

S 39: Schutzbrille/Gesichtsschutz tragen.

\vspace{20pt}
\textbf{Materialien:}

Stativ, Bürettenklammer, Bürette, Vollpipette, Messkolben, Erlenmeyerkolben,
Trichter, destilliertes Wasser, Salzsäure, Natronlauge, Indikator

\vspace{20pt}
\textbf{Versuchsaufbau}\\
Bild1
Versuchsdruchführung :
Zur Vorbereitung des Versuches werden zunächst 2-3 Tropfen des Indikators
(Phenolphtalein) in die Probelösung hinzugefügt, der den erreichten
Neutralisationspunkt/Äquivalenzpunkt zwischen der Base und der Säure durch einen
Farbumschlag anzeigen soll. Die Maßlösung wird dann zu 25 ml in die Bürette
gefüllt. Die Probelösung sollte unter der Bürette neben dem Fuß des Stativs
stehen. Um den Farbumschlag des Äquivalenzpunktes besser zu erkennen, ist es
hilfreich ein weißes Blatt Papier unter die Probelösung zu platzieren. Der
Versuch kann nun beginnen. Der kleine Hahn der Bürette wird vorsichtig geöffnet,
damit die Maßlösung langsam in die darunter platzierte Probelösung tropfen kann.
Während dieses Vorgangs wird die Maßlösung dauerhaft umgerührt oder bewegt,
damit sich die beiden Lösungen gut vermischen. Sobald der Indikator einen pinken
Ton erreicht wird der Hahn der Bürette verschlossen. Im Folgenden kann die Menge
der verbrauchten Maßlösung an den Messstrichen der Bürette abgelesen werden.
Danach kann die fertige und vollständig neutralisierte Lösung in den Abfluss
entsorgt, und die Materialien wieder für den zweiten Durchgang vorbereitet
werden. Im zweiten Versuch wird dann im Unterschied zum ersten Versuch ein
anderer Indikator verwendet. Es werden wieder 2-3 Tropfen des neuen Indikators
in die Salzsäure Lösung hinzugefügt und der Versuch beginnt von Vorne.


\vspace{20pt}
\textbf{Versuchsbeobachtung:}\\

Nach dem Beginn des einlassen der Maßlösung, behält die Probelösung für eine
lange Zeit ihre klare durchsichtige Färbung, ändert dann jedoch schlagartig ihre
Färbung erst von einem blassen rosa, welches dann mit wenigen weiteren Tropfen
eine stärkere rosa Färbung bekommt. Der Farbumschlag erfolge bei der ersten
Durchführung bei 4ml (ABBILDUNG)-/ und bei der zweiten Durchführung bei
6.5ml(ABBILDUNG 2) Maßlösung.


\vspace{20pt}
\textbf{Erkl\"arung:}\\
Die Probe ändert aufgrund des Indikators schlagartig ihre Farbe auf rosa. Dies
zeigt, dass die “Base(die Maßlösung) die überhand gewonnen hat”. Phenolphtalein
verursacht als Reaktion darauf die rosa Färbung. Dies wird in der unten
folgenden Rechnung sowie in der Titrationskurve veranschaulicht.




% section section name (end)


















\end{document}
