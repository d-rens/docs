\lecture{1}{25-05-2023}{test}

\chapter{(5.1) -- Area between two curves}


\begin{figure}[ht]
    \centering
    \incfig[0.7]{area-between-curves-1}
    \caption{area-between-curves-1}
    \label{fig:area-between-curves-1}
\end{figure}

Area between $f(x)\quad \& \quad g(x)$.

\[ A= \text{Area under $f(x)$ - Area under $g(x)$} .\] 

\[ A= \int^b_a f(x) dx - \int^b_a g(x) dx .\] 

\[
    A=\int^b_a \left[f(x)-g(x)\right] dx
.\] 

\nt{
    \[ f(x) \geq g(x)\] 
    \[ \forall x \in [a,b] .\] 
    \[ \left( f(x) \text{ is above } g(x)\right) .\] 
}


\begin{figure}[htbp]
    \centering
    \incfig[0.3]{are-between-curves-w}
    \caption{are-between-curves-2}
    \label{fig:are-between-curves-2}
\end{figure}

\[ \int_{a}^{c} \left[ f(x)-0 \right] dx + \int_{c}^{b}  \left[ 0-f(x) \right] dx \] 
\[ \int_{a}^{c}  f(x) dx - \int_{c}^{b} f(x) dx \] 

\ex{Find the area bounded above by $y=2x+5$, and bounded below by $y=x^3$ on $\left[ 0,2 \right]$ }



\begin{align*}
A &=\int_{0}^{2}  (2x+5) - x^3 \text{dx}\\
&= \int_{0}^{2}2x+5=x^3 dx   \\
&= x^2+5x - \frac{x^4}{4}]^2_0  \\
A &= \left[ 2^2+5\cdot 2 \frac{2^4}{4} \right] - \left[ 0 \right]   \\
A&= 10 \\
\end{align*}

\ex{Find area between $y=x^2$ and  $y=x+6$.}

\begin{figure}[htbp]
    \centering
    \incfig[0.5]{area-between-curves-3}
    \caption{area-between-curves-3}
    \label{fig:area-between-curves-3}
\end{figure}    

\nt{
\textbf{Steps:}
\begin{enumerate}
    \item Find x-cords of the Intersection of the curves. ( Set $f(x)=g(x)$ ) 
    \item Which function is on the top? \\
        (Pick one point for each interval)
    \item set-up and solve
\end{enumerate}
}   

\begin{align*}
    x+6&= x^2 \\
    x^2-x-6&= 0 \\
    \left( x-3 \right)&  \left( x+2 \right) \\ 
    x-3 = 0 \quad &\quad  x+2 = 0\\
    x=3 \quad&\quad x=-2
\end{align*}

Those are the only places $f(x)$ and  $g(x)$ are intercepting, so those are the bounds of integration.
\[
A=\int_{-2}^{3}  
.\] 


\begin{figure}[htbp]
    \centering
    \incfig[0.4]{area-between-curves-4}
    \caption{figures/area-between-curves-4}
    \label{fig:area-between-curves-4}
\end{figure}

\begin{align*}
    A&= \int_{-2}^{3} \left( x+6 \right) -(x^2) \text{dx} \implies \int_{-2}^{3} x+6-x^2 \text{dx}   \\
    &= \left[ \frac{x^2}{2}+6x -\frac{x^3}{3}\right]_{-2}^3 = \left[
    \frac{3^2}{2}+6\cdot 3-\frac{3^3}{3} \right] - \left[
\frac{(-2)^2}{2}+6(-2)-\frac{-2^3}{3} \right]  \\
&= \left[  \frac{9}{2}+18-9  \right]-\left[ 2-12+\frac{8}{3} \right]=\left[ \frac{9}{2}+9 \right] -\left[ -10+\frac{8}{3} \right]  \\
&= \frac{27}{2}-\frac{-22}{3}\to \frac{27}{2}+\frac{22}{3}= \frac{125}{6} \\
\end{align*}


\ex{
Find area bound by $y=x^3$ and  $y=x$.
}

\begin{align*}
    x^3&= x \\
    x^3-x&= 0 \\
    x(x^2-1)&= 0 \\
    x(x+1)(x-1)&=0 \\
    x=0 ,\quad & x=1, \quad  x=-1
\end{align*}

\regel{Here we test again which function is where above the other, to know, in which
direction to integrate. (always the one on the top minus the one on the bottom,
as shown at the start of this chapter.) }

\begin{figure}[htbp]
    \centering
    \incfig[0.4]{area-between-curves-5}
    \caption{figures/area-between-curves-5}
    \label{fig:area-between-curves-5}
\end{figure}

Here we add those integrals together, because we want both ares together.

\begin{align*}
    A&= \int_{-1}^{0} x^3-x \ \text{dx} + \int_{0}^{1} x-x^3 \ \text{dx}   \\
    &= \left[  \frac{x^4}{4}-\frac{x^2}{2} \right]^0_{-1} + \left[ \frac{x^2}{2}-\frac{x^4}{4} \right]^1_0  \\
    &= \left[ \left( \frac{0}{4}- \frac{0}{2} \right) - \left( \frac{(-1)^4}{4}- \frac{(-1)^2}{2} \right)  \right]+ 
    \left[ \left( \frac{1^2}{2}- \frac{1^4}{4} \right)- \left( \frac{0}{2}- \frac{0}{4} \right)   \right]  \\
    &= \left[ 0-\left( \frac{1}{4} - \frac{1}{2} \right)  \right] + \left[ \left( \frac{1}{2}-\frac{1}{4} \right) -0 \right]  \\
    &= \frac{1}{4}+\frac{1}{4}= \frac{1}{2} \\
\end{align*}


\begin{figure}[ht]
    \centering
    \incfig[0.5]{area-between-curves-6}
    \caption{figures/area-between-curves-6}
    \label{fig:area-between-curves-6}
\end{figure}

In this case, the area represents the distance car 'me' is ahead of car 'you'.

This can then be solved by:

\[
A=\int_{0}^{b} V_1(t) - V_2(t) \ \text{dt}
.\] 

\ex{
Find the area bound by $x=y^2$ and $y=x-2$.
}

\begin{figure}[ht]
    \centering
    \incfig[0.5]{area-between-curves-7}
    \caption{figures/area-between-curves-7}
    \label{fig:area-between-curves-7}
\end{figure}

\begin{align*}
    x=y^2 \quad & \quad y=x-2\\
    x=y^2 \quad & \quad x=y+2\\
    y^2&= y+2 \\
       &\downarrow \\
    y^2-y-2&= 0 \\
    (y-2)(y+1)&= 0 \\
    y-2=0 \quad & \quad y+1=0 \\
    y=2 \quad & \quad y=-1 \\
\end{align*}

Now we have to decide with what respect to integrate, we can use both, but need
to be sure on which on to use.

So we take $x=y^2$, make out of it $\pm \sqrt{x} = \sqrt{y^2} $, out of this we then get $\pm \sqrt{x}=y$.

\nt{I got really confused at this point in the lecture, please notice that
$y=\pm \sqrt{x}$ and  $x=y^2$ are basically the same thing, can recommend to
check it out on desmos.}

%In terms of 'x' there would be three integrals to take, so we do it in terms of 'y'.

%First we need to get the coordinates for the intercepts, just on the different axis.
%$\leadsto x=4$, $\leadsto x=1$.

Now we simplify the graph, to make it easier to understand.

\begin{figure}[ht]
    \centering
    \incfig[0.6]{area-between-curves-8}
    \caption{figures/area-between-curves-8}
    \label{fig:area-between-curves-8}
\end{figure}


\begin{figure}[ht]
    \centering
    \incfig[0.5]{area-between-curves-9}
    \caption{figures/area-between-curves-9}
    \label{fig:area-between-curves-9}
\end{figure}

\begin{align*}
    A&= \int_{0}^{1} \sqrt{x}-(-\sqrt{x}) \ \text{dx} + \int_{1}^{4} \sqrt{x}-(x-2)\ \text{dx}   \\
     &= \int_{0}^{1} 2\sqrt{x} \ \text{dx} + \int_{1}^{4} \sqrt{x} - x + 2 \ \text{dx}    \\
     &= 2 \int_{0}^{1} x^{\frac{1}{2}}\ \text{dx}+ \int_{1}^{4} x^{\frac{1}{2}}-x+2 \ \text{dx}   \\
     &= \left[2\frac{x^{\frac{3}{2}}}{\frac{3}{2}}\right]_0^1+ \left[\frac{x^{\frac{3}{2}}}{\frac{3}{2}}-\frac{x^2}{2}+2x\right]_1^4 \\
     &= \left[\frac{4x^{\frac{3}{2}}}{\frac{3}{2}}\right]_0^1 + \left[ \frac{2x^{\frac{3}{1}}}{3}-\frac{x^2}{2}+2x \right]_1^4 \\
     &     \downarrow  \quad\text{evaluating}\quad \downarrow \\
     &= \left( \frac{4\cdot 1\cdot \frac{3}{1}}{3}-0 \right)+ \left[\left(
     \frac{2(4)^{\frac{3}{1}}}{3}- \frac{4^2}{2}+2\cdot 4  \right)-\left( \frac{2}{3}-\frac{1}{2}+2 \right)  \right]   \\
     &= \frac{4}{3}+\left[ \left(\frac{16}{3}-8+8\right) -\frac{13}{6}\right]  \\
     &= \frac{4}{3} + \frac{16}{3}-\frac{13}{6}= \frac{27}{6} \\
     &=\frac{9}{2}\\
\end{align*}

Now doing it with the other axis, in respect to 'y'.

\begin{figure}[ht]
    \centering
    \incfig[0.5]{area-between-curves-10}
    \caption{figures/area-between-curves-10}
    \label{fig:area-between-curves-10}
\end{figure}

\[
A= \int_{c}^{d} h(y)-g(y) \text{dy} 
.\] 

(If $h(y) \geq g(y)$ for [c,d].]

Now doing it with the prior example:

\begin{align*}
    x=y^2 \quad & \quad y=x-2\\
    y^2=& y+2\\
    y^2-y-2 &= 0 \quad \to \quad y=2,-1 \\ 
\end{align*}

\begin{figure}[ht]
    \centering
    \incfig[0.5]{area-between-curves-11}
    \caption{Area Between Curves 11}
    \label{fig:area-between-curves-11}
\end{figure}

\begin{align*}
    A &= \int_{-1}^{2} y+2-y^2 \ dy \\
      &= \left[ \frac{y^2}{2}+2y = \frac{y^3}{3}\right]_{-1}^2\\
      &= \frac{9}{2} \\
\end{align*}



\chapter{(5.2) -- Volume of solids by disks and washers method}

\section{Volume of solids by slicing}

\begin{figure}[ht]
    \centering
    \incfig[0.5]{disks-1}
    \caption{figures/disks-1}
    \label{fig:disks}
\end{figure}

Cut into thin slabs. Then use Summations to set up an integration.

To do this, find area of cross section.


\begin{figure}[ht]
    \centering
    \incfig[0.4]{disks-2}
    \caption{figures/disks-2}
    \label{fig:disks-2}
\end{figure}        

Here we can see that is on an axis or some like that:

\[ V=\left( y\cdot z \right) \cdot x .\] 

Where $y\cdot x$ is the surface area of the cross seaction and $x$ the length.

Now we can find the volume of any solid that is bound my planes $\perp$ to x-axis.
At points 'a' and 'b'.

\vspace{20pt}

\begin{figure}[htbp]
    \centering
    \incfig[0.6]{disks-3}
    \caption{figures/disks-3}
    \label{fig:disks-3}
\end{figure}
Cut into slabs with a width of $\Delta x$.

\begin{figure}[ht]
    \centering
    \incfig[0.6]{disks-4}
    \caption{figures/disks-4}
    \label{fig:disks-4}
\end{figure}

Pick arbitrary point '$x_{k.}$', on each sub-interval. Find Cross-sectional Area at '$x_{k.}$'.

\[ V_k=A(x_{k.})\cdot \Delta x \] 
Where $A(x_{k.}$ is the cross-sectional area and $\Delta x$ the length.
\[V=\sum^n_{k=1} A(x_{k.}\cdot \Delta x \quad  \text{(Approx V)} \] 
Because we don't want to have an approximation:
\begin{align*}
    V&= \lim_{n\to\infty} \sum^n_{k=1}A(x_{k.})\Delta x \\
    V&= \int_{a}^{b} A(x) \ \text{dx}  \\
\end{align*}

$A(x)$ = cross sectional area over [a,b]

\vspace{50pt}


\begin{figure}[ht]
    \centering
    \incfig[0.6]{disks-5}
    \caption{figures/disks-5}
    \label{fig:disks-5}
\end{figure}

\[V=\int_{c}^{d} A(y) dy\] 

\begin{figure}[ht]
    \centering
    \incfig[0.7]{disks-6}
    \caption{figures/disks-6}
    \label{fig:disks-6}
\end{figure}

Some things to the picture

\begin{align*}
    A &= \pi\cdot r \\
    A(x) &= \pi(1)=\pi \\
    V&= \int_{1}^{5} \pi dx  \\
\end{align*}

To get the volume we can either take the Integral.
\[ V=\pi x ]^5_1=\pi\cdot 5-\pi\cdot 1=4\pi .\] 

Or in this case also do it without calculus, but given that it's just a simple
task, and tasks wont stay simple, we did it with calculus.

\clearpage
\section{Solid of revolution}

First some examples, the following are all still do-able with just some
algebra, but the point to get across is, it can be done with every functino
there is, at least most of em, e.g. one could do it with $x^2$ in a definite
integral, by just doing 
\[ \pi\cdot \int_{-1}^{1} x^2 dx .\] 

\begin{figure}[ht]
    \centering
    \incfig{disks-7}
    \caption{figures/disks-7}
    \label{fig:disks-7}
\end{figure}

f(x) is :\\
continuous and bounded by ``a'' and ``b'' x=a, x=b\\
rotate about x-axis, sides are $\perp$ to x-axis.


\begin{figure}[ht]
    \centering
    \incfig[0.5]{disks-8}
    \caption{figures/disks-8}
    \label{fig:disks-8}
\end{figure}

\paragraph{Find Volume by slicing:}
\[ V= \int_{a}^{b} A(x) dx .\] 

\[ A(x)=Surface area of cross-section \] 
\begin{align*}
    A(x)&=\pi r^2, \quad \text{circle}\ r=f(x)\\
    A(x)&= \pi \left[ f(x) \right]^2 \\
    V&= \int_{a}^{b} \pi \left[ f(x) \right]^2  \\
\end{align*}
\begin{center}
$\uparrow$ Method of disks.
\end{center}

\ex{ Find the volume of the solid of revolution where $y=3 \sqrt{x}$ on $[1,4]$ is revolved about x-axis.

    First finding the integral:
    \begin{align*}
        V &=  \int_{a}^{b}\pi\left[ f(x) \right]^2 dx \\
          &= \pi \cdot \int_{1}^{4} [3\sqrt{x}]^2 dx\\
          &= \pi \cdot \int_{1}^{4} 9x\ dx \\
          &= 9\pi \int_{1}^{4} x\ dx = 9\pi\cdot \frac{x^2}{2}]_1^4 = \frac{9\pi}{2}[x^2]_1^4 = \frac{9\pi}{2}[4^2-1^2] = \frac{9\pi}{2}[15]=\frac{135\pi}{2}  \\
    \end{align*} }

\subsection{Derive Volume of a Sphere}

\begin{figure}[ht]
    \centering
    \incfig[0.85]{disks-9}
    \caption{figures/disks-9}
    \label{fig:disks-9}
\end{figure}

\nt{paused at 1:22h}










\end{document}
